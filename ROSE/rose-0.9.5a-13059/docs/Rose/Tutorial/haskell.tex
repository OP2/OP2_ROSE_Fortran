\chapter{The Haskell Interface}

ROSE's Haskell interface allows the user to analyse and
transform the Sage III IR from Haskell, a statically typed pure
functional programming language.  See \htmladdnormallink{\tt
http://www.haskell.org/}{http://www.haskell.org/}.

The interface exposes almost all Sage III APIs to Haskell, allowing
the user to call whichever APIs are required.  The interface also
supports an AST traversal mechanism inspired by Haskell's {\em scrap
your boilerplate} design pattern.

The Haskell interface also provides a convenient mechanism for a
user to rapidly experiment with the ROSE IR.  GHC's command-line
interpreter {\tt ghci} can be used to explore the IR interactively
by invoking API methods at will.

The Haskell interface relies on the Glasgow Haskell Compiler (GHC).
It is auto-configured so long as the GHC binaries are in your {\tt
\$PATH}.  If not, you will need to supply the path to the binaries
at configure time with the option {\tt --with-haskell=}{\em bindir},
where {\em bindir} is the path to GHC's {\tt bin} directory.

After installation, ROSE is available as a standard Haskell package
named {\tt rose}.  This means that you can supply the flag {\tt
-package rose} to the Haskell compiler in order to make the extension
available for use.

To understand the usage of the interface, it is
crucial to grasp how the concept of {\em monads} works
in Haskell.   For a useful tutorial on monads, the reader
is referred to the ``All About Monads'' tutorial found at
\htmladdnormallink{\tt http://www.haskell.org/all\_about\_monads/}{http://www.haskell.org/all\_about\_monads/}.

The simplest Haskell-based ROSE program is the identity translator,
whose code is listed in Figure~\ref{Tutorial:haskellIdentityTranslator}.

\begin{figure}[!h]
{\indent
{\mySmallFontSize

% Do this when processing latex to generate non-html (not using latex2html)
\begin{latexonly}
   \lstinputlisting[language=Haskell]{\TOPSRCDIR/projects/haskellport/tests/identityTranslator/Main.hs}
\end{latexonly}

% Do this when processing latex to build html (using latex2html)
\begin{htmlonly}
   \verbatiminput{\TOPSRCDIR/projects/haskellport/tests/identityTranslator/Main.hs}
\end{htmlonly}

% end of scope in font size
}
% End of scope in indentation
}
\caption{Haskell version of identity translator.}
\label{Tutorial:haskellIdentityTranslator}
\end{figure}

\section{Traversals}

As previously mentioned, the traversal mechanism is inspired by
the scrap-your-boilerplate pattern.  Our implementation of the
scrap-your-boilerplate pattern provides both {\em transformation} and
{\em query} traversals.  A transformation traversal applies a global
transformation to a tree by applying a given function to each tree
node, whereas a query traversal derives a result from a tree using
a function that produces a result from a node together with a {\em
combinator} which combines the results from several nodes (for example
in a summation query, the combinator may be the addition function).

In order to carry out a traversal, two steps are necessary. Firstly
one must build a {\em type extension}, a type-generic function built
from one or more type-specific functions.  Secondly one must employ
a {\em generic traversal combinator} which applies the type extension
throughout the program.

In our interface type extensions for transformations are built
using the functions {\tt mkMn}, which builds a type extension from
a type-specific function, and {\tt extMn}, which extends an existing
type extension with a type-specific function.  Likewise {\tt mkMqn}
and {\tt extMqn} for queries.  These functions perform static and
dynamic type checking such that they will only call the type-specific
functions when it is safe to do so.

The two generic traversal combinators are {\tt everywhereMc} and {\tt
everythingMc}.  They take two arguments: the type extension and the
tree to be traversed. {\tt everywhereMc} returns the transformed tree,
and {\tt everythingMc} the result of the query.

Tying everything together, Figure~\ref{Tutorial:haskellSimplify}
shows an example of a simple constant folding transformation.

\begin{figure}[!h]
{\indent
{\mySmallFontSize

% Do this when processing latex to generate non-html (not using latex2html)
\begin{latexonly}
   \lstinputlisting[language=Haskell]{\TOPSRCDIR/projects/haskellport/tests/simplify/Main.hs}
\end{latexonly}

% Do this when processing latex to build html (using latex2html)
\begin{htmlonly}
   \verbatiminput{\TOPSRCDIR/projects/haskellport/tests/simplify/Main.hs}
\end{htmlonly}

% end of scope in font size
}
% End of scope in indentation
}
\caption{Haskell version of constant folding transformation.}
\label{Tutorial:haskellSimplify}
\end{figure}

\section{Further Reading}

Reference documentation for the interface is available on ROSE's
website at:

\htmladdnormallink{\tt http://www.rosecompiler.org/ROSE\_HaskellAPI/}{http://www.rosecompiler.org/ROSE\_HaskellAPI/}
