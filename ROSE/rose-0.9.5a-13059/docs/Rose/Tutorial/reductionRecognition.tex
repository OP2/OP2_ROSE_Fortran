\chapter{Reduction Recognition}

Figures~\ref{Tutorial:exampleReductionRecognition} 
shows a translator which
finds the first loop of a main function and recognizes reduction operations
and variables within the loop.  
A reduction recognition algorithm (\lstinline{ReductionRecognition()}) is implemented in the SageInterface
namespace and follows the C/C++ reduction restrictions defined in the
OpenMP 3.0 specification. 

%-------------translator---------------------
\begin{figure}[!h]
{\indent
{\mySmallestFontSize


% Do this when processing latex to generate non-html (not using latex2html)
\begin{latexonly}
   \lstinputlisting{\TutorialExampleDirectory/reductionRecognition.C}
\end{latexonly}

% Do this when processing latex to build html (using latex2html)
\begin{htmlonly}
   \verbatiminput{\TutorialExampleDirectory/reductionRecognition.C}
\end{htmlonly}

% end of scope in font size
}
% End of scope in indentation
}
\caption{Example source code showing reduction recognition.}
\label{Tutorial:exampleReductionRecognition}
\end{figure}


Using this translator we can compile the code shown in 
figure~\ref{Tutorial:exampleInputCode_reductionRecognition}.  
The output is shown in figure~\ref{Tutorial:exampleOutput_reductionRecognition}.

%-------------input code---------------------
\begin{figure}[!h]
{\indent
{\mySmallFontSize
% Do this when processing latex to generate non-html (not using latex2html)
\begin{latexonly}
   \lstinputlisting{\TutorialExampleDirectory/inputCode_reductionRecognition.C}
\end{latexonly}

% Do this when processing latex to build html (using latex2html)
\begin{htmlonly}
   \verbatiminput{\TutorialExampleDirectory/inputCode_reductionRecognition.C}
\end{htmlonly}

% end of scope in font size
}
% End of scope in indentation
}
\caption{Example source code used as input to loop reduction recognition processor.}
\label{Tutorial:exampleInputCode_reductionRecognition}
\end{figure}

%-------------output---------------------

\begin{figure}[!h]
{\indent
{\mySmallFontSize
% Do this when processing latex to generate non-html (not using latex2html)
\begin{latexonly}
   \lstinputlisting{\TutorialExampleBuildDirectory/reductionRecognition.out}
\end{latexonly}

% Do this when processing latex to build html (using latex2html)
\begin{htmlonly}
   \verbatiminput{\TutorialExampleBuildDirectory/reductionRecognition.out}
\end{htmlonly}

% end of scope in font size
}
% End of scope in indentation
}
\caption{Output of input to reduction recognition processor.}
\label{Tutorial:exampleOutput_reductionRecognition}
\end{figure}



