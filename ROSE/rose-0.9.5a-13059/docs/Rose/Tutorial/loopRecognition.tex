\chapter{Recognizing Loops}

   Figures~\ref{Tutorial:exampleLoopRecognition-part1} and
\ref{Tutorial:exampleLoopRecognition-part2} show a translator which
reads an application and gathers a list of loop nests.  At the end of the traversal it
reports information about each loop nest, including the function where it occurred
and the depth of the loop nest.

\fixme{This example program is unfinished. It will output a list of objects representing
       information about perfectly nested loops.}

\begin{figure}[!h]
{\indent
{\mySmallestFontSize


% Do this when processing latex to generate non-html (not using latex2html)
\begin{latexonly}
   \lstinputlisting{\TutorialExampleBuildDirectory/loopRecognition.aa}
\end{latexonly}

% Do this when processing latex to build html (using latex2html)
\begin{htmlonly}
   \verbatiminput{\TutorialExampleDirectory/loopRecognition.C}
\end{htmlonly}

% end of scope in font size
}
% End of scope in indentation
}
\caption{Example source code showing loop recognition (part 1).}
\label{Tutorial:exampleLoopRecognition-part1}
\end{figure}

\begin{figure}[!h]
{\indent
{\mySmallestFontSize


% Do this when processing latex to generate non-html (not using latex2html)
\begin{latexonly}
   \lstinputlisting{\TutorialExampleBuildDirectory/loopRecognition.ab}
\end{latexonly}

% Do this when processing latex to build html (using latex2html)
\begin{htmlonly}
   \verbatiminput{\TutorialExampleDirectory/loopRecognition.C}
\end{htmlonly}

% end of scope in font size
}
% End of scope in indentation
}
\caption{Example source code showing loop recognition (part 2).}
\label{Tutorial:exampleLoopRecognition-part2}
\end{figure}


   Using this translator we can compile the code shown in 
figure~\ref{Tutorial:exampleInputCode_LoopRecognition}.  The 
output is shown in figure~\ref{Tutorial:exampleOutput_LoopRecognition}.

\begin{figure}[!h]
{\indent
{\mySmallFontSize


% Do this when processing latex to generate non-html (not using latex2html)
\begin{latexonly}
   \lstinputlisting{\TutorialExampleDirectory/inputCode_LoopRecognition.C}
\end{latexonly}

% Do this when processing latex to build html (using latex2html)
\begin{htmlonly}
   \verbatiminput{\TutorialExampleDirectory/inputCode_LoopRecognition.C}
\end{htmlonly}

% end of scope in font size
}
% End of scope in indentation
}
\caption{Example source code used as input to loop recognition processor.}
\label{Tutorial:exampleInputCode_LoopRecognition}
\end{figure}

\begin{figure}[!h]
{\indent
{\mySmallFontSize


% Do this when processing latex to generate non-html (not using latex2html)
\begin{latexonly}
   \lstinputlisting{\TutorialExampleBuildDirectory/loopRecognition.out}
\end{latexonly}

% Do this when processing latex to build html (using latex2html)
\begin{htmlonly}
   \verbatiminput{\TutorialExampleBuildDirectory/loopRecognition.out}
\end{htmlonly}

% end of scope in font size
}
% End of scope in indentation
}
\caption{Output of input to loop recognition processor.}
\label{Tutorial:exampleOutput_LoopRecognition}
\end{figure}



