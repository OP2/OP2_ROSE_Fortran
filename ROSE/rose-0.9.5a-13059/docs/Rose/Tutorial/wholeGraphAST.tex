\chapter{Advanced AST Graph Generation}
%\chapter{General AST Graph Generation}
\label{Tutorial:chapterGeneralASTGraphGeneration}
\fixme{This chapter brings more confusions. I suggest to remove it until it
is done right . -Leo}
\paragraph{What To Learn From This Example}
This example shows a maximally complete representation of the AST 
(often in more detail that is useful).

Where chapter~\ref{Tutorial:chapterASTGraphGenerator}
presented a ROSE-based translator which presented the AST as a tree, this chapter
presents the more general representation of the graph in which the AST is embedded.
The AST may be thought of as a subset of a more general graph or equivalently as
an AST (a tree in a formal sense) with annotations (extra edges and information),
sometimes referred to as a `{\em decorated} AST.

We present tools for seeing all the IR nodes in the graph containing the AST, including all types 
(SgType nodes), symbols (SgSymbol nodes), compiler generated IR nodes, and 
supporting IR nodes. In general
it is a specific filtering of this larger graph which is more useful to communicating
how the AST is designed and internally connected.  We use these graphs for internal 
debugging (typically on small problems where the graphs are reasonable in size).
The graphs presented using these graph mechanism present all back-edges, and
demonstrate what IR nodes are shared internally (typically SgType IR nodes).

First a few names, we will call the AST those nodes in the IR that are 
specified by a traversal using the ROSE traversal (SgSimpleTraversal, etc.).
We will call the graph of all IR nodes the {\em Graph of all IR nodes}.
the AST is embedded in the {\em Graph of all IR nodes}.  The AST {\em is} a tree,
while the {\em graph of all IR nodes} typically not a tree (in a Graph Theory sense)
since it typically contains cycles.

We cover the visualization of both the AST and the {\em Graph of all IR nodes}.
\begin{itemize}
   \item AST graph \\
   These techniques define ways of visualizing the AST and filtering IR nodes from being represented.
   \begin{itemize}
      \item Simple AST graphs
      \item Colored AST graphs
      \item Filtering the graph \\
      The AST graph may be generated for any subtree of the AST (not possible for the
    graphs of all IR nodes).  Additionally runtime options permit null pointers to be
    ignored. \fixme{Is this true?}.
   \end{itemize}
   \item {\em Graph of all IR nodes} \\
   These techniques define the ways of visualizing the whole graph of IR nodes and is
    based on the memory pool traversal as a means to access all IR nodes.  Even
    disconnected portions of the AST will be presented.
   \begin{itemize}
      \item Simple graphs
      \item Colored graphs
      \item Filtering the graph
   \end{itemize}
\end{itemize}

% DQ (4/14/2010): I have removed this overly complex example for now!
% we want to present the newer mechanism to tailoring the whole AST graphs using switches.
\fixme{Removed this example since the newer mechanism for tailoring the whole AST graphs needs to be presented.}
\commentout{
\section{Whole Graph Generation}

This example shows how to generate and visualize the AST from any input program.
Each node of the graph in figure~\ref{tutorial:exampleOutputWholeGraphAST} shows
a node of the Intermediate Representation (IR).  Each edge shows the connection
of the IR nodes in memory. The generated graph shows the connection of different 
IR nodes to form the AST.

\begin{figure}[!h]
{\indent
{\mySmallFontSize

% Do this when processing latex to generate non-html (not using latex2html)
\begin{latexonly}
   \lstinputlisting{\TutorialExampleDirectory/wholeGraphAST.C}
\end{latexonly}

% Do this when processing latex to build html (using latex2html)
\begin{htmlonly}
   \verbatiminput{\TutorialExampleDirectory/wholeGraphAST.C}
\end{htmlonly}

% end of scope in font size
}
% End of scope in indentation
}
\caption{Example source code to read an input program and generate an AST graph.}
\label{Tutorial:exampleWholeGraphAST}
\end{figure}

The program in figure~\ref{Tutorial:exampleWholeGraphAST} calls 
an internal ROSE function that traverses the AST and generates 
an ASCII file in {\tt dot} format.
Figure~\ref{Tutorial:exampleInputCode_WholeGraphAST} shows an input
code which is processed to generate a graph of the AST, generating a 
{\tt dot} file.   The {\tt dot} file is then processed
using {\tt dot} to generate a postscript file~\ref{tutorial:exampleOutputWholeGraphAST}
(within the {\tt Makefile}).
Note that a similar utility program already exists within ROSE/exampleTranslators
(and includes a utility to output an alternative PDF representation 
(suitable for larger ASTs) as well).  Figure~\ref{tutorial:exampleOutputWholeGraphAST}
(\TutorialExampleBuildDirectory/test.ps) can be found in the compile 
tree (in the tutorial directory) and viewed directly using ghostview 
or any postscript viewer to see more detail.


\begin{figure}[!h]
{\indent
{\mySmallFontSize

% Do this when processing latex to generate non-html (not using latex2html)
\begin{latexonly}
   \lstinputlisting{\TutorialExampleDirectory/inputCode_wholeGraphAST.C}
\end{latexonly}

% Do this when processing latex to build html (using latex2html)
\begin{htmlonly}
   \verbatiminput{\TutorialExampleDirectory/inputCode_wholeGraphAST.C}
\end{htmlonly}

% end of scope in font size
}
% End of scope in indentation
}
\caption{Example source code used as input to generate the AST graph.}
\label{Tutorial:exampleInputCode_WholeGraphAST}
\end{figure}

\begin{figure}
%\centerline{\epsfig{file=\TutorialExampleBuildDirectory/wholeGraphAST.ps,
%                    height=1.3\linewidth,width=1.0\linewidth,angle=0}}
\includegraphics[scale=0.5]{\TutorialExampleBuildDirectory/wholeGraphAST}
\caption{AST representing the source code file: inputCode\_wholeGraphAST.C.}
\label{tutorial:exampleOutputWholeGraphAST}
\end{figure}

   Figure~\ref{tutorial:exampleOutputWholeGraphAST} displays the individual
C++ nodes in ROSE's intermediate representation (IR).  Each circle represents
a single IR node, the name of the C++ construct appears in the center of the
circle, with the edge numbers of the traversal on top and the number of
child nodes appearing below.  Internal processing to build the graph generates
unique values for each IR node, a pointer address, which is displays at the bottom
of each circle.  The IR nodes are connected for form a tree, and abstract syntax
tree (AST). Each IR node is a C++ class, see SAGE III reference for details,
the edges represent the values of data members in the class (pointers which connect
the IR nodes to other IR nodes).  The edges are labeled with the names of the 
data members in the classes representing the IR nodes.
}

