\section{Definitions}

%  \item Classification of security flaws \\
% \subsection{Classification of Security Flaws}
         The development of seeding for security flaws is addressed using a specific 
         classification for security flaws: simple (local) security flaws, 
         security flaw chains (defined over the control flow) and composite security flaws
         (defined via state over the execution of the whole program).
         We define these according to discussions at NIST of classification work by MITRE.
      \begin{enumerate}
         \item Simple Security Flaws \\ 
               A simple security flaw is defined as an isolated single security flaw
               which can be locally defined in an application source code. It can be
               roughly characterized as a security flaw for which the fix is a local 
               modification.

         \item Security Flaw Chains \\ 
               A security flaw chain is a sequential set of operations along the control
               flow which collectively for a security flaw.  Each link in the chain may or
               may not be a security flaw, and may or may not even be a bad practice (a
               bug effecting maintainability or violating a specific style guide).
               Collectively, the set of operations form a security flaw.  We will be
               interested in evaluating a tools ability to identifying such security flaw 
               chains, so seeding such flaws will be a focus of this work. However,
               since such work is likely to use the same infrastructure as the 
               seeding of simple security flaws, the design of this support can be
               deferred to later.

         \item Composite Security Flaws \\
               A composite security flaw is a set of operations each of which may or 
               may not be a security flaw (e.g. same as for the links of a security flaw
               chain).  For the same reasons as for security flaw chains, we will be
               interested in seeding these sorts of flaws to test static analysis tools.
               The security flaw chain and the composite security flaw are likely related
               but the composite security flaw may be defined over the control flow of the
               whole program where as the security flaw chain is defined over a
               significantly smaller, (and more readily characterizable, subset of the
               control flow. Further design work specific to the design for this is
               deferred.
               \fixme{Need to verify/clarify the differences between the 
               {\it security flaw chain} and the {\it composite security flaw}.}
      \end{enumerate}




\section{Scope}

The design of a tool to seed security flaws and develop evaluation test suites requires
addressing a range of related subjects.  The following subsections outline more detail.
% required to handle more general security flaws is as follows:
% \begin{enumerate}
%   \item Range of Security Flaws \\
\subsection{Range of Security Flaws}
         We assume a range of 30-40 security flaws (yet to be identified)
         which can be seeded independently of each other.  Some candidate lists of
         specific security flaws exist,
         for example, the \href{http://nvd.nist.gov/cwe.cfm}{NIST list of security flaws}, 
         and the 
         \href{https://www.securecoding.cert.org/confluence/display/seccode/CERT+Secure+Coding+Standards}{CERT Secure Coding Rules}.
         A simple list of ten used in a recent study will be used until we have
         collectively can agree on a better list. Note that some of the descriptions
         are taken from the summaries for each security flaw on the CWE website, the 
         links to more information where it is available on Wikipedia, plus
         links to the relevant CWE web pages are also provided.  The goal is to be
         as clear as possible, at this early point in the design, about the definitions of
         the security flaws that might be targets for this seeding research.
      \begin{enumerate}
         \item Buffer Overflow \\
               There are a lot of different types of buffer overflow cases to address,
               more information is available at:
               \href{http://en.wikipedia.org/wiki/Buffer_overflow}{Wikipedia: Buffer Overflow Security Flaws},
               see also:
               \href{http://cwe.mitre.org/data/definitions/119.html}{CWE-119}.

         \item Command Injection \\
               Code injection can be used by an attacker to introduce (or "inject") code
               into a computer program to change the course of execution, more information
               is available at:
               \href{http://en.wikipedia.org/wiki/Command_injection}{Wikipedia: Command Injection},
               see also:
               \href{http://cwe.mitre.org/data/definitions/78.html}{CWE-78}.

         \item Use after {\em free} \\
               Referencing memory after it has been freed can cause a program to crash,
               use unexpected values, or execute code, more information is available at:
               \href{http://cwe.mitre.org/data/definitions/416.html}{CWE-416}.

         \item Format string vulnerability \\
               The software uses externally-controlled format strings in printf-style
               functions, which can lead to buffer overflows or data representation
               problems. More information is available at:
               \href{http://en.wikipedia.org/wiki/Format_string_vulnerabilities}{Wikipedia: Format string attack}, 
               see also:
               \href{http://cwe.mitre.org/data/definitions/134.html}{CWE-134}.

         \item Memory leaks \\
               more information is available at:
               \href{http://en.wikipedia.org/wiki/Memory_leak}{Memory leaks}, 
               see also:
               \href{http://cwe.mitre.org/data/definitions/401.html}{CWE-401}.

         \item Uninitialized variables \\
               The code uses a variable that has not been initialized, leading to
               unpredictable results, more information is available at:
               \href{http://en.wikipedia.org/wiki/Uninitialized_variable}{Wikipedia: Uninitialized variables}, 
               see also:
               \href{http://cwe.mitre.org/data/definitions/457.html}{CWE-457}.

         \item Improper return types \\
               If a function does not generate the correct return/status
               codes, or if at a call site it does not handle all possible return/status codes that could
               be generated by a function, then security issues may result. 
               More information is available at:
               \href{http://cwe.mitre.org/data/definitions/389.html}{CWE-389}.

         \item Null pointer dereference \\
               A NULL pointer dereference occurs when the application dereferences a
               pointer that it expects to be valid, but is NULL, typically causing a crash
               or exit, more information is available at:
               \href{http://en.wikipedia.org/wiki/Pointer}{Wikipedia: Null pointers}, 
               see also:
               \href{http://cwe.mitre.org/data/definitions/476.html}{CWE-476}.

         \item Tainted Data / Unvalidated user input \\
             % The software does not sufficiently verify the origin or authenticity of
             % data, in a way that causes it to accept invalid data,
               The software has an absent or incorrect protection mechanism that fails to
               properly validate input that can affect the control flow or data flow of a
               program,
               more information is available at:
               \href{http://en.wikipedia.org/wiki/Taint_checking}{Wikipedia: Taint checking}, 
               see also:
             % NIST has suggested that CWE-20 be used instead of CWE-345.
             % \href{http://cwe.mitre.org/data/definitions/345.html}{CWE-345}.
               \href{http://cwe.mitre.org/data/definitions/20.html}{CWE-20}.

         \item TOCTOU (Time-of-check-to-time-of-use) \\
               Time-of-check, time-of-use race conditions occur when between the time in
               which a given resource (or its reference) is checked, and the time that
               resource is used, a change occurs in the resource to invalidate the results
               of the check, more details on this security flaw is available at:
               \href{http://en.wikipedia.org/wiki/Time-of-check-to-time-of-use}{Wikipedia: Time-of-check-to-time-of-use},
               see also:
               \href{http://cwe.mitre.org/data/definitions/367.html}{CWE-367}.

\commentout{
List that just arrived from NIST:

Here is a list of weaknesses we are most interested in.  Many come
from our source code sec. spec.  (We kept some from the spec off this
list, because we are not as interested in researching them.)  We
grouped them into categories with CWE IDs.

We do *not* consider this list Frozen Forever.  Rather it is a
"living" document that you, we, and other will occasionally update
with times and circumstances.

* Insufficient input validation #20
   - OS Command Injection	 	 #78
   - Path Manipulation		 	 #73
   - Format string vulnerability 	#134
* Memory problems (we consider overflow and underflow the same problem):
   - Failure to Constrain Operations within Bounds ... #119 
      - Heap Overflow			#122
      - Stack Overflow			#121
   - Improper Null Termination		#170
   - Use After Free			#416
   - Double Free			#415
   - Sensitive information uncleared before release #266 (and #244)
   - Memory Leak			#401
   - Null Pointer Dereference		#476
* Numeric Errors
   - Integer Overflow			#190
   - Integer Coercion Error (cast problem) #192
   - Divide by zero			#369
* Timing issues
   - TOCTOU race			#367
   - Race condition Within threads 	#366
   - Unrestricted Critical Resource Lock #412
* Security Features
   - Hard coded password		#259
   - Leftover Debug Code		#489
   - Backdoors				#510
* Coding errors
   - Error Conditions, Return Values, Status Codes #389
      - Unchecked Return Value		#252
      - Unchecked Error Condition	#248
   - Dead code				#561
   - Uninitialized Variable		#457

}

      \end{enumerate}




\section{Requirements}

   This section lays out a design for the construction of a source-to-source
translator that will automatically insert security flaws into existing application codes.
The design is meant to be collaborative effort with NIST.


\subsection{A Database of Program Information}
         \label{sec:SpecificationOfDatabase}
         An aspect of the define to support security flaws seeding is recording
where in the application seeding occurs and matching the detected security flaws
back to the source code. Since the detection of the security flaws is an issues 
associated with running the static analysis tools on the generated seeded application
it is outside of the design goals for this design document.  However, the seeding
for arbitrary applications does require managing information about what security flaw
vulnerabilities were identified (see subsection \ref{sec:SpecificationOfVulnerabilities}) 
and the location of seeded security flaws (see subsection \ref{sec:SeedingOfSecurityFlaws}).
To support this we proposed a database (of some sort) to record such
details.  Work at NIST may be helpful in defining an appropriate schema to support this
(based on their experience with collecting related tool independent information specific
to security flaws).

%  \item Specification of vulnerabilities (domain) \\
\subsection{Specification of Vulnerabilities (domain)}
         \label{sec:SpecificationOfVulnerabilities}
         For each security flaw and input application code pairing, there is an evaluation
         of the application code to identify where such security flaws could exist
         (specifically, where such flaws could be seeded).  This step defines the
         domain (mathematically) from which the application vulnerabilities can be mapped
         to versions of the application with seeded security flaws. For example, if
         an application does not contain arrays, then it might have no vulnerability
         to buffer overflow security flaws, and thus it may be inappropriate to seed 
         such an application with this type of security flaw.  It is not the goal
         of security flaw seeding to be a generator of every type of security flaw
         independent of the input applications (security flaw generators can in principle
         do this, but see subsection \ref{sec:SecurityFlawGenerator} for an alternative 
         point of view).  However, if a vulnerability for a security flaw is detected in
         the application code, then every possible variation of code that can seed that
         security flaw into the application may be appropriate. For each security flaw,
         the design of the security flaw seeding tool would:
            \begin{enumerate}
               \item Identify the location of where the vulnerabilities could exist.
               \item Classify the location of the vulnerabilities by what analysis is
                     required to resolve it as a vulnerability (loop analysis, pointer
                     analysis, inter-procedural analysis, etc.).
               \item Record this information into a database (see 
                     subsection \ref{sec:SpecificationOfDatabase}) so that the location of
                     potential vulnerabilities can be saved separately from where
                     security flaws are seeded into the application.
            \end{enumerate}

% \item Specification of search space (range) \\
\subsection{Specification of Search Space (range)}
         \label{sec:SpecificationOfSearchSpace}
         The seeding of any single security flaws can happen in many ways. The mapping of
         the vulnerabilities (domain) to the space of possible seeded security flaws
         (range) may help define a convient mathematical basis for the description of
         security flaw seeding.  For example, a buffer overflow many be introduced 
         using transformed literal values, scalar variables,
         static declared arrays, heap declared arrays, dynamically differently
         defined arrays (along different paths), etc.  The number of fundamentally
         different ways in which a security flaw may be introduced (seeded) into source
         code defines the number of dimensions of the search space.  A number of
         ways on introducing security flaws are general and so can form a 
         security flaw invariant subspace. The dimensions of the search space can be
         considered to be the number of fundamentally different ways that a security flaw
         can be hidden. Example search space dimensions are:
         \begin{enumerate}
            \item For values associated with a security flaw, the values may be:
               \begin{enumerate}
                  \item Use of literal values (most trivial)
                  \item Use of scalar variables 
                  \item Use of array values
                  \item Use of structure field values
                  \item Use of pointers to values (most difficult)
               \end{enumerate}

            \item For expressions where security flaws are either embedded or 
                  which form a part of the security flaw, the expressions can 
                  include any of the expression language constructs available
                  in the target language. Expressions can be:
               \begin{enumerate}
                  \item unary operators (most trivial)
                  \item binary operators
                  \item function arguments
                  \item array index expressions
                  \item declaration initializers
                  \item template instantiations (template expressions)
               \end{enumerate}

            \item For statements containing vulnerabilities, the embedding of 
                  the security flaw in different statements, or the use of different
                  types of statements in the seeding of the security flaw allows
                  for the range of security flaws to be seeded. Statements 
                  can be:
               \begin{enumerate}
                  \item declarations (variables, functions, classes, templates, etc.)
                  \item loop constructs
                  \item branch constructs
                  \item template instantiations (template functions)
               \end{enumerate}

         \end{enumerate}
         In several cases, there can be a depth (nested levels) at which the 
         values can be hidden.  This defines a size along the respective
         dimension.

% \item Security flaw generator \\
\subsection{Security Flaw Test Code Generator}
         \label{sec:SecurityFlawGenerator}
         It is reasonable that the same infrastructure used to seed security flaws into
         existing applications could be used to as a generator of security flaws
         independent of any input application.  A simple code generator can be written
         using a range of techniques, from macros in m4, scripting languages such as Perl
         and Python, to CPP macros using C/C++.  However, leveraging the proposed work
         to build seeding via a source-to-source compiler may permit a trivial mode
         of operation as a simple code generator for the same security flaws without the 
         seeding.
       % Certainly a source-to-source compiler infrastructure can handle the generation
       % of code as a trivial transformation, even if it is a larger hammer than required.
         We could have the proposed seeding tool have a mode where it simply generates
         the different variations of the security flaws that it would otherwise seed into
         an input application.  This may even be a useful mode for debugging and
         presenting the security flaws that would be introduced.  From a compiler
         transformation perspective this adds little to the complexity of such a tool
         (at least if using ROSE); however the transformation can't be associated with
         a site in the source code for the recognized security flaw vulnerability. It is
         not yet clear if this detail is important.

%  \item Seeding of individual security flaws \\
\subsection{Seeding of Individual Security Flaws}
         \label{sec:SeedingOfSecurityFlaws}
         An aspect of the design of a security flaw seeding tool is the seeding of individual
         security flaws within each classification. Specifically each security flaw can be
         seeded in isolation; treated orthogonally to all other security flaws. Addressing
         more sophisticated security flaws would be done by treating them a either as
         Security Flaw Chains or Composite Security Flaws.
         For each classification of security flaws, we define the following steps:
            \begin{enumerate}
               \item Identify the location of potential vulnerabilities \\ Each security
                     flaw has a relevant domain (possible location in the input
                     application) a potential vulnerability where it 
                     can be introduced (seeded).  See item 
                     \ref{sec:SpecificationOfVulnerabilities} (above) for more details.

               \item Identify the space of transformations to represent the target bug \\
                     This defines the size of the search space (the size of the 
                     space representing the code that could be generated). See item 
                     \ref{sec:SpecificationOfSearchSpace} (above) for more details.

               \item Define the subset of points in the space to evaluate \\ 
                     The space of transformations can be expected to be unreasonably
                     large. It is reasonable define subspaces and explicitly define 
                     generated security flaws for only the subspece so that we can
                     avoid the exhaustive search of extremely large spaces; which might
                     generate too much code to test or take an unreasonable amount of time
                     to evaluate using the static analysis tools. This is our experience
                     from the generation of code to test performance optimizations
                     associated with a research area in compiler optimization called 
                     {\em autotuning} (also referred to as {\em empirical analysis}).

               \item Identify the granularity of the security flaw seeding \\ 
                     where the security flaw should be applied
                     (e.g. expression, statement, basic block, function, file).
                     For example, a buffer overflow may be able to be applied to a loop
                     nest and either within each loop of the loop nest of the inner most
                     loop depending on the granularity.  Similarly, multiple versions
                     of the code for any seeded security flaw may be copied at different
                     levels of granularity, from file level granularity to function
                     level granularity, or even block level granularity.  Each have
                     consequences to the amount of code generated and the degree to which
                     the seeding of the security flaw is allowed to change the input
                     application.

               \item Generate a copy of the relevant portion of the AST \\
                     to support the seeding of multiple variations of the security flaw.
                     AST copy mechanism in ROSE are well suited to this purpose; a copy of
                     the AST subtree is a defined level or granularity is generated and
                     the security flaw is seeded into the copy. Attention should be
                     paid to the way this copy is woven back into the control flow so that
                     static analysis will not ignore the seeded bug as dead code.

               \item Apply transformation to modify each copy of the AST \\
                     Each AST subtree is transformed to insert a single security flaw.

            \end{enumerate}

% \item Seeding likely to generate security flaw false positives \\
\subsection{Seeding of Security Flaw False Positives}
         \label{sec:FalsePositiveEvaluation}
         Since many static analysis tools are overly conservative in their program
         analysis they are sensitive to falsely detecting security flaws (false
         positives). Since detection of false positives are a significant noise
         in the detection of real security flaws the seeding code which might
         appear to be a security flaw is yet another way to evaluated static 
         analysis tools.  The detection of any valid code seeded into an 
         application as a security flaw would count as a false positive.
         There are a few strategies to the introduction of good code that
         would appear as a security flaw, specifically such seeded code
         should be introduced hidden behind different types of complexities.
         Complexities that we think would be effective in fooling static analysis
         to report good code as false security flaws would be:
         \begin{enumerate}
            \item Introduction of code that uses variables defined (in use-def sense) over
            complex paths (path sensitive complexity).
            \item Introduction of code that uses variables defined (in use-def sense) over
            complexity from different function calls (context sensitive complexity).
            \item Introduce flaws in dead code (dead code complexity)
            \item Introduce good code hidden behind code that appears dead (e.g. complex switch
            statements; Duff's devise) (dead code complexity)
            \item Introduce good code with parts assembled from template meta programming (C++
            specific template instantiation complexity)
            \item Hide valid code behind code requiring pointer analysis. 
            Hide valid code behind pointer manipulations of varying complexity that are
            not easily analyzable by different grades of pointer analysis.
         \end{enumerate}

% \end{enumerate}



\section{Design}

The define of a translator to seed bugs into existing applications will be based on 
the ROSE source-to-source compiler infrastructure.  Likely any source-to-source
compiler infrastructure might alternatively work as well, or could be adapted.

   The steps are implemented via a traversal over the AST, additional graphs
for more interesting program analysis may also be consulted.  Information is passed 
between phases of the bug seeding via AST attributes placed on IR nodes of the 
AST.  This approach is supporting in many program analysis infrastructures, and is
well supported in ROSE (ROSE includes visualization support as well to simplify
understanding the security flaw seeding process and debugging).

The design will addressed using a few phases:
\begin{enumerate}
   \item Identification of security flaw vulnerabilities \\
   This work identifies where in the source code a specific security flaw {\em could}
be seeded, it does not detect an existing security flaw in the input application.
this part of the project is expected to be of value all alone as a way to count
and record the locations of opportunities for flaw so that the number of identified
flaws by static analysis tools can be reasoned via a percentable of the opportunities
for the flaw instead of just a total count.  This sort of evaluation does not
require the security flaw seeding that is much of the purpose of the rest of the
design.  Note that markers (AST Attributes) are attached to relevant IR nodes during this
phase.  A data structure called {\tt SecurityVulnerabilityAttribute} is placed at
each location in the source code where a specific type of vulnerability could exist
(see figure xxx).

   \item Clone fragments of code in the application to support seeding \\
   The user decides if flaws will be seeded into the application or into copied pieces 
of the applications (clones). It is expected that if all possible security flaws were to
be seeded into a single application that this would confuse the static analysis and 
we would not have a meaningful evaluation of static analysis tools.  To support 
a more refined approach fragments of different sizes are copied to support the 
seeding of flaws (with a definable number of flaws per clone).


\end{enumerate}
