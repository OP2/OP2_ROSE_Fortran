% 9.5.07
% This is a sample documentation for Compass in the tex format.
% We restrict the use of tex to the following subset of commands:
%
% \section, \subsection, \subsubsection, \paragraph
% \begin{enumerate} (no-nesting), \begin{quote}, \item
% {\tt ... }, {\bf ...}, {\it ... }
% \htmladdnormallink{}{}
% \begin{verbatim}...\end{verbatim} is reserved for code segments
% ...''
%

\section{Forbidden Functions}
\label{ForbiddenFunctions::overview}
Many checks common to Compass center around the forbidden use of certain ``dangerous'' functions. This checker provides a way to forbid the use of those functions through the simple use of their name.

\subsection{Parameter Requirements}
The forbidden function checker can simultaneously look for any number of
functions, either member or non-member.  A set of parameters is used, named
using a counter.  Thus, the parameters of this checker have names of the
form {\tt ForbiddenFunctions.Function{\it n}}, for {\tt {\it n}} from zero
to some limit.  The forbidden function analysis checks each name in turn
until one is missing.  Thus, if you have parameters named {\tt
ForbiddenFunctions.Function0} and {\tt ForbiddenFunctions.Function1} but no
parameter named {\tt ForbiddenFunctions.Function2}, the analysis will
search for two functions.  As a caution, if you skip a number, including
zero, no larger numbers will be scanned: any functions specified after a
skipped number will be ignored.

The format of a parameter is {\tt white space, function name, white space,
comma, reason}.  Leading and trailing white space is allowed next to the
function name, but any white space after the first comma will become part
of the reason string.  The function name is a fully qualified name, but the
leading {\tt ::} to indicate the global scope may be omitted.  Member
functions are given with their class qualifications, just as they would be
referred to when accessing a pointer to them.  Choosing one overload from
an overload set sharing the same name is not supported.  The reason field
is used to indicate why a particular function is forbidden; it may be any
string (not containing a newline), and is printed out as part of the error
message when the corresponding function is found.  It is also possible to
omit the comma and the reason field, leaving just the function name as the
parameter; in this case, no reason will be given for the function's
prohibition.

\subsection{Implementation}
This pattern is checked using a simple AST traversal visiting all SgFunctionCallExp nodes. For each node the name of the function being called is compared to those listed as forbidden functions. If a match is found between the function call name and the forbidden function name then flag an error.

\subsection{Non-Compliant Code Example}
In this example, it is assumed that the function {\tt compass\_forbidden\_function\_vfork} is forbidden by the parameter file.
\begin{verbatim}
//compass_forbidden_function_vfork() is a function meant to simulate a system
//routine. This makes no difference as an example of how forbiddenFunctions
//checks errors as it works only on function names. The reason we have used
//this example is to prevent compass from treating this test file as an error
//during `make verify' in which compass checks its own source code.

int compass_forbidden_function_vfork();
void helloWorld(int, char**);

double function();
double good_function();

namespace A {
  double function2();
}

struct B {
  void memberFunction();
};

int main(int argc, char** argv) {
  int w;
  w = compass_forbidden_function_vfork() + 5;
  helloWorld(argc, argv);
  double x = 3.0 * function();
  double y = 5.0 * good_function();
  double z = A::function2();
  B b;
  b.memberFunction();
  int (*fp)() = compass_forbidden_function_vfork;
  return 0;
}
\end{verbatim}

\subsection{Compliant Solution}
The compliant example would use a function not listed in the parameter file.

\subsection{Mitigation Strategies}
\subsubsection{Static Analysis} 

Compliance with this rule can be checked using structural static analysis checkers using the following algorithm:

\begin{enumerate}
\item For each SgFunctionCallExp node if the name of the function being called is forbidden by the parameter file then report error. 
\end{enumerate}

\subsection{References}

\htmladdnormallink{Foster}{} , ``James C.Foster, Vitaly Osipov, Nish Bhalla, Niels Heinen, Buffer Overflow Attacks, ISBN 1-932266-67-4, p. 211''

\htmladdnormallink{Secure Coding : MSC30-C. Do not use the rand function}{https://www.securecoding.cert.org/confluence/display/seccode/POS33-C.+Do+not+use+vfork\%28\%29}

\htmladdnormallink{Secure Coding : POS33-C. Do not use vfork()}{https://www.securecoding.cert.org/confluence/display/seccode/POS33-C.+Do+not+use+vfork\%28\%29}

[ \htmladdnormallink{ISO/IEC 9899-1999:TC2}{https://www.securecoding.cert.org/confluence/display/seccode/AA.+C+References} ] Section 7.19.9.2, "The {\tt fseek} function"; 7.19.9.5, "The {\tt rewind} function"

[ \htmladdnormallink{Klein 02}{https://www.securecoding.cert.org/confluence/display/seccode/AA.+C+References} ] [ \\*
   \htmladdnormallink{ISO/IEC 9899-1999}{https://www.securecoding.cert.org/confluence/display/seccode/AA.+C+References} ] Section 7.20.1.4, "The strtol, strtoll, strtoul, and strtoull functions," Section 7.20.1.2, "The atoi, atol, and atoll functions," and Section 7.19.6.7, "The sscanf function"

