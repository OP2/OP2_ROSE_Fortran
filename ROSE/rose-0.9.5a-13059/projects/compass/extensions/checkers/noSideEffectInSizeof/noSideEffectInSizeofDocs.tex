
   \section{Secure Coding : EXP06-A. Operands to the sizeof operator should not contain side effects}

   The {\tt sizeof} operator yields the size (in bytes) of its operand, which may be an expression or the parenthesized name of a type. If the type of the operand is not a variable length array type the operand is {\bf not} evaluated.

   Providing an expression that appears to produce side effects may be misleading to programmers who are not aware that these expressions are not evaluated. As a result, programmers may make invalid assumptions about program state leading to errors and possible software vulnerabilities.
   \subsection{Non-Compliant Code Example}

   In this example, the variable {\tt a} will still have a value 14 after {\tt b} has been initialized.
 \code{ 

\noindent
\ttfamily
\hlstd{}\hlline{\ \ \ \ 1\ }\hlstd{}\hltyp{int\ }\hlstd{a\ =\ }\hlnum{14}\hlstd{;\\
}\hlline{\ \ \ \ 2\ }\hlstd{}\hltyp{int\ }\hlstd{b\ =\ }\hlkey{sizeof}\hlstd{}\hlsym{(}\hlstd{a++}\hlsym{)}\hlstd{;\\
}\hlline{\ \ \ \ 3\ }\hlstd{}\\
\mbox{}\\
\normalfont



}


   The expression {\tt a++} is not evaluated. Consequently, side effects in the expression are not executed.
   \subsubsection{Implementation Specific Details}

   This example compiles cleanly under Microsoft Visual Studio 2005 Version 8.0, with the /W4 option.
   \subsection{Compliant Solution}

   In this compliant solution, the variable {\tt a} is incremented.
 \code{ 

\noindent
\ttfamily
\hlstd{}\hlline{\ \ \ \ 1\ }\hlstd{}\hltyp{int\ }\hlstd{a\ =\ }\hlnum{14}\hlstd{;\\
}\hlline{\ \ \ \ 2\ }\hlstd{}\hltyp{int\ }\hlstd{b\ =\ }\hlkey{sizeof}\hlstd{}\hlsym{(}\hlstd{a}\hlsym{)}\hlstd{;\\
}\hlline{\ \ \ \ 3\ }\hlstd{a++;\\
}\hlline{\ \ \ \ 4\ }\hlstd{}\\
\mbox{}\\
\normalfont




}

   \subsubsection{Implementation Specific Details}

   This example compiles cleanly under Microsoft Visual Studio 2005 Version 8.0, with the /W4 option.
   \subsection{Risk Assessment}

   If expressions that appear to produce side effects are supplied to the {\tt sizeof} operator, the returned result may be different then expected. Depending on how this result is used, this could lead to unintended program behavior.

   \begin{tabular}[c]{| c| c| c| c| c| c|}
   \hline
   {\bf Rule} & {\bf Severity} & {\bf Likelihood} & {\bf Remediation Cost} & {\bf Priority} & {\bf Level} \\ \hline
   EXP06-A & {\bf 1} (low) & {\bf 1} (unlikely) & {\bf 3} (low) & {\bf P3} & {\bf L3} \\ \hline
   \end{tabular}
   \subsubsection{Related Vulnerabilities}

   Search for vulnerabilities resulting from the violation of this rule on the \htmladdnormallink{CERT website}{https://www.kb.cert.org/vulnotes/bymetric?searchview\&query=FIELD+contains+EXP06-A} .
   \subsection{References}

   [ \htmladdnormallink{ISO/IEC 9899-1999}{https://www.securecoding.cert.org/confluence/display/seccode/AA.+C+References} ] Section 6.5.3.4, "The sizeof operator"
