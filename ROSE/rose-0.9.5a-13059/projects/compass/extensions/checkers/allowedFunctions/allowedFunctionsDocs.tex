% 9.5.07
% This is a sample documentation for Compass in the tex format.
% We restrict the use of tex to the following subset of commands:
%
% \section, \subsection, \subsubsection, \paragraph
% \begin{enumerate} (no-nesting), \begin{quote}, \item
% {\tt ... }, {\bf ...}, {\it ... }
% \htmladdnormallink{}{}
% \begin{verbatim}...\end{verbatim} is reserved for code segments
% ...''
%

\section{Allowed Functions}
\label{AllowedFunctions::overview}

The validation of code properties often requires the detection of certain
functions used. In some cases, the list of undesirable functions is too large
for a mechanism like forbidden functions to be effective. This allowed functions
checker performs the opposite by filtering function references against a
premade accepted list. Only those function on this list are allowed in the
source and any other function reference used is a reported violation. Usually,
the list of allowed functions is generated based off ``trusted sources'' input
to the checker.

\subsection{Parameter Requirements}

The compass\_parameter requirement for this checker has several options
including
%
\begin{itemize}
\item AllowedFunctions.OutFile specifies the file name path for the output of
	this checker. The list of all allowed functions are written to this
	file formatted identically to compass\_parameters. All other options
	passed to allowedFunctions by compass\_parameters is also preserved 
	in this output file.
\item AllowedFunctions.Function\# appends an allowed function given in the
	manged name format written by this checker. The parameters of this
	format must start sequentially at zero and increment until
	FunctionNum minus one.
\item AllowedFunctions.FunctionNum a simultaneous boolean flag and state
	variable; specifies the total number of allowed functions in the
	accepted list. If this number is postive, the allowed function checker
	executes in testing mode, where all new function references are written
	to OutFile as new allowed functions. A negative value for FunctionNum
	switches the checker into detection mode where all function references
	not read from the list in compass\_parameters are flagged as 
	violations.
\item AllowedFunctions.Library\# adds to a list of paths treated as safe
	libraries. Any sources found in these paths will be ignored by 
	allowedFunctions.
\end{itemize}

\subsection{Implementation}

This checker functions in two modes depending on the boolean sign of the
AllowedFunctions.FunctionNum parameter. 

Positive signed FunctionNum puts 
this checker into testing mode where new function references are added to
the list of allowed functions. This mode begins by reading the 
compass\_parameters file for found allowed functions and other options
given to allowedFunctions. The existing list of allowed functions is output
in-order to the output file specified by OutFile. The allowedFunctions
traversal then begins by visiting all function and member function references
and detecting their membership in the set of allowed functions. Functions
not found in this set are added and written to the OutFile while those already 
found do nothing. 

The function reference is detected and output using a special
mangled string generated by allowedFunctions that serves as its unique
identifier. This format is recursively generated by looking at the types of
a functions return and argument parameters as well as its qualified name
joined in a comma-separated string. The string sequence is return type, 
qualified name with scope, and arguments or void, etc. 
One simple example is for the {\tt memchr} function

\begin{verbatim}
AllowedFunctions.Function39=*void,::memchr,*void,int,size_t,
\end{verbatim}

After all existing and new allowed functions are written, the FunctionNum
parameter is updated with the current number of allowed functions in the list
given in OutFile. Essentially, the testing mode is designed for multiple
executions of compass with different sources such that a list of allowed
functions grows and is maintained in a format usable for compass\_parameters.

The other operation mode (detection), reads from compass\_parameters the set 
of allowed functions. Similarly, the traversal visits all function and
member function references and constructs the unique manged string identifier
for the function. This mangled string is looked up in the set of all allowed
functions. If the function is found then execution continues; but a function
that is not found in the set of allowed functions is reported as a violation.
In general, this will be the mode most users will run AllowedFunctions under.

Several exclusion mechanism are implemented by allowedFunctions for ignoring
local function definitions and function calls occuring from source designated
as a safe library. Their implementation relies on the file classification
mechanism in ROSE string support to distinguish between user, system, and
library sources.
