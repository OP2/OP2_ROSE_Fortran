\chapter{Loop Optimization}

    This section is specific to loop optimization and show several
tutorial examples using the optimization mechanisms within ROSE. 
\fixme{We might want to reference Qing's work explicitly since this is
       really just showing off here work.}

\section{Example Loop Optimizer}
     Simple example translator showing use of pre-defined loop optimizations.
\fixme{We are not running performance tests within this tutorial, but perhaps we could later.}

   Figure~\ref{Tutorial:exampleLoopOptimization} shows the code required to
call some loop optimizations within ROSE.  The translator that we build for this tutorial
is simple and takes the following command line options to control which optimizations are done.


\begin{verbatim}
   -ic1 :loop interchange for more reuses
   -bk1/2/3 <blocksize> :block outer/inner/all loops
   -fs1/2 :single/multi-level loop fusion for more reuses 
   -cp <copydim> :copy array 
   -fs0 : loop fission
   -splitloop: loop splitting 
   -unroll [locond] [nvar] <unrollsize> : loop unrolling 
   -bs <stmtsize> : break up statements in loops
   -annot <filename>: 
        Read annotation from a file which defines side effects of functions
   -arracc <funcname> :
        Use special function to denote array access (the special function can be replaced
        with macros after transformation). This option is for circumventing complex 
        subscript expressions for linearized multi-dimensional arrays.
   -opt <level=0> : The level of loop optimizations to apply (By default, only the outermost
                 level is optimized).
   -ta <int> : Max number of nodes to split for transitive dependence analysis (to limit the 
            overhead of transitive dep. analysis)
   -clsize <int> : set cache line size in evaluating spatial locality (affect decisions in 
                applying loop optimizations)
   -reuse_dist <int> : set maximum distance of reuse that can exploit cache (used to evaluate 
                 temporal locality of loops)
\end{verbatim}

\begin{figure}[!h]
{\indent
{\mySmallFontSize


% Do this when processing latex to generate non-html (not using latex2html)
\begin{latexonly}
   \lstinputlisting{\TutorialExampleDirectory/loopOptimization.C}
\end{latexonly}

% Do this when processing latex to build html (using latex2html)
\begin{htmlonly}
   \verbatiminput{\TutorialExampleDirectory/loopOptimization.C}
\end{htmlonly}

% end of scope in font size
}
% End of scope in indentation
}
\caption{Example source code showing use of loop optimization mechanisms.}
\label{Tutorial:exampleLoopOptimization}
\end{figure}

\clearpage
\section{Matrix Multiply Example}

   Using the matrix multiply example code shown in 
figure~\ref{Tutorial:exampleInputCode_LoopOptimization}, we run the loop optimizer in
figure~\ref{Tutorial:exampleLoopOptimization} and generate the code shown in 
figure~\ref{Tutorial:exampleOutput_LoopOptimization}.

\begin{figure}[!h]
{\indent
{\mySmallFontSize


% Do this when processing latex to generate non-html (not using latex2html)
\begin{latexonly}
   \lstinputlisting{\TutorialExampleDirectory/inputCode_LoopOptimization_blocking.C}
\end{latexonly}

% Do this when processing latex to build html (using latex2html)
\begin{htmlonly}
   \verbatiminput{\TutorialExampleDirectory/inputCode_LoopOptimization_blocking.C}
\end{htmlonly}

% end of scope in font size
}
% End of scope in indentation
}
\caption{Example source code used as input to loop optimization processor.}
\label{Tutorial:exampleInputCode_LoopOptimization}
\end{figure}

\begin{figure}[!h]
{\indent
{\mySmallFontSize


% Do this when processing latex to generate non-html (not using latex2html)
\begin{latexonly}
   \lstinputlisting{\TutorialExampleBuildDirectory/rose_inputCode_LoopOptimization_blocking.C}
\end{latexonly}

% Do this when processing latex to build html (using latex2html)
\begin{htmlonly}
   \verbatiminput{\TutorialExampleBuildDirectory/rose_inputCode_LoopOptimization_blocking.C}
\end{htmlonly}

% end of scope in font size
}
% End of scope in indentation
}
\caption{Output of loop optimization processor showing matrix multiply
         optimization (using options: {\tt -bk1 -fs0}).}
\label{Tutorial:exampleOutput_LoopOptimization}
\end{figure}


\clearpage
\section{Loop Fusion Example}

   Using the loop fusion example code shown in 
figure~\ref{Tutorial:exampleInputCode_LoopOptimization_fusion}, we run the loop optimizer in
figure~\ref{Tutorial:exampleLoopOptimization} and generate the code shown in 
figure~\ref{Tutorial:exampleOutput_LoopOptimization_fusion}.

\begin{figure}[!h]
{\indent
{\mySmallFontSize


% Do this when processing latex to generate non-html (not using latex2html)
\begin{latexonly}
   \lstinputlisting{\TutorialExampleDirectory/inputCode_LoopOptimization_fusion.C}
\end{latexonly}

% Do this when processing latex to build html (using latex2html)
\begin{htmlonly}
   \verbatiminput{\TutorialExampleDirectory/inputCode_LoopOptimization_fusion.C}
\end{htmlonly}

% end of scope in font size
}
% End of scope in indentation
}
\caption{Example source code used as input to loop optimization processor.}
\label{Tutorial:exampleInputCode_LoopOptimization_fusion}
\end{figure}

\begin{figure}[!h]
{\indent
{\mySmallFontSize


% Do this when processing latex to generate non-html (not using latex2html)
\begin{latexonly}
   \lstinputlisting{\TutorialExampleBuildDirectory/rose_inputCode_LoopOptimization_fusion.C}
\end{latexonly}

% Do this when processing latex to build html (using latex2html)
\begin{htmlonly}
   \verbatiminput{\TutorialExampleBuildDirectory/rose_inputCode_LoopOptimization_fusion.C}
\end{htmlonly}

% end of scope in font size
}
% End of scope in indentation
}
\caption{Output of loop optimization processor showing loop fusion (using options: {\tt -fs2}).}
\label{Tutorial:exampleOutput_LoopOptimization_fusion}
\end{figure}


\section{Example Loop Processor (LoopProcessor.C)}
     This section contains a more detail translator which uses the command-line for input
    of specific loop processing options and is more sophisticated than the previous
    translator used to handle the previous two examples.

   Figure~\ref{Tutorial:exampleLoopProcessor} shows the code required to
call the loop optimizations within ROSE.  The translator that we build for this tutorial
is simple and takes command line parameters to control which optimizations are done.

\begin{figure}[!h]
{\indent
{\mySmallFontSize


% Do this when processing latex to generate non-html (not using latex2html)
\begin{latexonly}
   \lstinputlisting{\TutorialExampleBuildDirectory/loopProcessor.C.aa}
\end{latexonly}

% Do this when processing latex to build html (using latex2html)
\begin{htmlonly}
   \verbatiminput{\TutorialExampleBuildDirectory/loopProcessor.C.aa}
\end{htmlonly}

% end of scope in font size
}
% End of scope in indentation
}
\caption{Detailed example source code showing use of loop optimization mechanisms
    (loopProcessor.C part 1).}
\label{Tutorial:exampleLoopProcessor}
\end{figure}

\begin{figure}[!h]
{\indent
{\mySmallFontSize


% Do this when processing latex to generate non-html (not using latex2html)
\begin{latexonly}
   \lstinputlisting{\TutorialExampleBuildDirectory/loopProcessor.C.ab}
\end{latexonly}

% Do this when processing latex to build html (using latex2html)
\begin{htmlonly}
   \verbatiminput{\TutorialExampleBuildDirectory/loopProcessor.C.ab}
\end{htmlonly}

% end of scope in font size
}
% End of scope in indentation
}
\caption{loopProcessor.C source code (Part 2).}
\label{Tutorial:exampleLoopProcessor2}
\end{figure}

\clearpage
\section{Matrix Multiplication Example (mm.C)}

   Using the matrix multiplication example code shown in 
figure~\ref{Tutorial:exampleInputCode_LoopOptimization_mm}, we run the loop optimizer in
figure~\ref{Tutorial:exampleLoopProcessor} and generate the code shown in 
figure~\ref{Tutorial:exampleOutput_LoopOptimization_mm}.

\begin{figure}[!h]
{\indent
{\mySmallFontSize


% Do this when processing latex to generate non-html (not using latex2html)
\begin{latexonly}
   \lstinputlisting{\TutorialExampleDirectory/inputCode_LoopOptimization_mm.C}
\end{latexonly}

% Do this when processing latex to build html (using latex2html)
\begin{htmlonly}
   \verbatiminput{\TutorialExampleDirectory/inputCode_LoopOptimization_mm.C}
\end{htmlonly}

% end of scope in font size
}
% End of scope in indentation
}
\caption{Example source code used as input to loopProcessor, show in figure~\ref{Tutorial:exampleLoopProcessor}.}
\label{Tutorial:exampleInputCode_LoopOptimization_mm}
\end{figure}

\begin{figure}[!h]
{\indent
{\mySmallFontSize


% Do this when processing latex to generate non-html (not using latex2html)
\begin{latexonly}
   \lstinputlisting{\TutorialExampleBuildDirectory/rose_inputCode_LoopOptimization_mm.C}
\end{latexonly}

% Do this when processing latex to build html (using latex2html)
\begin{htmlonly}
   \verbatiminput{\TutorialExampleBuildDirectory/rose_inputCode_LoopOptimization_mm.C}
\end{htmlonly}

% end of scope in font size
}
% End of scope in indentation
}
\caption{Output of loopProcessor using input from figure~\ref{Tutorial:exampleInputCode_LoopOptimization_mm}
    (using options: {\tt -bk1 -fs0}).}
\label{Tutorial:exampleOutput_LoopOptimization_mm}
\end{figure}


 \clearpage
\section{Matrix Multiplication Example Using Linearized Matrices (dgemm.C)}

   Using the matrix multiplication example code shown in 
figure~\ref{Tutorial:exampleInputCode_LoopOptimization_dgemm}, we run the loop optimizer in
figure~\ref{Tutorial:exampleLoopProcessor} and generate the code shown in 
figure~\ref{Tutorial:exampleOutput_LoopOptimization_dgemm}.

\begin{figure}[!h]
{\indent
{\mySmallFontSize


% Do this when processing latex to generate non-html (not using latex2html)
\begin{latexonly}
   \lstinputlisting{\TutorialExampleDirectory/inputCode_LoopOptimization_dgemm.C}
\end{latexonly}

% Do this when processing latex to build html (using latex2html)
\begin{htmlonly}
   \verbatiminput{\TutorialExampleDirectory/inputCode_LoopOptimization_dgemm.C}
\end{htmlonly}

% end of scope in font size
}
% End of scope in indentation
}
\caption{Example source code used as input to loopProcessor, show in figure~\ref{Tutorial:exampleLoopProcessor}.}
\label{Tutorial:exampleInputCode_LoopOptimization_dgemm}
\end{figure}

\begin{figure}[!h]
{\indent
{\mySmallFontSize


% Do this when processing latex to generate non-html (not using latex2html)
\begin{latexonly}
   \lstinputlisting{\TutorialExampleBuildDirectory/rose_inputCode_LoopOptimization_dgemm.C}
\end{latexonly}

% Do this when processing latex to build html (using latex2html)
\begin{htmlonly}
   \verbatiminput{\TutorialExampleBuildDirectory/rose_inputCode_LoopOptimization_dgemm.C}
\end{htmlonly}

% end of scope in font size
}
% End of scope in indentation
}
\caption{Output of loopProcessor using input from figure~\ref{Tutorial:exampleInputCode_LoopOptimization_dgemm}
    (using options: {\tt -bk1 -unroll nvar 16}).}
\label{Tutorial:exampleOutput_LoopOptimization_dgemm}
\end{figure}



 \clearpage
\section{LU Factorization Example (lufac.C)}

   Using the LU factorization example code shown in 
figure~\ref{Tutorial:exampleInputCode_LoopOptimization_lufac}, we run the loop optimizer in
figure~\ref{Tutorial:exampleLoopProcessor} and generate the code shown in 
figure~\ref{Tutorial:exampleOutput_LoopOptimization_lufac}.

\begin{figure}[!h]
{\indent
{\mySmallFontSize


% Do this when processing latex to generate non-html (not using latex2html)
\begin{latexonly}
   \lstinputlisting{\TutorialExampleDirectory/inputCode_LoopOptimization_lufac.C}
\end{latexonly}

% Do this when processing latex to build html (using latex2html)
\begin{htmlonly}
   \verbatiminput{\TutorialExampleDirectory/inputCode_LoopOptimization_lufac.C}
\end{htmlonly}

% end of scope in font size
}
% End of scope in indentation
}
\caption{Example source code used as input to loopProcessor, show in figure~\ref{Tutorial:exampleLoopProcessor}.}
\label{Tutorial:exampleInputCode_LoopOptimization_lufac}
\end{figure}

\begin{figure}[!h]
{\indent
{\mySmallFontSize


% Do this when processing latex to generate non-html (not using latex2html)
\begin{latexonly}
   \lstinputlisting{\TutorialExampleBuildDirectory/rose_inputCode_LoopOptimization_lufac.C}
\end{latexonly}

% Do this when processing latex to build html (using latex2html)
\begin{htmlonly}
   \verbatiminput{\TutorialExampleBuildDirectory/rose_inputCode_LoopOptimization_lufac.C}
\end{htmlonly}

% end of scope in font size
}
% End of scope in indentation
}
\caption{Output of loopProcessor using input from figure~\ref{Tutorial:exampleInputCode_LoopOptimization_lufac}
    (using options: {\tt -bk1 -fs0  -splitloop -annotation}).}
\label{Tutorial:exampleOutput_LoopOptimization_lufac}
\end{figure}


 \clearpage
\section{Loop Fusion Example (tridvpk.C)}

   Using the loop fusion example code shown in 
figure~\ref{Tutorial:exampleInputCode_LoopOptimization_tridvpk}, we run the loop optimizer in
figure~\ref{Tutorial:exampleLoopProcessor} and generate the code shown in 
figure~\ref{Tutorial:exampleOutput_LoopOptimization_tridvpk}.

\begin{figure}[!h]
{\indent
{\mySmallFontSize


% Do this when processing latex to generate non-html (not using latex2html)
\begin{latexonly}
   \lstinputlisting{\TutorialExampleDirectory/inputCode_LoopOptimization_tridvpk.C}
\end{latexonly}

% Do this when processing latex to build html (using latex2html)
\begin{htmlonly}
   \verbatiminput{\TutorialExampleDirectory/inputCode_LoopOptimization_tridvpk.C}
\end{htmlonly}

% end of scope in font size
}
% End of scope in indentation
}
\caption{Example source code used as input to loopProcessor, show in figure~\ref{Tutorial:exampleLoopProcessor}.}
\label{Tutorial:exampleInputCode_LoopOptimization_tridvpk}
\end{figure}

\begin{figure}[!h]
{\indent
{\mySmallFontSize


% Do this when processing latex to generate non-html (not using latex2html)
\begin{latexonly}
   \lstinputlisting{\TutorialExampleBuildDirectory/rose_inputCode_LoopOptimization_tridvpk.C}
\end{latexonly}

% Do this when processing latex to build html (using latex2html)
\begin{htmlonly}
   \verbatiminput{\TutorialExampleBuildDirectory/rose_inputCode_LoopOptimization_tridvpk.C}
\end{htmlonly}

% end of scope in font size
}
% End of scope in indentation
}
\caption{Output of loopProcessor input from figure~\ref{Tutorial:exampleInputCode_LoopOptimization_tridvpk}
    (using options: {\tt -fs2 -ic1 -opt 1 }).}
\label{Tutorial:exampleOutput_LoopOptimization_tridvpk}
\end{figure}

