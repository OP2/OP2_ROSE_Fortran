\chapter{Generating Control Flow Graphs}

   The control flow of a program is broken into {\em basic blocks}
as nodes with control flow forming edges between the basic blocks.
Thus the control flow forms a graph which often labeled edges (true 
and false), and basic blocks representing sequentially executed code.
This chapter presents the Control Flow Graph (CFG) and the ROSE 
application code for generating such graphs for any function in an 
input code.  The CFG forms a fundamental building block for more 
complex forms of program analysis.

\begin{figure}[!h]
{\indent
{\mySmallFontSize


% Do this when processing latex to generate non-html (not using latex2html)
\begin{latexonly}
   \lstinputlisting{\TutorialExampleDirectory/buildCFG.C}
\end{latexonly}

% Do this when processing latex to build html (using latex2html)
\begin{htmlonly}
   \verbatiminput{\TutorialExampleDirectory/buildCFG.C}
\end{htmlonly}

% end of scope in font size
}
% End of scope in indentation
}
\caption{Example source code showing visualization of control flow graph.}
\label{Tutorial:exampleBuildCFG}
\end{figure}

   Figure~\ref{Tutorial:exampleBuildCFG} shows the code required to generate
the control flow graph for each function of an application.  Using the input code shown in
figure~\ref{Tutorial:exampleInputCode_BuildCFG} the first function's control flow graph is
shown in figure~\ref{Tutorial:exampleBuildCFGGraph}.

\begin{figure}[!h]
{\indent
{\mySmallFontSize


% Do this when processing latex to generate non-html (not using latex2html)
\begin{latexonly}
   \lstinputlisting{\TutorialExampleDirectory/inputCode_ControlFlowGraphAnalysis.C}
\end{latexonly}

% Do this when processing latex to build html (using latex2html)
\begin{htmlonly}
   \verbatiminput{\TutorialExampleDirectory/inputCode_ControlFlowGraphAnalysis.C}
\end{htmlonly}

% end of scope in font size
}
% End of scope in indentation
}
\caption{Example source code used as input to build control flow graph.}
\label{Tutorial:exampleInputCode_BuildCFG}
\end{figure}


\begin{figure}
%\centerline{\epsfig{file=\TutorialExampleBuildDirectory/controlFlowGraph.ps,
%                    height=1.3\linewidth,width=1.0\linewidth,angle=0}}
\includegraphics[scale=0.5]{\TutorialExampleBuildDirectory/controlFlowGraph}
\caption{Control flow graph for function in input code file: inputCode\_1.C.}
\label{Tutorial:exampleBuildCFGGraph}
\end{figure}

   Figure~\ref{Tutorial:exampleBuildCFGGraph} shows the control flow graph for the
function in the input code in figure~\ref{Tutorial:exampleInputCode_BuildCFG}.



