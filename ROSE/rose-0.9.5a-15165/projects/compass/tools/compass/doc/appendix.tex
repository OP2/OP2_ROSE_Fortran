\commentout{
\section{Add Your Own Detector}

    Detectors written in Compass make direct use of ROSE and are 
designed to be copied and extended by users to develop their own 
detectors. We welcome the contribution of these detectors back to 
the ROSE team for inclusion into future releases of Compass;
full credit for all work will be provide to all authors.
Compass is an open source project using ROSE, an open source
compiler infrastructure.

  Guidelines for contributions:
\begin{itemize}
   \item Use any Compass detector and an example.
   \item provide the documentation about your detector.
   \item Use any features in ROSE to support your detector; AST, Control Flow graph,
    System dependence Graph, Call Graph, Class Hierarchy Graph, etc.
   \item Your detector should have no side-effects (on the AST).
\end{itemize}
}

\section{Design And Extensibility of Compass Detectors}

    The design of the detectors is intended to be simple
and with little required to be specified to build individual
detectors.  Of course some detectors may be non-trivial
(e.g. null pointer analysis, buffer overflow detectors, etc. 
(not yet provided in Compass)) the majority are simple.  All
detectors are meant to be side-effect free and are the subject
of separate research to independently provide automated 
combining (evaluation of multiple patterns in a single AST
traversal) and parallelization of the pattern evaluations on 
the AST.

\subsection{Input Parameter Specification}

    Parameters to all detectors are specified in an 
input parameter file (if required).  This permits numerous
knobs associated with different pattern detectors and separate
input files be specified for different software projects.

\subsection{Pattern Detection}
    Currently it is assumed that patterns will be detected as
part of a traversal of the AST.  See the ROSE Tutorial for example and 
general documentation on the different sorts of traversals possible 
within ROSE.

\subsection{Output Specification}

   Output of source position information specific to detected 
patterns are output in GNU standard source position formats.
See {\bf http://www.gnu.org/prep/standards/html\_node/Errors.html}
for more details on this format specification and now it is used
by external tools (e.g. emacs, etc.).








