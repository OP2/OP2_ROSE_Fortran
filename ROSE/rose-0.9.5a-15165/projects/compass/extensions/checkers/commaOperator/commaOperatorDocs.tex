% 9.5.07
% This is a sample documentation for Compass in the tex format.
% We restrict the use of tex to the following subset of commands:
%
% \section, \subsection, \subsubsection, \paragraph
% \begin{enumerate} (no-nesting), \begin{quote}, \item
% {\tt ... }, {\bf ...}, {\it ... }
% \htmladdnormallink{}{}
% \begin{verbatim}...\end{verbatim} is reserved for code segments
% ...''
%

\section{Comma Operator}
\label{CommaOperator::overview}

The comma operator is commonly considered confusing without any redeeming
value. This checker makes sure that it is not used. It reports any use of the
built-in comma operator and any declaration of an overloaded comma operator.

\subsection{Parameter Requirements}

This checker does not require any parameters.

\subsection{Non-Compliant Code Example}

\begin{verbatim}
int f_noncompliant(int n)
{
    return (n++, n++, n); // not OK (twice): comma operator
}
\end{verbatim}

\subsection{Compliant Solution}

\begin{verbatim}
int f_compliant(int n)
{
    n++;
    n++;
    return n;
}
\end{verbatim}

\subsection{Mitigation Strategies}
\subsubsection{Static Analysis} 

Compliance with this rule can be checked using structural static analysis
checkers identifying any appearance of the built-in comma operator and any
declaration of a function overloading the comma operator.

\subsection{References}

% Write some references
% ex. \htmladdnormallink{ISO/IEC 9899-1999:TC2}{https://www.securecoding.cert.org/confluence/display/seccode/AA.+C+References} Forward, Section 6.9.1, Function definitions''
A reference to this pattern is: The Programming Research Group:
``High-Integrity C++ Coding Standard Manual'', Item~10.19: ``Do not use the
comma operator.''
