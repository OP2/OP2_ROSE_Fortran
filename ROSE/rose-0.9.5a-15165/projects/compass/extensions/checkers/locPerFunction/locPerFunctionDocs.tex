\section{[No Reference] : Loc Per Function}

This analysis detects for each function the amount of lines of code (LOC) 
and checks the value against a user defined max value.
If LOC $>$ max value, then an exception is triggered.

\subsection{Non-Compliant Code Examples}
\begin{verbatim}
// if LocPerFunction.Size = 2
void fail() {
  int x;
  x = 5;
  x = 5;
  x = 5;
}
\end{verbatim}


\subsection{Compliant Solution}
\begin{verbatim}
// if LocPerFunction.Size = 2
void pass() {
  int x;
  x = 5;
  x = 5;
}
\end{verbatim}

\subsection{Parameter Requirements}

LocPerFunction.Size defines the max value for a permissive LOC.

\subsection{Implementation}

The simple implementation of this checker is defined below:

\begin{verbatim}
  if (isSgFunctionDeclaration(sgNode)) {
    SgFunctionDeclaration* funcDecl = isSgFunctionDeclaration(sgNode);
    SgFunctionDefinition* funcDef = funcDecl->get_definition();
    if (funcDef) {
      Sg_File_Info* start = funcDef->get_body()->get_startOfConstruct();
      Sg_File_Info* end = funcDef->get_body()->get_endOfConstruct();
      ROSE_ASSERT(start);
      ROSE_ASSERT(end);
      int lineS = start->get_line();
      int lineE = end->get_line();
      loc_actual = lineE-lineS;
      if (loc_actual>loc) {
        output->addOutput(new CheckerOutput(funcDef));
      }
    }
  }
\end{verbatim}   

\subsection{References}

