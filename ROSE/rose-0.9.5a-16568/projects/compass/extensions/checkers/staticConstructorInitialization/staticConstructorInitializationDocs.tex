%General Suggestion: Ideally, the documentation of a style checker should 
%be around one page.
\section{Static Constructor Initialization}

\label{StaticConstructorInitialization::overview}

\quote{Namespace: \texttt{StaticConstructorInitialization}}

\subsection{Introduction}

   Static constructor initialization is a common portability bug for large
scale applications; this detector finds these and reports the locations.
This test locates where static initialization would force constructors to be called
(i.e. for classes).  When this happens the order can be dependent upon the compiler
and is typically different across different compiler and versions of compilers.  The
result is that subtle dependences can be broken which are a frequent portability problem
for large scale applications.

The reference for this checker is: \\
   This test was a special request from a LLNL code team.

\subsection{Parameter Requirements}

   This test requires no input parameters.

\subsection{Implementation}

    This pattern is detected using a simple traversal without inherited 
or synthesized attributes.  The implementation does not yet look behind typedefs to
identify more complex cases where static constructor initiaization might be hidden,
it is not yet clear if typedefs can hide such cases and so this detector requires
more testing.

\subsection{Example of Failing Output Code}

     See example: tests/staticConstructorInitializationDetectorTest.C

%The following lines are for references to the examples in the
%documentation.
\begin{latexonly}
\codeFontSize{
\lstinputlisting{\ExampleDirectory/../staticConstructorInitialization/staticConstructorInitializationTest1.C}
}
\end{latexonly}

%%%%%%%%%%%%%%%%%%%%%%%%%%%%%%%%%%%%%%%%%%%%%%%%%%%
%If there is strange known behaviour, you can write a 
%subsection that describes that problem.

