% 9.5.07
% This is a sample documentation for Compass in the tex format.
% We restrict the use of tex to the following subset of commands:
%
% \section, \subsection, \subsubsection, \paragraph
% \begin{enumerate} (no-nesting), \begin{quote}, \item
% {\tt ... }, {\bf ...}, {\it ... }
% \htmladdnormallink{}{}
% \begin{verbatim}...\end{verbatim} is reserved for code segments
% ...''
%

\section{Do Not Call Putenv With Auto Var}
\label{DoNotCallPutenvWithAutoVar::overview}

% write your introduction
The POSIX function putenv() is used to set environment variable values. The
putenv() function does not create a copy of the string supplied to it as an
argument, rather it inserts a pointer to the string into the environment array.
If a pointer to a buffer of automatic storage duration is supplied as an
argument to putenv(), the memory allocated for that buffer may be overwritten
when the containing function returns and stack memory is recycled. This behavior
is noted in the Open Group Base Specifications Issue 6 [Open Group 04]:

A potential error is to call putenv() with an automatic variable as the
argument, then return from the calling function while string is still part of
the environment.

The actual problem occurs when passing a pointer to an automatic variable to
putenv().  An automatic pointer to a static buffer would work as intended.

\subsection{Parameter Requirements}

%Write the Parameter specification here.
   No Parameter specifications.

\subsection{Implementation}

%Details of the implementation go here.
The putenv() function is not required to be thread-safe, and the one in libc4,
libc5 and glibc2.0 is not, but the glibc2.1 version is.

Description for libc4, libc5, glibc: If the argument string is of the form name,
and does not contain an `=' character, then the variable name is removed from
the environment. If putenv() has to allocate a new array environ, and the
previous array was also allocated by putenv(), then it will be freed. In no case
will the old storage associated to the environment variable itself be freed.

The libc4 and libc5 and glibc 2.1.2 versions conform to SUSv2: the pointer
argument given to putenv() is used. In particular, this string becomes part of
the environment; changing it later will change the environment. (Thus, it is an
error to call putenv() with a pointer to a buffer of automatic storage duration
as the argument, then return from the calling function while the string is still
part of the environment.) However, glibc 2.0-2.1.1 differs: a copy of the string
is used. On the one hand this causes a memory leak, and on the other hand it
violates SUSv2. This has been fixed in glibc2.1.2.

The BSD4.4 version, like glibc 2.0, uses a copy.

SUSv2 removes the `const' from the prototype, and so does glibc 2.1.3.

The FreeBSD implementation of putenv() copies the value of the provided string,
and the old values remain accessible indefinitely. As a result, a second call to
putenv() assigning a differently sized value to the same name results in a
memory leak.

\subsection{Non-Compliant Code Example}

% write your non-compliant code subsection
In this non-compliant coding example, a pointer to a buffer of automatic storage
duration is used as an argument to putenv() [Dowd 06]. The TEST environment
variable may take on an unintended value if it is accessed once func() has
returned and the stack frame containing env has been recycled.

Note that this example also violates rule [DCL30-C. Declare objects with
appropriate storage durations].

\begin{verbatim}

int func(char *var) {
  char env[1024];

  if (snprintf(env, sizeof(env),"TEST=%s", var) < 0) {
    /* Handle Error */
  }

  return putenv(env);
}

\end{verbatim}

\subsection{Compliant Solution}

% write your compliant code subsection
The setenv() function allocates heap memory for environment variables. This
eliminates the possibility of accessing volatile, stack memory.

\begin{verbatim}

int func(char *var) {
  return setenv("TEST", var, 1);
}

\end{verbatim}

\subsection{Mitigation Strategies}
\subsubsection{Static Analysis} 

Compliance with this rule can be checked using structural static analysis checkers using the following algorithm:

\begin{enumerate}
\item{Checks to see if the variable being passed to putenv() is declared in the global
scope.  If it is not, the checker creates new output.}
\end{enumerate}

\subsection{References}

% Write some references
% ex. \htmladdnormallink{ISO/IEC 9899-1999:TC2}{https://www.securecoding.cert.org/confluence/display/seccode/AA.+C+References} Forward, Section 6.9.1, Function definitions''

\htmladdnormallink{Open Group 04}{http://www.opengroup.org/onlinepubs/009695399/toc.htm} The putenv() function

\htmladdnormallink{ISO/IEC9899-1999}{https://www.securecoding.cert.org/confluence/display/seccode/AA.+C+References\#AA.CReferences-ISO\%2FIEC98991999}Section 6.2.4, "Storage durations of objects," and Section 7.20.3, ``Memory management functions''

\htmladdnormallink{Dowd 06}{https://www.securecoding.cert.org/confluence/display/seccode/AA.+C+References\#AA.CReferences-Dowd06} Chapter 10, ``UNIX Processes'' (Confusing putenv() and setenv())
