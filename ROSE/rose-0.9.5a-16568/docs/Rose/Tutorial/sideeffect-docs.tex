\documentclass[times, 10pt]{article} 

%\usepackage{akit/latex8}
\usepackage{times}
\usepackage{epsfig}
\usepackage{subfigure}

\long\def\com#1{}

\title{Side-effect Analysis}

\begin{document}

\date{}

\maketitle


ROSE provides an alias-free flow-insensitive side-effect analysis
based on the algorithm presented by Cooper and Kennedy~\cite{cooper+:pldi88}.
The implementation requires that ROSE by configured with MySQL since
it potentially performs whole-program analysis and thus requires
persistence across ROSE invocations.

The implementation does not support virtual functions.  Further it
does not directly account for potential aliases, though the original
algorithm may incorporate an external alias analysis in a flow-insensitive
manner (Section 5 of~\cite{cooper+:pldi88}).  The algorithm is
intended to augment the {\it DMOD(s)} set of variables modified by
statement {\it s} independent of aliasing with a set {\it ALIAS(p)}
for a procedure {\tt p} to create the final set {\it MOD(s)}.  
Intuitively, if ${\it s \in p}$ then 
${\it \forall x \in DMOD(s)}$, if ${\it \exists <x,y> \in ALIAS(p)}$,
{\it y} is added to {\it MOD(s)}~\cite{cooper+:pldi88}.  The
{\it ALIAS(p)} set could be provided by additional analysis or via
ROSE's annotation language which provides a facility to
specify aliasing~\ref{annotation section}.  An alternate approach
would extend the annotation language to explicitly specify variables
that are not aliased.  Using this information a transformation or
subsequent analysis might determine that although a variable {\it x}
is modified because an annotation asserts that {\it y} does not alias {\it x},
{\it y} is known not to be modified.

During the analysis a function or method may be invoked for which
the source code is not available.  Typical examples include
external libraries or system calls.  To overcome this limitation
the user may specify the side effects in an annotation 
file~\ref{annotation section}.  This file should be passed to
the ROSE {\tt dbAnnotationParser} tool which populates a database
with the annotation information making it available for subsequent
analysis.

The abstract base class {\tt SideEffectAnalysis} provides access to
the side-effect implementation via the {\tt create} method.  The
analysis is performed by invoking one of the three {\tt calcSideEffect}
methods.  This overloaded method may be run on an entire project by
passing it a reference to an {\tt SgProject}, a single file by
passing it a reference to an {\tt SgFile}, or an arbitrary AST
node by passing it a reference to an {\tt SgNode}.  The results of
the analysis are provided via the {\tt getGMOD} and {\tt getDMOD} methods.
{\tt getGMOD} takes a function name and returns a list of side effects
for that function.  The class provides the method {\tt getCalledFunctions}
to return a list of invoked functions encountered during the analysis;
such functions may subsequently be passed to {\tt getGMOD}.  {\tt getDMOD}
behaves similarly but acts on a node descriptor instead of a
function name, returning any side effects corresponding to that node.
Node descriptors corresponding to a given node are provided by
the {\tt getNodeIdentifier} method.  Please see the {\tt testSideEffect}
test for example use of this interface.

SHOULD WE INCLUDE ALL OR EXCERPTS OF testSideEffect?


\bibliographystyle{plain}
\bibliography{sideeffect}
\end{document}
