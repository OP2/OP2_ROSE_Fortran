\documentclass[times, 10pt]{article} 

%\usepackage{akit/latex8}
\usepackage{times}
\usepackage{epsfig}
\usepackage{subfigure}

\long\def\com#1{}

\title{ROSE Annotation Language and Parser Library}

\begin{document}

\date{}

\maketitle


The guiding philosophy of ROSE states that traditional compilers lack
sufficient semantic understanding of user-defined abstractions to perform
optimizations and transformations that would otherwise be legal.  A 
developer may codify this semantic understanding in a transformation that
neither adheres to the constraints of, nor performs, analyses such as side
effect or alias analysis.  The transformation guarantees its own safety
by exploiting knowledge of the abstraction or application (e.g., that
actual arguments are not aliased, that a virtual function invocation never
incurs side effects, etc.).

ROSE supports a second approach to optimization-- one that aids rather
than obviates traditional analysis.  To this end ROSE provides an
annotation language through which a developer may assert program
properties that a compiler might otherwise be too conservative to 
discover (e.g., a function is side-effect free).  The first approach to
optimization leverages application- or domain-specific transformations,
while the second approach uses standard and general transformations
(e.g., code motion and inlining of a function) that are enabled by
assertions specified by the annotation (e.g., that the function is side-effect
free).

The annotation language was developed and described by Guyer and Lin~\cite{guyer+:dsl99, guyer03},
while the implementation is adapted from their C-Breeze compiler.  The
first reference provides a solid introduction to the annotation language,
while Guyer's dissertation provides considerable additional depth.  Please
note that there are some discrepancies between the paper and the final
implementation incorporated in to ROSE and described in Guyer's
dissertation.  These discrepancies include changes to nested properties, the
structure definitions, and the removal of the "none" keyword in relation to 
properties.

\com{
The original language included keywords to trigger transformations
(e.g., specialize, remove, replace) and drive analysis (e.g., analyze).
The restricted subset of the language implemented in ROSE is intended
only to provide program assertions in the form of annotations.  Therefore,
these more dynamic features of the language (appendices A.6 and A.7 of
the grammar~\cite{guyer+:dsl99}) have been removed.
}%com

Annotations are specified in a separate file, such as the one shown
in Figure~\ref{figure:annfile} taken
from the ROSE distribution.

\begin{figure}
\begin{verbatim}

%{

}%

procedure fseek(stream, offset, whence)
{
        access { stream, offset, whence }
        modify { stream }
}
 
procedure ftell(stream, pos)
{
        access { stream, pos }
        modify { pos }
}
 
\end{verbatim}
\caption{Annotation File Describing I/O Library}
\label{figure:annfile}
\end{figure}

Any statements between {\tt \%\{ \}\% } may be used to provide access to C/C++
definitions.  For example, {\tt \#include } statements may be placed here.
While still a requirement in the annotation file, this prologue is
a vestige of the complete implementation in C-Breeze.

This file models two library calls that would otherwise be inaccessible
to compiler analysis:  fseek and ftell.  The definitions begin with the
procedure keyword and are followed by the name of the function, a
(typeless) parameter list, and a body that provides annotations.
In the example above the annotations state that all parameters are
accessed and that a subset of them are modified.

The subset of Guyer's original annotation grammar~\cite{guyer03} supported 
by ROSE is described in Figures~\ref{annbnf1},~\ref{annbnf2}, 
and~\ref{annbnf3}.  This grammar departs from Guyer's original specification
in that neither action annotations nor C-style numerical comparisons
are supported.

\begin{figure}
\begin{tabular}[c]{l c l}
{\it annotation-file} & $\rightarrow$ & {\it annotation} * \\
\\
{\it annotation} & $\rightarrow$ & {\it header} \\
& $|$ & {\it property} \\
& $|$ & {\it procedure} \\
& $|$ & {\it global} \\
\\
{\it header} & $\rightarrow$ & \%\{ \\
& & C CODE \\
& & \%\} \\
\\
{\it procedure} & $\rightarrow$ & {\tt procedure} IDENTIFIER ( {\it identifier-list } ) \\
& & \{ {\it procedure-annotation * } \} \\
\\
{\it procedure-annotation} & $\rightarrow$ & {\it pointer-annotation} \\
& $|$ & {\it dependence-annotation} \\
& $|$ & {\it analysis-rule-annotation} \\
& $|$ & {\it report-annotation} \\
\\
{\it identifier-list} & $\rightarrow$ & IDENTIFIER [, {\it identifier-list} ] \\
\\
{\it pointer-annotation} & $\rightarrow$ & {\tt on\_entry } \{ {\it pointer-structure} * \} \\
& $|$ & {\tt on\_exit} \{ {\it pointer-structure} * \} \\
& $|$ & {\tt on\_exit} \{ {\it cond-pointer-structure} * \} \\
\\
{\it cond-pointer-structure} & $\rightarrow$ & {\tt if } ( {\it condition} ) \{ {\it pointer-structure} * \} \\
& $|$ & {\tt default} \{ {\it pointer-structure} * \} \\
\\
{\it pointer-structure} & $\rightarrow$ & [{\tt I/O}] IDENTIFIER \\
& $|$ & IDENTIFIER --\textgreater{} [{\tt new}] {\it pointer-structure} \\
& $|$ & IDENTIFIER {\it pointer-structure * } \\
& $|$ & {\tt delete} IDENTIFIER \\
\\
{\it dependence-annotation} & $\rightarrow$ & {\tt access} \{ {\it identifier-list} \} \\
& $|$ & {\tt modify} \{ {\it identifier-list} \} \\
\\
{\it property} & $\rightarrow$ & {\tt property} IDENTIFIER : {\it enum-property-definition} \\
& $|$ & {\tt property} IDENTIFIER : {\it set-property-definition} \\
\\
{\it enum-property-definition} & $\rightarrow$ & [ {\it direction} ] {\it property-values } [ {\it initial-value} ] \\
\\
{\it direction} & $\rightarrow$ & {\tt @forward} \\
& $|$ & {\tt @backward} \\
\\
{\it initial-value} & $\rightarrow$ & {\tt initially} IDENTIFIER \\
\\
{\it property-values} & $\rightarrow$ & {\it property-value-list} \\
\\
{\it property-value-list} & $\rightarrow$ & {\it property-value} [, {\it property-value-list}]\\
\\
{\it property-value} & $\rightarrow$ & IDENTIFIER [ {\it property-values} ] \\
\\
{\it set-property-definition} & $\rightarrow$ & \{ {\tt union-set} \} \\
& $|$ & \{ {\tt intersect-set} \} \\
& $|$ & \{ {\tt union-equivalence} \} \\
& $|$ & \{ {\tt intersect-equivalence} \} \\
\end{tabular}
\caption{Annotation Language}
\label{annbnf1}
\end{figure}

\begin{figure}
\begin{tabular}[c]{l c l}
{\it analysis-rule-annotation} & $\rightarrow$ & {\tt analyze} IDENTIFIER \{ {\it analysis-rule} * \}\\
& $|$ & {\tt analyze} IDENTIFIER \{ {\it analysis-effect} * \}\\
\\
{\it analysis-rule} & $\rightarrow$ & {\tt if} ( {\it condition} ) \{ {\it analysis-effect} * \} \\
& $|$ & {\tt default} \{ {\it analysis-effect} * \} \\
\\
{\it condition} & $\rightarrow$ & {\it test} \\
& $|$ & {\it condition} {\tt $|$$|$} {\it condition} \\
& $|$ & {\it condition} {\tt \&\&} {\it condition} \\
& $|$ & {\tt !} {\it condition} \\
& $|$ & ( {\it condition} ) \\
\\
{\it test} & $\rightarrow$ & {\it enum-property-test} \\
& $|$ & {\it set-property-test} \\
& $|$ & {\it numeric-test} \\
& $|$ & {\it binding-test} \\
\\
{\it enum-property-test} & $\rightarrow$ & [IDENTIFIER : ] \\
& & IDENTIFIER {\it temporal-operator} {\tt is-??} \\
& $|$ & [IDENTIFIER : ] \\
& & IDENTIFIER {\it temporal-operator} {\it enum-property-operator} IDENTIFIER \\
\\
{\it temporal-operator} & $\rightarrow$ & {\tt @before} \\
& $|$ & {\tt @after} \\
& $|$ & {\tt @always} \\
& $|$ & {\tt @ever} \\
\\
{\it enum-property-operator} & $\rightarrow$ & {\tt is-exactly} \\
& $|$ & {\tt is-atleast} \\
& $|$ & {\tt could-be} \\
& $|$ & {\tt is-atmost} \\
\\
{\it set-property-test} & $\rightarrow$ & IDENTIFIER IDENTIFIER IDENTIFIER \\
& $|$ & IDENTIFIER {\tt is-element-of} IDENTIFIER \\
& $|$ & IDENTIFIERS {\tt is-\{\}} \\
\\
{\it numeric-test} & $\rightarrow$ & IDENTIFIER {\tt is-\#} \\
\end{tabular}
\caption{Annotation Language}
\label{annbnf2}
\end{figure}

\begin{figure}
\begin{tabular}[c]{l c l}
{\it binding-test} & $\rightarrow$ & IDENTIFIER {\tt is-aliasof} IDENTIFIER \\
& $|$ & IDENTIFIER {\tt is-sameas} IDENTIFIER \\
& $|$ & IDENTIFIER {\tt is-empty} \\
\\
{\it analysis-effect} & $\rightarrow$ & {\it numeric-assignment} \\
& $|$ & {\it enum-property-assignment} \\
& $|$ & {\it set-property-operation} \\
\\
{\it numeric-assignment} & $\rightarrow$ & IDENTIFIER {\tt = } IDENTIFIER \\
\\
{\it enum-property-assignment} & $\rightarrow$ & IDENTIFIER \textless{}- IDNETIFIER \\
& $|$ & IDENTIFIER \textless{}-+ IDENTIFIER \\
\\
{\it set-property-operation} & $\rightarrow$ & {\tt add} IDENTIFIER \\
& $|$ & IDENTIFIER IDENTIFIER IDENTIFIER \\
\\
{\it report-annotation} & $\rightarrow$ & {\tt report} {\it report-element-list} ; \\
& $|$ & {\tt report if} ( {\it condition} ) {\it report-element-list} ; \\
& $|$ & {\tt error} {\it report-element-list} ; \\
& $|$ & {\tt error if} ( {\it condition} ) {\it report-element-list} ; \\
\\
{\it report-element-list} & $\rightarrow$ & {\it report-element} [{\tt ++}{\it report-element-list}] \\
\\
{\it report-element} & $\rightarrow$ & STRING \\
& $|$ & IDENTIFIER : IDENTIFIER {\it temporal-operator} \\
& $|$ & {\it num-expresion} {\it temporal-operator} \\
& $|$ & {\it program-location} \\
& $|$ & [ IDENTIFIER ] \\
\\
{\it program-location} & $\rightarrow$ & {\tt @callsite} \\
& $|$ & {\tt @context} \\
\\
{\it global} & $\rightarrow$ & {\tt global} \{ {\it pointer-structure} * \} \\
& $|$ & {\it analysis-rule-annotation} \\
\end{tabular}
\caption{Annotation Language (continued)}
\label{annbnf3}
\end{figure}

Annotations are made accessible to ROSE analysis modules or
transformations via the Annotations class.  A subset of the functionality 
provided by the \linebreak {\tt Annotations} class is shown in 
Figure~\ref{figure:annuse} by a sample application
that parses each annotation file passed as input and prints 
the access and modify lists for each procedure modeled within that file.

This class is instantiated with the name of an annotation file and 
parses annotations within that file upon creation.  The application
iterates over the procedures in each annotation file via the 
{\tt procedures()} accessor method.  The {\tt procedureAnn} class exposed by
this method in turn provides access to the procedure's formal parameters
(via the {\tt formal\_params()} method), any accessed variables 
(via the {\tt uses()} method), and any modified variables 
(via the {\tt defs()} method).

This simple application is intended to be an illustrative example,
rather than an exhaustive description, of the capabilities of the
annotation parser.

\begin{figure}
\begin{verbatim}

#include <iostream> 
#include "config.h" 
#include <rose.h> 
#include "broadway.h" 
 
int main(int argc, char **argv) 
{ 
 
  for(int n = 1; n < argc; ++n) { 
 
    Annotations * a = new Annotations(*(new string(argv[n])), NULL); 
 
    for (procedures_map_cp p = a->procedures().begin(); 
         p != a->procedures().end(); 
         ++p) { 
      procedureAnn * proc = p->second; 
 
      cout << "procedure " << proc->name() << " ("; 
       
      for (var_list_cp p = proc->formal_params().begin(); 
           p != proc->formal_params().end(); 
           ++p) { 
        annVariable * decl = *p; 
        if (p != proc->formal_params().begin()) 
          cout << ", "; 
        cout << decl->name(); 
      } 
      cout << ")" << endl; 
      cout << "{" << endl; 
       
      cout << "  access { "; 
      var_set_cp p;
      for (p = proc->uses().begin(); p != proc->uses().end(); ++p) { 
        annVariable * decl = *p; 
        if (p != proc->uses().begin()) 
          cout << ", "; 
        cout << decl->name(); 
      } 
      cout << "  }" << endl; 
       
      cout << "  modify { "; 
      for (p = proc->defs().begin(); p != proc->defs().end(); ++p) { 
        annVariable * decl = *p; 
        if (p != proc->defs().begin()) 
          cout << ", "; 
        cout << decl->name(); 
      } 
       
      cout << "  }" << endl; 
      cout << "}" << endl; 
       
    } 
  } 
  return(0); 
} 
 
\end{verbatim}
\caption{Example Use of Annotation Library}
\label{figure:annuse}
\end{figure}





\bibliographystyle{plain}
\bibliography{parser}
\end{document}


\com{
ROSE INTERNAL MEMO:

From: "Sam Guyer" <sammy@cs.utexas.edu>
To: "'Brian White'" <bwhite@csl.cornell.edu>
Subject: RE: Broadway front end
Date: Wed, 7 Jan 2004 15:33:55 -0600
MIME-Version: 1.0
Content-Type: text/plain;
        charset="us-ascii"
Content-Transfer-Encoding: 7bit
X-Mailer: Microsoft Office Outlook, Build 11.0.5510
In-Reply-To: <Pine.LNX.4.44.0401071445350.21601-100000@cacao.csl.cornell.edu>
Thread-Index: AcPVWBOmk8nqbjZsQre+NaPWdyCrfwACggYw
X-MimeOLE: Produced By Microsoft MimeOLE V6.00.2800.1165
Status: RO
X-Status:
X-Keywords:
X-UID: 90
 
Hi Brian,
 
I'm glad to hear that you're getting somewhere!
So, the annotation changed in several ways between the paper (which appeared
in 1999) and the fully implemented system. First, we got rid of the "none"
option -- the new syntax allows nested properties, so you can just use that
to give a least element:
 
property Vehicle = { None { car { Ford, Chevy }
                            boat { sail, power }}
 
We also made the structure definitions more convenient. In the paper you had
to say "Data of X" and "View of Y". Now you can write things like this:
 
on_entry { obj --> view_1 { DATA --> data_1
                            Other --> Bar} }
 
This annotation describes a pointer to a structure that contains two
pointers to other objects. In this context "DATA" and "Other" are field
names (they may appear multiple times) and you use them just as you would in
C:
 
access { some_structure.Data }
 
Does that make sense? You can find a much more detailed description of the
current annotation language in my thesis (on my web page).
 
-Sam
 
--
Samuel Guyer
Postdoctoral Fellow
Dept. of Computer Sciences
University of Texas at Austin
http://www.cs.utexas.edu/users/sammy
 
> -----Original Message-----
> From: Brian White [mailto:bwhite@csl.cornell.edu]
> Sent: Wednesday, January 07, 2004 1:55 PM
> To: Sam Guyer
> Subject: RE: Broadway front end
>
>
> Hello Sam,
>
> I'm back at your annotation language and have made some progress.
>
> I've liberally gutted some of the glue that seems to have connected the
> front- and back-ends; I now have a stand-alone annotation parser.  Is
> there a manual that describes the current version of the annotation
> language?
>
> In doing some rudimentary testing I'm trying to parse the examples from
> your "An Annotation Language for Optimizing Software Libraries" paper.  It
> seems like some things have changed:  in particular, properties no longer
> seem to be specified as
>
> property Distribution = { General = none, ColPanel, RowPanel, Local, Empty
> };
>
> [ " = none" specified a name for bottom in the lattice. ]
>
> but now should be specified as
>
> property Distribution : { General, ColPanel, RowPanel, Local, Empty }
>
> where I can't figure out how to name bottom.
>
> Also I seem to be having some problems parsing "DATA of view_1" in
>
> on_entry { obj --> view_1, DATA of view_1 --> data_1 }
>
> At this time I'm most interested in modify and access, so these particular
> problems might not be important.  But in general it would be helpful if I
> could have some authoritative manual to supplement my reading of the
> grammar and guessing at some of the semantics.
>
> Thanks,
> Brian
}%com
