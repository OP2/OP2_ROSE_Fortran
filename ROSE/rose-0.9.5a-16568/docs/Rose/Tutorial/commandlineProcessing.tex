\chapter{Command-line Processing Within Translators}

     ROSE includes mechanism to simplify the processing of command-line arguments
so that translators using ROSE can trivially replace compilers within
makefiles.  This example shows some of the many command-line handling
options within ROSE and the ways in which customized options may be added for 
specific translators.

% \subsection{Recognizing custom command-line options}
% \subsection{Adding options to internal ROSE command-line driven mechanisms}

\begin{figure}[!h]
{\indent
{\mySmallFontSize

\label{Tutorial:exampleCommandlineProcessing}

% Do this when processing latex to generate non-html (not using latex2html)
\begin{latexonly}
   \lstinputlisting{\TutorialExampleDirectory/commandlineProcessing.C}
\end{latexonly}

% Do this when processing latex to build html (using latex2html)
\begin{htmlonly}
   \verbatiminput{\TutorialExampleDirectory/commandlineProcessing.C}
\end{htmlonly}

% end of scope in font size
}
% End of scope in indentation
}
\caption{Example source code showing simple command-line processing within ROSE translator.}
\end{figure}


\begin{figure}[!h]
{\indent
{\mySmallFontSize

\label{Tutorial:exampleOutput_CommandlineProcessing}

% Do this when processing latex to generate non-html (not using latex2html)
\begin{latexonly}
   \lstinputlisting{\TutorialExampleBuildDirectory/commandlineProcessing.out}
\end{latexonly}

% Do this when processing latex to build html (using latex2html)
\begin{htmlonly}
   \verbatiminput{\TutorialExampleBuildDirectory/commandlineProcessing.out}
\end{htmlonly}

% end of scope in font size
}
% End of scope in indentation
}
\caption{Output of input code using commandlineProcessing.C}
\end{figure}


\section{Commandline Selection of Files}

\paragraph{Overview} This example shows the optional processing of specific files selected
after the call to the frontend to build the project.  First the SgProject if build and
{\em then} the files are selected for processing via ROSE or the backend compiler
directly.

   This example demonstrates the separation of the construction of a SgProject with
valid SgFile objects for each file on the command line, but with an empty SgGlobal scope,
and the call to the frontend, called for each SgFile in a separate loop over all the 
SgFile objects.

\begin{figure}[!h]
{\indent
{\mySmallFontSize

\label{Tutorial:exampleCommandlineProcessing}

% Do this when processing latex to generate non-html (not using latex2html)
\begin{latexonly}
   \lstinputlisting{\TutorialExampleDirectory/commandlineProcessing.C}
\end{latexonly}

% Do this when processing latex to build html (using latex2html)
\begin{htmlonly}
   \verbatiminput{\TutorialExampleDirectory/commandlineProcessing.C}
\end{htmlonly}

% end of scope in font size
}
% End of scope in indentation
}
\caption{Example source code showing simple command-line processing within ROSE translator.}
\end{figure}


\begin{figure}[!h]
{\indent
{\mySmallFontSize

\label{Tutorial:exampleOutput_CommandlineProcessing}

% Do this when processing latex to generate non-html (not using latex2html)
\begin{latexonly}
   \lstinputlisting{\TutorialExampleBuildDirectory/commandlineProcessing.out}
\end{latexonly}

% Do this when processing latex to build html (using latex2html)
\begin{htmlonly}
   \verbatiminput{\TutorialExampleBuildDirectory/commandlineProcessing.out}
\end{htmlonly}

% end of scope in font size
}
% End of scope in indentation
}
\caption{Output of input code using commandlineProcessing.C}
\end{figure}




