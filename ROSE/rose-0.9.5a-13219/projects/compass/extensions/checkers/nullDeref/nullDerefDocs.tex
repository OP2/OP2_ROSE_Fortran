%General Suggestion: Ideally, the documentation of a style checker should 
%be around one page.
\section{CERT EXP33-C and EXP34-C : Null Dereference}

NULL Dereference checker. If any variable that could be NULL is dereferenced, a warning is issued.
This is an implementation of US-CERT rules: EXP33-C - Do not reference uninitialized variables and
EXP34-C - Ensure a pointer is valid before dereferencing it.

\subsubsection{EXP33-C - Do not reference uninitialized variables}

Local, automatic variables can assume unexpected values if they are used before they are 
initialized. 
C99 specifies \emph{If an object that has automatic storage duration is not 
initialized explicitly, its value is indeterminate}
[ISO/IEC 9899-1999]. In practice, this value defaults to whichever values are currently stored in stack memory. 
While uninitialized memory often 
contains zero, this is not guaranteed. Consequently, uninitialized memory can cause a program 
to behave in an unpredictable or unplanned manner and may provide an avenue for attack.

\subsection{EXP34-C - Ensure a pointer is valid before dereferencing it}

Attempting to dereference an invalid pointer results in undefined behavior, 
typically abnormal program termination. Given this, pointers should be checked 
to make sure they are valid before they are dereferenced.

\subsection{Non-Compliant Code Examples}

\subsubsection{EXP33-C - Do not reference uninitialized variables}

In this example, the set\_flag() function is supposed to set a the variable sign to 1 if 
number is positive and -1 if number is negative. However, the programmer forgot 
to account for number being 0. If number is 0, then sign will remain uninitialized. 
Because sign is uninitialized, it assumes whatever value is at that location 
in the program stack. This may lead to unexpected, incorrect program behavior.

\begin{verbatim}
void set_flag(int number, int *sign_flag) {
  if (number > 0) {
    *sign_flag = 1;
  }
  else if (number <0) {
    *sign_flag = -1;
  }
  int x = *sign_flag;
}

int main(int argc, char** argv) {
  int sign;
  set_flag(0,&sign);
  return 0;
}
\end{verbatim}

\subsubsection{EXP34-C - Ensure a pointer is valid before dereferencing it}


In this example, input\_str is copied into dynamically allocated memory referenced by str. 
If malloc() fails, it returns a NULL pointer that is assigned to str. 
When str is dereferenced in strcpy(), the program behaves in an unpredictable manner.

\begin{verbatim}
#include "assert.h"
#include <stdlib.h>

void testme() {
  // case 1
  int size = 5;
  char* str = (char*) malloc(size+1);
  char z = *str;

  // case 2
  int *p = 0;
  int l = *p;
  
  // case 3
  char *k=0;
  free(k);

}
\end{verbatim}

\subsection{Compliant Solution}

\subsubsection{EXP33-C - Do not reference uninitialized variables}

We do not check the expressions in if conditions, and hence 
it is irrelevant what the if conditions state. However, because
an if condition occurs, there might be a path that leaves sign\_flag 
uninitialized. In this case a simple assert helps to avoid the
warning caused by this analysis.

\begin{verbatim}
#include "assert.h"

void set_flag(int number, int *sign_flag) {
  assert(sign_flag);
  if (number > 0) {
    *sign_flag = 1;
  }
  else if (number <0) {
    *sign_flag = -1;
  }
  int x = *sign_flag;
}

int main(int argc, char** argv) {
  int sign;
  set_flag(0,&sign);
  return 0;
}
\end{verbatim}

\subsection{EXP34-C - Ensure a pointer is valid before dereferencing it}
\begin{verbatim}
#include "assert.h"
#include <stdlib.h>

void testme() {
  // case 1
  int size = 5;
  char* str = (char*) malloc(size+1);
  if (str==NULL) {
    *str = '5';
  }
  char z = *str;

  // case 2
  int *p = 0;
  assert(p);
  int l = *p;
  
  // case 3
  char *k=0;
  assert(k);
  free(k);

}
\end{verbatim}

\subsection{Parameter Requirements}

None.

\subsection{Implementation}

We use a dataflow analysis to determine null dereference. The
dataflow analysis is based on an \emph{breadth first search} (bfs) algorithm, implemented in BOOST.
The implementation does a bfs backwards traversal for each:

a) SgArrowExp

b) SgPointerDerefExp

c) SgAssignInitializer

d) SgFunctionCallExp (free)

The above are the points that need to be validated by the algorithm. At each such point
the program might be invalid due to NULL pointer dereferences. Therefore, the variable
at that point must be determined and the programmed is tracked back (dataflow).

If at any point an assertion is found, the analysis is aborted for that run, i.e. 
no null pointer dereference is present.

\subsection{References}

\htmladdnormallink{EXP33C}{https://www.securecoding.cert.org/confluence/display/seccode/EXP33-C.+Do+not+reference+uninitialized+variables} , ``Do not reference uninitialized variables''\\*
\htmladdnormallink{EXP34C}{https://www.securecoding.cert.org/confluence/display/seccode/EXP34-C.+Ensure+a+pointer+is+valid+before+dereferencing+it} , ``Ensure a pointer is valid before dereferencing it''


%%%%%%%%%%%%%%%%%%%%%%%%%%%%%%%%%%%%%%%%%%%%%%%%%%%
%If there is strange known behaviour, you can write a 
%subsection that describes that problem.

