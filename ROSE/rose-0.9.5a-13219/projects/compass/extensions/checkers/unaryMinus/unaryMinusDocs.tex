% 9.5.07
% This is a sample documentation for Compass in the tex format.
% We restrict the use of tex to the following subset of commands:
%
% \section, \subsection, \subsubsection, \paragraph
% \begin{enumerate} (no-nesting), \begin{quote}, \item
% {\tt ... }, {\bf ...}, {\it ... }
% \htmladdnormallink{}{}
% \begin{verbatim}...\end{verbatim} is reserved for code segments
% ...''
%

\section{Unary Minus}
\label{UnaryMinus::overview}

% write your introduction
The unary minus operator should only be used with signed types, as its use
with unsigned types will never result in a negative value. This checker
reports any uses of the built-in unary minus operator on an unsigned type.

\subsection{Parameter Requirements}

This checker does not require any parameters.

\subsection{Non-Compliant Code Example}

\begin{verbatim}
unsigned int f_noncompliant(unsigned int u)
{
    return -u;
}
\end{verbatim}

\subsection{Compliant Solution}

\begin{verbatim}
int f_compliant(int n)
{
    return -n;
}
\end{verbatim}

\subsection{Mitigation Strategies}
\subsubsection{Static Analysis} 

Compliance with this rule can be checked using structural static analysis checkers using the following algorithm:

\begin{enumerate}
\item Check the type of the operand of any unary minus expression; emit a
diagnostic if it is an unsigned integer type.
\end{enumerate}

\subsection{References}

% Write some references
% ex. \htmladdnormallink{ISO/IEC 9899-1999:TC2}{https://www.securecoding.cert.org/confluence/display/seccode/AA.+C+References} Forward, Section 6.9.1, Function definitions''
A reference for this rule is: The Programming Research Group: ``High-Integrity
C++ Coding Standard Manual'', Item~10.21: ``Apply unary minus to operands of
signed type only.''
