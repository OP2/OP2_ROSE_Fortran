% 9.5.07
% This is a sample documentation for Compass in the tex format.
% We restrict the use of tex to the following subset of commands:
%
% \section, \subsection, \subsubsection, \paragraph
% \begin{enumerate} (no-nesting), \begin{quote}, \item
% {\tt ... }, {\bf ...}, {\it ... }
% \htmladdnormallink{}{}
% \begin{verbatim}...\end{verbatim} is reserved for code segments
% ...''
%

\section{Magic Number}
\label{MagicNumber::overview}

% write your introduction
This test checks for the presence of `magic numbers' in the source code. Magic
numbers are all constants of integer or floating point type that occur outside
of initializer expressions. The user may configure the checker to ignore
certain common constants such as 0 or 1. This detector reports not only
hand-written constants but also those that were created by macro expansion.
 
Note that C++ does not have negative constants; \verb|-1| is not an integer
constant but rather the unary minus operator applied to the constant~\verb|1|.
To ignore occurrences of \verb|-1|, you must therefore instruct the checker to
ignore the constant \verb|1|, but this will also ignore all `positive'
occurrences.

\subsection{Parameter Requirements}

The magic number detector requires two entries in the parameter file, one of
which is the list of integer constants to ignore, the other the list of
floating point constants to ignore. Constants in both lists are separated by
whitespace; as explained above, it does not make sense to specify negative
values.
 
An example of parameter entries is:
\begin{quote}
{\bf MagicNumberDetector.allowedIntegers = }

{\bf MagicNumberDetector.allowedFloats = 42.0 3.14159}
\end{quote}
 
This specification has an empty list of integers, so every integer constant
(of any type) will be flagged as a magic number; the floating point constants
42.0 and 3.14159 are allowed to appear in the source code, but
all others are treated as magic numbers. Note that floating point numbers are
compared by numeric value, which may result in strange effects due to inexact
representation.


\subsection{Non-Compliant Code Example}

\begin{verbatim}
int f_noncompliant(int n)
{
    int x;
    x = 42; // not OK: magic number
    return x + n;
}
\end{verbatim}

\subsection{Compliant Solution}

\begin{verbatim}
int f_compliant(int n)
{
    int x = 42; // OK: constant only used in initializer
    return x + n;
}
\end{verbatim}

\subsection{Mitigation Strategies}
\subsubsection{Static Analysis} 

Compliance with this rule can be checked using structural static analysis checkers using the following algorithm:

\begin{enumerate}
\item For every integer or floating point literal, examine the enclosing
statement to find out whether it occurs as part of the initializer in a
variable declaration. If not, emit a diagnostic.
\end{enumerate}

\subsection{References}

% Write some references
% ex. \htmladdnormallink{ISO/IEC 9899-1999:TC2}{https://www.securecoding.cert.org/confluence/display/seccode/AA.+C+References} Forward, Section 6.9.1, Function definitions''
A reference for this pattern is: H.~Sutter, A.~Alexandrescu: ``C++ Coding
Standards'', Item~17: ``Avoid magic numbers''. 
