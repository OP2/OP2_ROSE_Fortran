% 9.5.07
% This is a sample documentation for Compass in the tex format.
% We restrict the use of tex to the following subset of commands:
%
% \section, \subsection, \subsubsection, \paragraph
% \begin{enumerate} (no-nesting), \begin{quote}, \item
% {\tt ... }, {\bf ...}, {\it ... }
% \htmladdnormallink{}{}
% \begin{verbatim}...\end{verbatim} is reserved for code segments
% ...''
%

\section{Allocate And Free Memory In The Same Module At The Same Level Of Abstraction}
\label{AllocateAndFreeMemoryInTheSameModule::overview}

% write your introduction
Allocating and freeing memory in different modules and levels of abstraction
burdens the programmer with tracking the lifetime of that block of memory. This
may cause confusion regarding when and if a block of memory has been allocated
or freed, leading to programming defects such as double-free vulnerabilities,
accessing freed memory, or writing to unallocated memory.

To avoid these situations, it is recommended that memory be allocated and freed
at the same level of abstraction, and ideally in the same code module.

The affects of not following this recommendation are best demonstrated by an
actual vulnerability. Freeing memory in different modules resulted in a
vulnerability in MIT Kerberos 5 MITKRB5-SA-2004-002 . The problem was that the
MIT Kerberos 5 code contained error-handling logic, which freed memory allocated
by the ASN.1 decoders if pointers to the allocated memory were non-NULL.
However, if a detectable error occured, the ASN.1 decoders freed the memory that
they had allocated. When some library functions received errors from the ASN.1
decoders, they also attempted to free, causing a double-free vulnerability.

\subsection{Parameter Requirements}

%Write the Parameter specification here.
   No Parameter specifications.

\subsection{Implementation}

%Details of the implementation go here.
   No implementation yet!

\subsection{Non-Compliant Code Example}

% write your non-compliant code subsection
This example demonstrates an error that can occur when memory is freed in
different functions. The array, which is referred to by list and its size,
number, are then passed to the verify\_list() function. If the number of
elements in the array is less than the value MIN\_SIZE\_ALLOWED, list is
processed. Otherwise, it is assumed an error has occurred, list is freed, and
the function returns. If the error occurs in verify\_list(), the dynamic memory
referred to by list will be freed twice: once in verify\_list() and again at the
end of process\_list().

\begin{verbatim}

% write your non-compliant code example
int verify_size(char *list, size_t list_size) {
  if (size < MIN_SIZE_ALLOWED) {
    /* Handle Error Condition */
    free(list);
    return -1;
  }
  return 0;
}

void process_list(size_t number) {
  char *list = malloc(number);

  if (list == NULL) {
    /* Handle Allocation Error */
  }

  if (verify_size(list, number) == -1) {
    /* Handle Error */

  }

  /* Continue Processing list */

  free(list);
}

\end{verbatim}

\subsection{Compliant Solution}

% write your compliant code subsection
To correct this problem, the logic in the error handling code in verify\_list()
should be changed so that it no longer frees list. This change ensures that list
is freed only once, in process\_list().

\begin{verbatim}

% write your compliant code example
int verify_size(char *list, size_t list_size) {
  if (size < MIN_SIZE_ALLOWED) {
    /* Handle Error Condition */
    return -1;
  }
  return 0;
}

void process_list(size_t number) {
  char *list = malloc(number);

  if (list == NULL) {
    /* Handle Allocation Error */
  }

  if (verify_size(list, number) == -1) {
    /* Handle Error */
  }

  /* Continue Processing list */

  free(list);
}

\end{verbatim}

\subsection{Mitigation Strategies}
\subsubsection{Static Analysis} 

Compliance with this rule can be checked using structural static analysis checkers using the following algorithm:

\begin{enumerate}
\item Write your checker algorithm
\end{enumerate}

\subsection{References}

% Write some references
% ex. \htmladdnormallink{ISO/IEC 9899-1999:TC2}{https://www.securecoding.cert.org/confluence/display/seccode/AA.+C+References} Forward, Section 6.9.1, Function definitions''

\htmladdnormallink{ISO/IEC9899-1999}{https://www.securecoding.cert.org/confluence/display/seccode/MEM00-A.+Allocate+and+free+memory+in+the+same+module\%2C+at+the+same+level+of+abstraction} MEM00-A. Allocate and free memory in the same module, at the same level of abstraction
