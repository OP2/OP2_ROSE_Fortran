% This is a sample documentation for Compass in the tex format.
% We restrict the use of tex to the following subset of commands:
%
% \section, \subsection, \subsubsection, \paragraph
% \begin{enumerate} (no-nesting), \begin{quote}, \item
% {\tt ... }, {\bf ...}, {\it ... }
% \htmladdnormallink{}{}
% \begin{verbatim}...\end{verbatim} is reserved for code segments
% ...''
%

\section{Lower Range Limit}
\label{LowerRangeLimit::overview}

By always using inclusive lower limits and exclusive upper
limits, a whole class of off-
by-one errors is eliminated. Furthermore, the following
assumptions always apply:
1)the size of the interval equals the
difference of the two  \\
2) the limits are equal if the interval is empty\\
3) the upper limit is never less than the lower limit\\

Examples: instead of saying x>=23 and x<=42, use x>=23 and
x<43.


\subsection{Parameter Requirements}

%Write the Parameter specification here.
   No parameters required.

\subsection{Implementation}

%Details of the implementation go here.
   In a fairly straight-forward implementation we search the strictly
   lower than operator.  

\subsection{Non-Compliant Code Example}

% write your non-compliant code subsection


\begin{verbatim}
int x = 5;
if (x < 5)
{
x++;
}
% write your non-compliant code example

\end{verbatim}

\subsection{Compliant Solution}

% write your compliant code subsection

\begin{verbatim}
int x = 5;
if (x <= 4)
{
x++;
}
% write your compliant code example

\end{verbatim}

\subsection{Mitigation Strategies}
\subsubsection{Static Analysis} 

Compliance with this rule can be checked using structural static analysis checkers using the following algorithm:

\begin{enumerate}
\item find less than operator 
\item raise alert 
\end{enumerate}

\subsection{References}

% Write some references
% ex. \htmladdnormallink{ISO/IEC 9899-1999:TC2}{https://www.securecoding.cert.org/confluence/display/seccode/AA.+C+References} Forward, Section 6.9.1, Function definitions''
\htmladdnormallink{Abbreviated Code Inspection
  Checklist}{http://www.lore.ua.ac.be/Teaching/SE3BAC} Section 11.1.1,
Control Variables''
