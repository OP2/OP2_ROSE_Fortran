%@@%General Suggestion: Ideally, the documentation of a style checker should 
%@@%be around one page.
%@@\section{Asynchronous Signal Handler}
%@@
%@@\label{AsynchronousSignalHandler::overview}
%@@
%@@\quote{Namespace: \texttt{AsynchronousSignalHandler}}
%@@
%@@\subsection{Introduction}
%@@
%@@%Write your introduction here.
%@@   No introduction yet!
%@@
%@@%Give an exact reference to the pattern in the literature.
%@@%Here is the example:
%@@%The reference for this checker is: H. Sutter, A. Alexandrescu:
%@@%``C++ Coding Standards'', Item 28: ``Prefer the canonical form ++ and --. 
%@@%Prefer calling the prefix forms''.
%@@
%@@No reference to literature as yet!
%@@
%@@\subsection{Parameter Requirements}
%@@
%@@%Write the Parameter specification here.
%@@   No Parameter specifications yet!
%@@
%@@\subsection{Implementation}
%@@
%@@%Details of the implementation go here.
%@@   No implementation yet!
%@@
%@@\subsection{Example of Failing Output Code}
%@@
%@@%Examples of the patterns go here.
%@@     See example: asynchronousSignalHandlerTest1.C
%@@
%@@%The following lines are for references to the examples in the
%@@%documentation.
%@@\begin{latexonly}
%@@{\codeFontSize
%@@\lstinputlisting{\ExampleDirectory/../asynchronousSignalHandler/asynchronousSignalHandlerTest1.C}
%@@}
%@@\end{latexonly}
%@@
%@@%%%%%%%%%%%%%%%%%%%%%%%%%%%%%%%%%%%%%%%%%%%%%%%%%%%
%@@%If there is strange known behaviour, you can write a 
%@@%subsection that describes that problem.
%@@
