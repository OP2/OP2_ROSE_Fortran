% 9.5.07
% This is a sample documentation for Compass in the tex format.
% We restrict the use of tex to the following subset of commands:
%
% \section, \subsection, \subsubsection, \paragraph
% \begin{enumerate} (no-nesting), \begin{quote}, \item
% {\tt ... }, {\bf ...}, {\it ... }
% \htmladdnormallink{}{}
% \begin{verbatim}...\end{verbatim} is reserved for code segments
% ...''
%

\section{Malloc Return Value Used In If Stmt}
\label{MallocReturnValueUsedInIfStmt::overview}

``ALE3D Coding Standards \& Style Guide'' item \#4.5 states that
\begin{quote}
When using raw {\tt malloc()} and {\tt new}, developers should check the return value for {\tt NULL}. This is especially important when allocating large blocks of memory, which may exhaust heap resources.
\end{quote}

\subsection{Parameter Requirements}
This checker takes no parameters and inputs source file.

\subsection{Implementation}
This pattern is checked using a simple AST traversal that seeks out function references to malloc. Then the parent nodes are traversed up until a basic scope block is found at which point a nested AST traversal seeks If-statement conditional expressions containing the memory block returned from malloc. If no such If-statement conditional is found in the immediate basic containing block scope then an error is flagged.

\subsection{Non-Compliant Code Example}
The non-compliant code fails to check the return value of {\tt malloc()}.

\begin{verbatim}
#include <stdlib.h>

int main()
{
  int *iptr = (int*)malloc( 256*sizeof(int) );

  return 0;
} //main()
\end{verbatim}

\subsection{Compliant Solution}
The compliant solution uses an if statement to check the return value of {\tt malloc()} for {\tt NULL}.

\begin{verbatim}
#include <stdlib.h>

int main()
{
  int *iptr = (int*)malloc( 256*sizeof(int) );

  if( iptr == NULL )
    return 1;

  return 0;
} //main()
\end{verbatim}

\subsection{Mitigation Strategies}
\subsubsection{Static Analysis} 

Compliance with this rule can be checked using structural static analysis checkers using the following algorithm:

\begin{enumerate}
\item Perform AST traversal visiting function call nodes corresponding to {\tt malloc()}.
\item For each call to {\tt malloc()} traverse its parent nodes until an if statement or the end of a basic block is reached.
\item If an if statement is encountered, check that the if statement performs a comparison involving the return value from {\tt malloc()}; if this is not the case then flag a violation.
\item If a basic block is reached, then flag a violation as the return value of {\tt malloc()} may be out of scope.
\item Report any violations. 
\end{enumerate}

\subsection{References}

Arrighi B., Neely R., Reus J. ``ALE3D Coding Standards \& Style Guide'', 2005.
