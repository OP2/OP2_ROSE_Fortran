%General Suggestion: Ideally, the documentation of a style checker should 
%be around one page.
\section{[No Reference] : New Delete}

This analysis checks for the validity of all delete operations in a specific source code.
It checks for violations of:

a) Deleting an array with a simple delete operator instead of an delete array operator.

b) Deleting a NULL pointer.

c) Checks for the deletion of uninitialized pointers.


\subsection{Non-Compliant Code Examples}

In the following examples, a) b) and c) from above are demonstrated.

\begin{verbatim}
class Y {
  int y;
};

void fail() {
  int x=2;
  // deleting array
  Y* m = new Y[5];
  delete m;

  // deleting NULL
  Y* n = 0;
  delete n;

  // deleting NULL
  Y* c ;
  if (x==5) {
    x=7;
    delete c;
  }
}
\end{verbatim}

\subsection{Compliant Solution}
Use assert statements to make sure that a pointer cannot be null,
or use the delete[] operator if a new[] operator precedes on that pointer. 

\subsection{Parameter Requirements}

None.


\subsection{Implementation}

This analysis uses the BOOST library in order to utilize the breadth first search (BFS) algorithm.
Together with the control flow graph and the BFS, the implementation backtracks the code from the
delete operation to its definition. On violations of the described cases, the analysis results in warnings.

The algorithm searches first for each occurance of a SgDeleteExp and backtracks this Node to
its definition. If we find a SgNew operation, we need to see if the delete and new operations match,
i.e. whether they are both operations on pointers or arrays.

The following cases are checked for and handled during the backwards dataflow analysis:
\begin{verbatim}
  case V_SgNewExp: 
  case V_SgVarRefExp:   
  case V_SgAddressOfOp: 
  case V_SgCastExp:
  case V_SgIntVal: 
\end{verbatim}

The above indicates a recursive algorithm.

\subsection{References}

