\section{Creating a 'struct' for Global Variables}
\fixme{TODO: This tutorial uses low level AST manipulation. We should have a more
concise version using SageInterface and SageBuilder functions.}
   This is an example written to support the Charm++ tool. This translator
extracts global variables from the program and builds a structure to hold them.
The support is part of a number of requirements associated with using Charm++
and AMPI.

   Figure~\ref{Tutorial:exampleGlobalVariableHandling} shows repackaging of global
variables within an application into a struct. All reference to the global variables
are also transformed to reference the original variable indirectly through the structure.
This processing is part of preprocessing to use Charm++.  

   This example shows the low level handling directly at the level of the IR.

\begin{figure}[!h]
{\indent
{\mySmallestFontSize


% Do this when processing latex to generate non-html (not using latex2html)
\begin{latexonly}
%  \lstinputlisting{\TutorialExampleDirectory/addFunctionDeclaration.C}
   \lstinputlisting{\TutorialExampleBuildDirectory/CharmSupport.aa}
\end{latexonly}

% Do this when processing latex to build html (using latex2html)
\begin{htmlonly}
   \verbatiminput{\TutorialExampleDirectory/CharmSupport.C}
\end{htmlonly}

% end of scope in font size
}
% End of scope in indentation
}
\caption{Example source code shows repackaging of global variables to a struct (part 1).}
\label{Tutorial:exampleGlobalVariableHandling}
\end{figure}

\begin{figure}[!h]
{\indent
{\mySmallestFontSize


% Do this when processing latex to generate non-html (not using latex2html)
\begin{latexonly}
   \lstinputlisting{\TutorialExampleBuildDirectory/CharmSupport.ab}
\end{latexonly}

% Do this when processing latex to build html (using latex2html)
\begin{htmlonly}
%   \verbatiminput{\TutorialExampleDirectory/addFunctionDeclaration.C}
\end{htmlonly}

% end of scope in font size
}
% End of scope in indentation
}
\caption{Example source code shows repackaging of global variables to a struct (part 2).}
\label{Tutorial:exampleGlobalVariableHandling2}
\end{figure}

\begin{figure}[!h]
{\indent
{\mySmallestFontSize


% Do this when processing latex to generate non-html (not using latex2html)
\begin{latexonly}
   \lstinputlisting{\TutorialExampleBuildDirectory/CharmSupport.ac}
\end{latexonly}

% Do this when processing latex to build html (using latex2html)
\begin{htmlonly}
%   \verbatiminput{\TutorialExampleDirectory/addFunctionDeclaration.C}
\end{htmlonly}

% end of scope in font size
}
% End of scope in indentation
}
\caption{Example source code shows repackaging of global variables to a struct (part 3).}
\label{Tutorial:exampleGlobalVariableHandling3}
\end{figure}

\begin{figure}[!h]
{\indent
{\mySmallestFontSize


% Do this when processing latex to generate non-html (not using latex2html)
\begin{latexonly}
   \lstinputlisting{\TutorialExampleBuildDirectory/CharmSupport.ad}
\end{latexonly}

% Do this when processing latex to build html (using latex2html)
\begin{htmlonly}
%   \verbatiminput{\TutorialExampleDirectory/addFunctionDeclaration.C}
\end{htmlonly}

% end of scope in font size
}
% End of scope in indentation
}
\caption{Example source code shows repackaging of global variables to a struct (part 4).}
\label{Tutorial:exampleGlobalVariableHandling4}
\end{figure}

\begin{figure}[!h]
{\indent
{\mySmallestFontSize


% Do this when processing latex to generate non-html (not using latex2html)
\begin{latexonly}
   \lstinputlisting{\TutorialExampleBuildDirectory/CharmSupport.ae}
\end{latexonly}

% Do this when processing latex to build html (using latex2html)
\begin{htmlonly}
%   \verbatiminput{\TutorialExampleDirectory/addFunctionDeclaration.C}
\end{htmlonly}

% end of scope in font size
}
% End of scope in indentation
}
\caption{Example source code shows repackaging of global variables to a struct (part 5).}
\label{Tutorial:exampleGlobalVariableHandling5}
\end{figure}

\begin{figure}[!h]
{\indent
{\mySmallFontSize


% Do this when processing latex to generate non-html (not using latex2html)
\begin{latexonly}
   \lstinputlisting{\TutorialExampleDirectory/inputCode_ExampleCharmSupport.C}
\end{latexonly}

% Do this when processing latex to build html (using latex2html)
\begin{htmlonly}
   \verbatiminput{\TutorialExampleDirectory/inputCode_ExampleCharmSupport.C}
\end{htmlonly}

% end of scope in font size
}
% End of scope in indentation
}
\caption{Example source code used as input to translator adding new function.}
\label{Tutorial:exampleInputCode_AddFunctionDeclaration}
\end{figure}

\begin{figure}[!h]
{\indent
{\mySmallFontSize


% Do this when processing latex to generate non-html (not using latex2html)
\begin{latexonly}
   \lstinputlisting{\TutorialExampleBuildDirectory/rose_inputCode_ExampleCharmSupport.C}
\end{latexonly}

% Do this when processing latex to build html (using latex2html)
\begin{htmlonly}
   \verbatiminput{\TutorialExampleBuildDirectory/rose_inputCode_ExampleCharmSupport.C}
\end{htmlonly}

% end of scope in font size
}
% End of scope in indentation
}
\caption{Output of input to translator adding new function.}
\label{Tutorial:exampleOutput_AddFunctionDeclaration}
\end{figure}


