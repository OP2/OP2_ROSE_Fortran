\chapter{The ROSE Infrastructure}

\label{overviewOfRose:overviewOfRose}

% \fixme{This chapter is incomplete and will be filled in by Rich.}

\commentout{
This chapter includes many details about how to
use ROSE. This is a chapter that I think Rich would like to write
(or at least start and contribute to it).  It will cover:
\begin{itemize}
   \item ROSE documentation
   \begin{itemize}
      \item ROSE features
      \item Language front-ends
      \item Code generation
      \item ROSE directory organization
   \end{itemize}
   \item SAGE III Overview
   \begin{itemize}
      \item Figure to describe IR node hierarchy
      \item description of categories of IR nodes
      \item relationship to C++ grammar
      \item Integrate with current "Sage III Intermediate Representation" (current Chapter 10).
      \item Properties of IR nodes (No null pointers, what few pointers are NULL, what IR nodes are shared,defining/non-defining declarations)
      \item AST Merge Mechanism
      \item File I/O
      \item How modifiers are handled
   \end{itemize}
\end{itemize}
}

\section{Introduction}
     This chapter was requested by several people who wanted to understand 
how ROSE was designed and implemented.  ROSE supports a number of different 
languages and used different parsers and or frontends to address each on.
For C, C99, UPC, and C++; we use the EDG frontend.  While for Fortran we
use the Open Fortran Parser as a parser and build the fronend end required.
ROSE contains a midend, where analysis support is made available and and backend 
which does the code generation from the IR.

   The goal of the design of the IR is to not loose any source code information.
Thus ROSE is especially well suited to source-to-source translation. However
then means that the IR for ROSE is quite large and this has advantages and
disadvantages.  The IR forms the base for an abstract syntax tree, so clearly
some syntatic details are lost in the IR, but these are regenerated in the 
back-end (which has language specific support).

   More languages could be added to ROSE, ROSE is designed to be langauge
neutral, but it is implemented in C++. PHP has for example been added to 
ROSE, but it represnted initial work and an experiment with the general subject of
run-time typed scripting language support.

\section{Design}

Fundamentally, ROSE has three parts:
\begin{enumerate}
   \item frontend, which addresses language specific parsers/frontend issues (and the
         binary disassembly for the case of the binary support in ROSE);
   \item midend, which addresses analysis and transformation issues; 
   \item backend which addresses code generation issues.
\end{enumerate}

   The frontend constructs an AST which saves as much as possible about the structure of
the original source code (or binary for the case of the ROSE binary supposrt).

   This section will cover the design goals etc. of ROSE.

\section{Directory Structure}

   The top level of the ROSE directory tree has a simple design.
All the source code is in {\tt src}, all the tests are in {\tt tests},
all the documentation is in {\tt docs}.  ROSE uses {\bf autoconf} and {\bf automake}
so there is an autoconf generated {\em configure} script included.  The
{\tt conf} directory contains all the {\em autconf macros} used in ROSE.
The {\tt projects} directory contains a collection of ongoing and past
projects in ROSE that are either not large enough or mature enough to stand
along as sperate projects.  We use this location to incubate developing tools
or technologies built on ROSE, as they are developed some are moved
into the ROSE {\tt src} directory proper.  The README file contains
information on how to install ROSE, and information about where information
on ROSEis located.

\fixme{Add more detail about each directory.}

% the directory figure is too huge to be useful, also as more stuff being
% added in, LaTex complains Dimension too large quite often. Liao
\begin{comment}
\begin{latexonly}
% Do this when processing latex to generate non-html (not using latex2html)
\begin{figure}[tb]
%\centerline{\epsfig{file=roseDirectoryMap.ps,height=1.0\linewidth,width=1.4\linewidth,angle=270}}
%\centerline{\psfig{file=roseDirectoryMap.pdf,height=1.0\linewidth,width=1.4\linewidth,angle=270}}
\includegraphics[scale=0.12,angle=270]{roseDirectoryMap}
\caption{Directory structure of ROSE project.}
\end{figure}
\end{latexonly}

\begin{htmlonly}
% Do this when processing latex to generate non-html (not using latex2html) 
% Note that specification of the angle causes latex2html to SKIP output of 
% the figure and the "height=1.5\linewidth,width=4.0\linewidth" is also required
% (I suspect this is a bug in latex2html).
\begin{figure}[tb]
%\centerline{\epsfig{file=ROSE.ps,height=1.0\linewidth,width=1.0\linewidth,angle=0}}
%\centerline{\epsfig{file=roseDirectoryMap.ps,height=1.5\linewidth,width=4.0\linewidth,angle=0}}
%\centerline{\epsfig{file=roseDirectoryMap.ps,height=1.5\linewidth,width=4.0\linewidth}}
%\centerline{\epsfig{file=roseDirectoryMap.ps}}

%\centerline{\psfig{file=roseDirectoryMap.pdf,height=1.5\linewidth,width=4.0\linewidth}}
\includegraphics[scale=0.25,angle=270]{roseDirectoryMap}
\caption{Directory structure of ROSE project.  This figure is formed from a Perl script
    ({\tt ROSE/scripts/lsvis}) which generates the graph when the documentation is built.}
\label{designAndImplementation:directoryStructure}
\end{figure}
\end{htmlonly}

\end{comment}

\section{Implementation of ROSE}
   ROSE is implemented in C++.  It supports source-to-source
analysis and transformations on source code in a language neutral
way (or alternativeively in a collection of langauge specific ways).

     This section will be added to in the future.
     
\subsection{Implementation of ROSETTA}

    ROSETTA is a tools built internally to generate code for ROSE
so that ROSE follows simple and consistant design rules.  ROSE
relies heavily on code generatation as a way to automate as much
as possible and permits ROSE to be maintained by as easily as possible.
ROSETTA is thus used so that we can avoid spending all our time 
doing mainainance.  ROSETTA is however not very ROSE specific and
might be more generally useful, we have not pursued this line of work.
We are happy to have ROSETTA be only used in ROSE, it is however
separated out in the {\tt src/ROSETTA/src} and {\tt src/ROSETTA/Grammar}
directories.

     This section will be added to in the future.



\subsection{Implementation of Fortran support}

    All fortran support in ROSE used the Open Fortran Parser (OFP)
developed at Los Alamos and part of a community effort to
define an open fortran parser that tracks the Fortran language
(supports Fortran 2003 and the anticipated Fortran 2008).
ROSE uses the OFP and builds from the parser the implementations
of the parser actions required to construct a proper Fortran frontend.
That the Fortran frontendin ROSE uses the ROSE IR means that the
analysis in the midend can be used (or has been fixed up for use with Fortran).
A backend is also defined in ROSE so that source-to-source support for 
Fortran is provide.  
