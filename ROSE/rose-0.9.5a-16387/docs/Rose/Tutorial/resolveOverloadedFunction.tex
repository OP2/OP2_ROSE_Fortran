\chapter{Resolving Overloaded Functions}

\begin{figure}[!h]
{\indent
{\mySmallFontSize


% Do this when processing latex to generate non-html (not using latex2html)
\begin{latexonly}
   \lstinputlisting{\TutorialExampleDirectory/resolveOverloadedFunction.C}
\end{latexonly}

% Do this when processing latex to build html (using latex2html)
\begin{htmlonly}
   \verbatiminput{\TutorialExampleDirectory/resolveOverloadedFunction.C}
\end{htmlonly}

% end of scope in font size
}
% End of scope in indentation
}
\caption{Example source code showing mapping of function calls to overloaded function declarations.}
\label{Tutorial:exampleResolvingOverloadedFunctions}
\end{figure}

   Figure~\ref{Tutorial:exampleResolvingOverloadedFunctions} shows a translator which
reads an application and reposts on the mapping between function calls and function
declarations.  This is trivial since all overloaded function resolution is done within
the frontend and so need not be computed (this is because all type resolution is done 
in the frontend and stored in the AST explicitly).  Other compiler infrastructures often
require this to be figured out from the AST, when type resolution is unavailable, and
while not too hard for C, this is particularly complex for C++ (due to overloading
and type promotion within function arguments).  

   Figure~\ref{Tutorial:exampleInputCode_ResolvingOverloadedFunctions} shows the
input code used to get the translator.
Figure~\ref{Tutorial:exampleOutput_ResolvingOverloadedFunctions} shows the resulting output.


\begin{figure}[!h]
{\indent
{\mySmallFontSize


% Do this when processing latex to generate non-html (not using latex2html)
\begin{latexonly}
   \lstinputlisting{\TutorialExampleDirectory/inputCode_ResolvingOverloadedFunctions.C}
\end{latexonly}

% Do this when processing latex to build html (using latex2html)
\begin{htmlonly}
   \verbatiminput{\TutorialExampleDirectory/inputCode_ResolvingOverloadedFunctions.C}
\end{htmlonly}

% end of scope in font size
}
% End of scope in indentation
}
\caption{Example source code used as input to resolveOverloadedFunction.C.}
\label{Tutorial:exampleInputCode_ResolvingOverloadedFunctions}
\end{figure}

\begin{figure}[!h]
{\indent
{\mySmallFontSize


% Do this when processing latex to generate non-html (not using latex2html)
\begin{latexonly}
   \lstinputlisting{\TutorialExampleBuildDirectory/resolveOverloadedFunction.out}
\end{latexonly}

% Do this when processing latex to build html (using latex2html)
\begin{htmlonly}
   \verbatiminput{\TutorialExampleBuildDirectory/resolveOverloadedFunction.out}
\end{htmlonly}

% end of scope in font size
}
% End of scope in indentation
}
\caption{Output of input to resolveOverloadedFunction.C.}
\label{Tutorial:exampleOutput_ResolvingOverloadedFunctions}
\end{figure}



