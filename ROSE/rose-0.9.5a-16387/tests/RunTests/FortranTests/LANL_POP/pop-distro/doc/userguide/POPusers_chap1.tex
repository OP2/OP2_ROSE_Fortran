\chapter{Introduction}\label{ch:intro}

Here we present a bit of POP history so you may
marvel at the once-revolutionary advances and rest
secure in the knowledge that we are working diligently
(and sometimes even successfully) to continue such advances
in the future.  If you are bored with such historical
ramblings or simply wish to rush into actually running
ocean simulations, feel free to skip over this section
and perhaps read it later over a nice cappuccino, or
a glass of port beside the fire, or maybe a beer at
your favorite brew pub.

\section{Brief history of POP development}\label{sec:POPhistory}

The Parallel Ocean Program (POP) was developed at LANL under
the sponsorship of the Department of Energy's CHAMMP program, which
brought massively parallel computers to the realm of climate modeling.
POP is a descendent of the Bryan-Cox-Semtner class of ocean
models first developed by Kirk Bryan and Michael Cox \cite{Bryan69} at
the NOAA Geophysical Fluid Dynamics Laboratory in Princeton, NJ in
the late 1960�s. POP had its origins in a version of the model
developed by  Semtner and Chervin \cite{Semtner86}
\cite{ChervinSemtner88}.  The complete
``family tree'' of the BCS models is displayed in
\hyperref{the figure below}
         {Figure }{}
         {fig:POPtree}
(courtesy of Bert Semtner, Naval Postgraduate School).

\begin{figure}\label{fig:POPtree}
\caption{Bryan-Cox-Semtner model family tree}
\begin{htmlonly}
\htmladdimg{SRC="POPfamily.gif" ALT="POPFamilyTree"}
\end{htmlonly}
\begin{latexonly}
Figure currently exists in on-line version only.
\end{latexonly}
\end{figure}

Under the CHAMMP program, the Semtner-Chervin version was
rewritten in CM Fortran for the Connection Machine
CM-2 and CM-5 massively parallel computers. Experience
with the resulting model led to a number of changes
resulting in what is now known as the Parallel Ocean
Program (POP).  Although originally motivated by the adaptation
of POP for massively parallel computers, many of these changes
improved not only its computational performance but the fidelity
of the model�s physical representation of the real ocean. The
most significant of these improvements are summarized below.
Details can be found in articles by Smith, \cite{SmithEtal92},
Dukowicz et al., \cite{DukEtal93}, and
Dukowicz and Smith \cite{DukSmith94}.  The model has
continued to develop to adapt to new machines, incorporate
new numerical algorithms and introduce new physical
parameterizations.

\section{Improvements introduced in POP}\label{sec:POPimprove}

\subsection{Surface-pressure formulation of barotropic mode}
\label{sec:improve-psfc}

The barotropic streamfunction formulation in the standard
BCS models required an additional equation to be solved for each
continent and island that penetrated the ocean surface. This was
costly even on machines like Cray parallel-vector-processor computers,
which had fast memory access. To reduce the number of equations to
solve with the barotropic streamfunction formulation, it was common
practice to submerge islands, connect them to nearby continents with
artificial land bridges, or merge an island chain into a single mass
without gaps.�The first modification created artificial gaps,
permitting increased flow, while the latter two closed channels that
should exist.

On distributed-memory parallel computers, these added equations
were even more costly because each required gathering data from an
arbitrarily large set of processors to perform a line-integral around
each landmass. This computational dilemma was addressed by developing
a new formulation of the barotropic mode based on surface pressure.
The boundary condition for the surface pressure at a land-ocean
interface point is local, which eliminates the non-local line-integral.

Consequently, the surface-pressure formulation permits any
number of islands to be included at no additional computational cost,
so all channels can be treated as precisely as the resolution of the
grid permits.

Another problem with the barotropic streamfunction formulation is that
the elliptic problem to be solved is ill-conditioned if bottom
topography has large spatial gradients.  The bottom topography must be
smoothed to maintain numerical stability.  Although this reduces the
fidelity of the simulation, it does have the �desirable� side effect
(given the other limitations of the streamfunction approach mentioned
above) of submerging many islands, thereby reducing the number of
equations to be solved. In contrast, the surface-pressure formulation
allows more realistic, unsmoothed bottom topography to be used with
no reduction in time step.

\subsection{Free-surface boundary condition}\label{sec:improve-freesfc}

The original ``rigid-lid'' boundary condition was replaced by an
implicit free-surface boundary condition that allows the air-sea
interface to evolve freely and makes sea-surface height a prognostic
variable.

\subsection{Latitudinal scaling of horizontal diffusion}
\label{sec:improve-latscale}

Scaling of the horizontal diffusion coefficient by $\cos^n(\theta)$
was introduced, where $\theta$ is latitude, $n=1$ for Laplacian mixing
and $n=3$ for bi-harmonic mixing. This {\em optional} scaling
prevents horizontal diffusion from limiting the time step severely at
high latitudes, yet keeps diffusion large enough to maintain numerical
stability.

\subsection{Pressure-averaging}\label{sec:improve-pressavg}

After the temperature and salinity have been updated to
time-step $n+1$ in the baroclinic routines, the density $\rho^{n+1}$
and pressure $p^{n+1}$ can be computed. By computing the pressure
gradient with a linear combination of $p$ at three time-levels
($n-1$, $n$, and $n+1$), a technique well known in atmospheric
modeling \cite{BrownCampana78}, it is possible to increase the
time-step by as much as a factor of two.

\subsection{Designed for parallel computers}\label{sec:improve-parallel}

The code is written in Fortran90 and can be run on a variety
of parallel and serial computer architectures. Originally,
the code was written using a data-parallel approach for the
Thinking Machines Connection Machine.  Later versions used
a more traditional domain decomposition style using MPI
or SHMEM for inter-processor communications.  The most
recent version of the code supports current clusters
of shared-memory multi-processor nodes through the use of
thread-based parallism (OpenMP) between processors on a
node and message-passing (MPI or SHMEM) for communication
between nodes.  The flexibility of mixing thread-based
and message-passing programming models gives the user the
option of choosing the best combination of styles to suit
a given machine.

\subsection{General orthogonal coordinates in horizontal}
\label{sec:improve-grid}

The primitive equations were reformulated and discretized
to allow the use of any locally orthogonal horizontal grid. This
provides alternatives to the standard latitude-longitude grid with
its singularity at the North Pole.

This generalization made possible the development of the
\hyperref{displaced-pole grid}
         {displaced-pole grid (Fig.}{)}
         {fig:dpgrid},
which moves
the singularity arising from convergence of meridians at
the North Pole into an adjacent landmass such as North America,
Russia or Greenland. Such a displaced pole leaves a smooth,
singularity-free grid in the Arctic Ocean. That grid joins
smoothly at the equator with a standard Mercator grid in the
Southern Hemisphere.  The most recent versions of the code
also support a
\hyperref{tripole grid}
         {tripole grid (Fig.}{)}
         {fig:tripole}
in which two poles can be placed opposite each other in land
masses near the North Pole to give more uniform grid spacing
in the Arctic Ocean while maintaining all the advantages
of the displaced pole grids.

\begin{figure}\label{fig:dpgrid}
\caption{Displaced-pole grid}
\begin{htmlonly}
\htmladdimg{SRC="POPdpgrid.jpg" ALT="POPDipoleGrid"}
\end{htmlonly}
\begin{latexonly}
Figure currently exists in on-line version only.
\end{latexonly}
\end{figure}

\begin{figure}\label{fig:tripole}
\caption{Tripole grid}
\begin{htmlonly}
\htmladdimg{SRC="POPtripole.jpg" ALT="POPTripoleGrid"}
\end{htmlonly}
\begin{latexonly}
Figure currently exists in on-line version only.
\end{latexonly}
\end{figure}

\section{POP applications to date}\label{sec:applications}

\subsection{High-resolution global and regional modeling}
\label{sec:high-res}

In the period 1994-97, POP was used to perform high resolution
($0.28^\circ$ at the Equator) global ocean simulations, running on the
Thinking Machines CM5 computer then located at LANL's Advanced
Computing Laboratory. (Output from these runs is available at
\htmladdnormallink{http://climate.acl.lanl.gov}
                  {http://climate.acl.lanl.gov}.
The primary motivation for performing such high-resolution simulations
is to resolve mesoscale eddies that play an important role in the
dynamics of the ocean. Comparison of sea-surface
height variability measured by the TOPEX/Poseidon satellite with that
simulated by POP gave convincing evidence that still higher resolution
was required (Fu and Smith \cite{FuSmith96};
Maltrud \cite{MaltrudEtal98}).

At the time, it was not possible to do a higher resolution calculation
on the global scale, so an Atlantic Ocean simulation was done with
$0.1^\circ$ resolution at the Equator. This calculation agreed well
with observations of sea-surface height variability in
the Gulf Stream.  Many other features of the flow were also well
simulated \cite{SmithEtal99}.  Using the $0.1^\circ$ case as a
benchmark, lower resolution cases were done at $0.2^\circ$ and
$0.4^\circ$; a comparison can be found at
\htmladdnormallink{http://neit.cgd.ucar.edu/oce/bryan/woce-poster.html}
                  {http://neit.cgd.ucar.edu/oce/bryan/woce-poster.html}.

Recently, it has become possible to begin a $0.1^\circ$ global
simulation and such a simulation has been started.  In addition,
higher resolution North Atlantic simulations have also been
initiated.

\subsection{Coupled models}\label{sec:coupled-sims}

POP and the Los Alamos sea-ice model (CICE) have been adopted 
as the ocean and sea ice components of the Community Climate 
System Model (CCSM) at NCAR.  POP and CICE are
also being used in coupled model development efforts at Colorado
State University and UCLA. Information on CICE can be found at
\htmladdnormallink{http://climate.acl.lanl.gov/models/cice}
                  {http://climate.acl.lanl.gov/models/cice}.
