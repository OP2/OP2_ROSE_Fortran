% 9.5.07
% This is a sample documentation for Compass in the tex format.
% We restrict the use of tex to the following subset of commands:
%
% \section, \subsection, \subsubsection, \paragraph
% \begin{enumerate} (no-nesting), \begin{quote}, \item
% {\tt ... }, {\bf ...}, {\it ... }
% \htmladdnormallink{}{}
% \begin{verbatim}...\end{verbatim} is reserved for code segments
% ...''
%

\section{Avoid Using The Same Handler For Multiple Signals}
\label{AvoidUsingTheSameHandlerForMultipleSignals::overview}

It is possible to safely use the same handler for multiple signals, but doing so
increases the likelihood of a security vulnerability. The delivered signal is
masked and is not delivered until the registered signal handler exits. However,
if this same handler is registered to handle a different signal, execution of
the handler may be interrupted by this new signal. If a signal handler is
constructed with the expectation that it cannot be interrupted, a vulnerability
might exist. To eliminate this attack vector, each signal handler should be
registered to handle only one type of signal.

\subsection{Parameter Requirements}

%Write the Parameter specification here.
   No Parameter specifications.

\subsection{Implementation}

%Details of the implementation go here.
   No implementation yet!

\subsection{Non-Compliant Code Example}

% write your non-compliant code subsection
This non-compliant program registers a single signal handler to process both
SIGUSR1 and SIGUSR2. The variable sig2 should be set to one if one or more
SIGUSR1 signals are followed by SIGUSR2.

\begin{verbatim}

% write your non-compliant code example
#include <signal.h>
#include <stdlib.h>
#include <string.h>

volatile sig_atomic_t sig1 = 0;
volatile sig_atomic_t sig2 = 0;

void handler(int signum) {
  if (sig1) {
     sig2 = 1;
  }
  if (signum == SIGUSR1) {
    sig1 = 1;
  }
}

int main(void) {
  signal(SIGUSR1, handler);
  signal(SIGUSR2, handler);

  while (1) {
    if (sig2) break;
    sleep(SLEEP_TIME);
  }

  /* ... */

  return 0;
}

\end{verbatim}

The problem with this code is that there is a race condition in the
implementation of handler(). If handler() is called to handle SIGUSR1 and is
interrupted to handle SIGUSR2, it is possible that sig2 will not be set. This
non-compliant code example also violates SIG31-C. Do not access or modify shared
objects in signal handlers.

\subsection{Compliant Solution}

% write your compliant code subsection
This compliant solution registers two separate signal handlers to process
SIGUSR1 and SIGUSR2. The sig1\_handler() handler waits for SIGUSER1. After this
signal occurs, the sig2\_handler() is registered to handle SIGUSER2. This
solution is fully compliant and accomplishes the goal of detecting whether one
or more SIGUSR1 signals are followed by SIGUSR2.

\begin{verbatim}

% write your compliant code example
#include <signal.h>
#include <stdlib.h>
#include <string.h>

volatile sig_atomic_t sig1 = 0;
volatile sig_atomic_t sig2 = 0;

void sig1_handler(int signum) {
  sig1 = 1;
}

void sig2_handler(int signum) {
  sig2 = 1;
}

int main(void) {
  signal(SIGUSR1, handler);

  while (1) {
    if (sig1) break;
    sleep(SLEEP_TIME);
  }

  signal(SIGUSR2, handler);
  while (1) {
    if (sig2) break;
    sleep(SLEEP_TIME);
  }

  /* ... */

  return 0;
}

\end{verbatim}

\subsection{Mitigation Strategies}
\subsubsection{Static Analysis} 

Compliance with this rule can be checked using structural static analysis checkers using the following algorithm:

\begin{enumerate}
\item Write your checker algorithm
\end{enumerate}

\subsection{References}

% Write some references
% ex. \htmladdnormallink{ISO/IEC 9899-1999:TC2}{https://www.securecoding.cert.org/confluence/display/seccode/AA.+C+References} Forward, Section 6.9.1, Function definitions''

\htmladdnormallink{ISO/IEC 03}{https://www.securecoding.cert.org/confluence/display/seccode/SIG00-A.+Avoid+using+the+same+handler+for+multiple+signals} SIG00-A. Avoid using the same handler for multiple signals
