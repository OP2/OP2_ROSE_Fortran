% Liao, 2/25/2008. Changed the title from Call graph to Class Hierarchy Graph
\chapter{Generating the Class Hierarchy Graph}

\begin{figure}[!h]
{\indent
{\mySmallFontSize

\label{Tutorial:exampleBuildCH}

% Do this when processing latex to generate non-html (not using latex2html)
\begin{latexonly}
   \lstinputlisting{\TutorialExampleDirectory/classHierarchyGraph.C}
\end{latexonly}

% Do this when processing latex to build html (using latex2html)
\begin{htmlonly}
   \verbatiminput{\TutorialExampleDirectory/classHierarchyGraph.C}
\end{htmlonly}

% end of scope in font size
}
% End of scope in indentation
}
\caption{Example source code showing visualization of class hierarchy graph.}
\end{figure}

For C++, because of multiple inheritance, a class hierarchy graph is a directed graph 
with pointers from a class to a superclass. A superclass is a class which does not inherit
from any other class. A class may inherit from a superclass by inheriting from another
class which does rather than by a direct inheritance.

Figure~\ref{Tutorial:exampleBuildCH} shows the code required to generate
the class hierarchy graph for each class of an application.  Using the input code shown in
figure~\ref{Tutorial:exampleInputCode_BuildCH} the first function's call graph is
shown in figure~\ref{Tutorial:exampleBuildCHGraph}.

\begin{figure}[!h]
{\indent
{\mySmallFontSize

\label{Tutorial:exampleInputCode_BuildCH}

% Do this when processing latex to generate non-html (not using latex2html)
\begin{latexonly}
   \lstinputlisting{\TutorialExampleDirectory/inputCode_ClassHierarchyGraph.C}
\end{latexonly}

% Do this when processing latex to build html (using latex2html)
\begin{htmlonly}
   \verbatiminput{\TutorialExampleDirectory/inputCode_ClassHierarchyGraph.C}
\end{htmlonly}

% end of scope in font size
}
% End of scope in indentation
}
\caption{Example source code used as input to build class hierarchy graph.}
\end{figure}


\begin{figure}
% \centerline{\epsfig{file=\TutorialExampleBuildDirectory/callGraph.ps,
%                    height=1.3\linewidth,width=1.0\linewidth,angle=0}}
\includegraphics[scale=0.7]{\TutorialExampleBuildDirectory/classHierarchyGraph}
\caption{Class hierarchy graph in input code file: inputCode\_ClassHierarchyGraph.C.}
\label{Tutorial:exampleBuildCHGraph}
\end{figure}

   Figure~\ref{Tutorial:exampleBuildCHGraph} shows the class hierarchy graph for the
classes in the input code in figure~\ref{Tutorial:exampleInputCode_BuildCH}.



