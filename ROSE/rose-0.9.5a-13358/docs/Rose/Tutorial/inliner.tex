\chapter{Calling the Inliner}
\label{chap:inliner}

   Figure~\ref{Tutorial:example_inliningTransformation} shows an
example of how to use the inline mechanism.  This chapter presents
an example translator to to inlining of function calls where they
are called.  Such transformations are quite complex in a number of cases
(one case is shown in the input code; a function call in a for loop 
conditional test).  The details of functionality are hidden from the
user and a high level interface is provided.




\section{Source Code for Inliner}

    Figure~\ref{Tutorial:example_inliningTransformation}
shows an example translator which calls the inliner mechanism.
The code is designed to only inline up to ten functions.
the list of function calls is recomputed after any function call
is successfully inlined. 

The input code is shown in figure~\ref{Tutorial:exampleInputCode_inliningTransformation},
the output of this code is shown in 
figure~\ref{Tutorial:exampleOutput_inliningTransformation}.

\begin{figure}[!h]
{\indent
{\mySmallFontSize

% Do this when processing latex to generate non-html (not using latex2html)
\begin{latexonly}
   \lstinputlisting{\TutorialExampleDirectory/inlineTransformations.C}
\end{latexonly}

% Do this when processing latex to build html (using latex2html)
\begin{htmlonly}
   \verbatiminput{\TutorialExampleDirectory/inlineTransformations.C}
\end{htmlonly}

% end of scope in font size
}
% End of scope in indentation
}
\caption{Example source code showing how to instrument using Tau. }
\label{Tutorial:example_inliningTransformation}
\end{figure}





\section{Input to Demonstrate Function Inlining}

   Figure~\ref{Tutorial:exampleInputCode_inliningTransformation}
shows the example input used for demonstration of an inlining transformation.

\begin{figure}[!h]
{\indent
{\mySmallFontSize

% Do this when processing latex to generate non-html (not using latex2html)
\begin{latexonly}
   \lstinputlisting{\TutorialExampleDirectory/inputCode_inlineTransformations.C}
\end{latexonly}

% Do this when processing latex to build html (using latex2html)
\begin{htmlonly}
   \verbatiminput{\TutorialExampleDirectory/inputCode_inlineTransformations.C}
\end{htmlonly}

% end of scope in font size
}
% End of scope in indentation
}
\caption{Example source code used as input to program to the inlining transformation.}
\label{Tutorial:exampleInputCode_inliningTransformation}
\end{figure}





\section{Final Code After Function Inlining}

   Figure~\ref{Tutorial:exampleOutput_inliningTransformation} 
shows the results from the inlining of three function calls.
The first two function calls are the same, and trivial. The
second function call appears in the test of a for loop and is
more complex.


\begin{figure}[!h]
{\indent
{\mySmallFontSize

% Do this when processing latex to generate non-html (not using latex2html)
\begin{latexonly}
   \lstinputlisting{\TutorialExampleBuildDirectory/rose_inputCode_inlineTransformations.C}
\end{latexonly}

% Do this when processing latex to build html (using latex2html)
\begin{htmlonly}
   \verbatiminput{\TutorialExampleBuildDirectory/rose_inputCode_inlineTransformations.C}
\end{htmlonly}

% end of scope in font size
}
% End of scope in indentation
}
\caption{Output of input code after inlining transformations.}
\label{Tutorial:exampleOutput_inliningTransformation}
\end{figure}



























