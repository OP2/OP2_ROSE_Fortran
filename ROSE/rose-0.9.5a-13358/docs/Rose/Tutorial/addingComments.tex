\chapter{Handling Comments, Preprocessor Directives, And Adding Arbitrary Text to Generated Code}

\paragraph{What To Learn From These Examples}
Learn how to access existing comments and CPP directives
and modify programs to include new ones. Where such comments 
can be automated they can be used to explain transformations 
or for more complex transformations using other tools
designed to read the generated comments.  Also included is
how to add arbitrary text to the generated code (often useful
for embedded system programming to support back-end compiler 
specific functionality).

   This chapter deals with comments and preprocessor directives.
These are often dropped from compiler infrastructures and ignored
by make tools.  ROSE takes great care to preserve all comments and
preprocessor directives.  Where they exist in the input code we 
save them (note that EDG drops them from their AST) and weave them
back into the AST.  

Note that {\it \#pragma} is not a CPP directive and is part of the
C and C++ grammar, thus represented explicitly in the AST (see SgPragmaDeclaration).

\section{How to Access Comments and Preprocessor Directives}

   Comments and CPP directives are treated identically within ROSE and are 
saved as special preprocessor attributes to IR nodes within the AST.
Not all IR nodes can have these specific type of attributes, only 
SgLocatedNodes can be associated with such preprocessor attributes.
The more general {\it persistent attribute mechanism} within ROSE is separate from this
preprocessor attribute mechanism and is available on a wider selection of IR nodes.

\subsection{Source Code Showing How to Access Comments and Preprocessor Directives}

    Figure~\ref{Tutorial:example_collectComments}
shows an example translator which access the comments and preprocessor directives on each
statement. Note that in this example the AST is traversed twice, first header files are
ignored and then the full AST (including header files) are traversed (generated additional
comments).

The input code is shown in figure~\ref{Tutorial:exampleInputCode_collectComments},
the output of this code is shown in 
figure~\ref{Tutorial:exampleOutput_collectComments} for the source file only.
Figure~\ref{Tutorial:exampleOutput_collectComments_including_headers} shows the
same input code processed to output comments and preprocessor directives assembled from 
the source file {\em and} all header files.

\begin{figure}[!h]
{\indent
{\mySmallFontSize

% Do this when processing latex to generate non-html (not using latex2html)
\begin{latexonly}
   \lstinputlisting{\TutorialExampleDirectory/collectComments.C}
\end{latexonly}

% Do this when processing latex to build html (using latex2html)
\begin{htmlonly}
   \verbatiminput{\TutorialExampleDirectory/collectComments.C}
\end{htmlonly}

% end of scope in font size
}
% End of scope in indentation
}
\caption{Example source code showing how to access comments. }
\label{Tutorial:example_collectComments}
\end{figure}



\subsection{Input to example showing how to access comments and CPP directives}

   Figure~\ref{Tutorial:exampleInputCode_collectComments}
shows the example input used for demonstration of how to collect comments and CPP directives.

\begin{figure}[!h]
{\indent
{\mySmallFontSize

% Do this when processing latex to generate non-html (not using latex2html)
\begin{latexonly}
   \lstinputlisting{\TutorialExampleDirectory/inputCode_collectComments.C}
\end{latexonly}

% Do this when processing latex to build html (using latex2html)
\begin{htmlonly}
   \verbatiminput{\TutorialExampleDirectory/inputCode_collectComments.C}
\end{htmlonly}

% end of scope in font size
}
% End of scope in indentation
}
\caption{Example source code used as input to collection of comments and CPP directives.}
\label{Tutorial:exampleInputCode_collectComments}
\end{figure}



\subsection{Comments and CPP Directives collected from source file (skipping headers)}

   Figure~\ref{Tutorial:exampleOutput_collectComments} 
shows the results from the collection of comments and CPP directives within the input
source file only (without -rose:collectAllCommentsAndDirectives).

\begin{figure}[!h]
{\indent
{\mySmallFontSize

% Do this when processing latex to generate non-html (not using latex2html)
\begin{latexonly}
   \lstinputlisting{\TutorialExampleBuildDirectory/collectCommentsSkipHeaderFiles.out}
\end{latexonly}

% Do this when processing latex to build html (using latex2html)
\begin{htmlonly}
   \verbatiminput{\TutorialExampleBuildDirectory/collectCommentsSkipHeaderFiles.out}
\end{htmlonly}

% end of scope in font size
}
% End of scope in indentation
}
\caption{Output from collection of comments and CPP directives on the input source file only.}
\label{Tutorial:exampleOutput_collectComments}
\end{figure}


\subsection{Comments and CPP Directives collected from source file and all header files}

   Figure~\ref{Tutorial:exampleOutput_collectComments_including_headers} 
shows the results from the collection of comments and CPP directives within the input
source file and all headers (with -rose:collectAllCommentsAndDirectives).

\begin{figure}[!h]
{\indent
{\mySmallFontSize

% Do this when processing latex to generate non-html (not using latex2html)
\begin{latexonly}
   \lstinputlisting{\TutorialExampleBuildDirectory/collectCommentsAcrossHeaderFiles.out}
\end{latexonly}

% Do this when processing latex to build html (using latex2html)
\begin{htmlonly}
   \verbatiminput{\TutorialExampleBuildDirectory/collectCommentsAcrossHeaderFiles.out}
\end{htmlonly}

% end of scope in font size
}
% End of scope in indentation
}
\caption{Output from collection of comments and CPP directives on the input source file
    and all header files.}
\label{Tutorial:exampleOutput_collectComments_including_headers}
\end{figure}




\section{Collecting \#define C Preprocessor Directives}

   This example shows how to collect the \#define directives as
a list for later processing.

\subsection{Source Code Showing How to Collect \#define Directives}

    Figure~\ref{Tutorial:example_collectDefineDirectives}
shows an example translator which access the comments and preprocessor directives on each
statement. Note that in this example the AST is traversed twice, first header files are
ignored and then the full AST (including header files) are traversed (generated additional
comments).

The input code is shown in figure~\ref{Tutorial:exampleInputCode_collectDefineDirectives},
the output of this code is shown in 
Figure~\ref{Tutorial:exampleOutput_collectDefineDirectives} shows the
same input code processed to output comments and preprocessor directives assembled from 
the source file {\em and} all header files.

\begin{figure}[!h]
{\indent
{\mySmallFontSize

% Do this when processing latex to generate non-html (not using latex2html)
\begin{latexonly}
   \lstinputlisting{\TutorialExampleDirectory/collectDefineDirectives.C}
\end{latexonly}

% Do this when processing latex to build html (using latex2html)
\begin{htmlonly}
   \verbatiminput{\TutorialExampleDirectory/collectDefineDirectives.C}
\end{htmlonly}

% end of scope in font size
}
% End of scope in indentation
}
\caption{Example source code showing how to access comments. }
\label{Tutorial:example_collectDefineDirectives}
\end{figure}



\subsection{Input to example showing how to access comments and CPP directives}

   Figure~\ref{Tutorial:exampleInputCode_collectDefineDirectives}
shows the example input used for demonstration of how to collect comments and CPP directives.

\begin{figure}[!h]
{\indent
{\mySmallFontSize

% Do this when processing latex to generate non-html (not using latex2html)
\begin{latexonly}
   \lstinputlisting{\TutorialExampleDirectory/inputCode_collectDefineDirectives.C}
\end{latexonly}

% Do this when processing latex to build html (using latex2html)
\begin{htmlonly}
   \verbatiminput{\TutorialExampleDirectory/inputCode_collectDefineDirectives.C}
\end{htmlonly}

% end of scope in font size
}
% End of scope in indentation
}
\caption{Example source code used as input to collection of comments and CPP directives.}
\label{Tutorial:exampleInputCode_collectDefineDirectives}
\end{figure}


\subsection{Comments and CPP Directives collected from source file and all header files}

   Figure~\ref{Tutorial:exampleOutput_collectDefineDirectives} 
shows the results from the collection of comments and CPP directives within the input
source file and all headers (with -rose:collectAllCommentsAndDirectives).

\begin{figure}[!h]
{\indent
{\mySmallFontSize

% Do this when processing latex to generate non-html (not using latex2html)
\begin{latexonly}
   \lstinputlisting{\TutorialExampleBuildDirectory/collectDefineDirectives.out}
\end{latexonly}

% Do this when processing latex to build html (using latex2html)
\begin{htmlonly}
   \verbatiminput{\TutorialExampleBuildDirectory/collectDefineDirectives.out}
\end{htmlonly}

% end of scope in font size
}
% End of scope in indentation
}
\caption{Output from collection of comments and CPP directives on the input source file
    and all header files.}
\label{Tutorial:exampleOutput_collectDefineDirectives}
\end{figure}





\section{Automated Generation of Comments}
\label{sec:AutomatedGenerationOfComments}

   Figure~\ref{Tutorial:example_addComments} shows an
example of how to introduce comments into the AST which will
then show up in the generated source code. The purpose for this is
generally to add comments to where transformations are introduced.
If the code is read by the use the generated comments can be
useful in identifying, marking, and/or explaining the transformation.

This chapter presents an example translator which just introduces a 
comment at the top of each function.  The comment includes the
name of the function and indicates that the comment is automatically 
generated.

Where appropriate such techniques could be used to automate the 
generation of documentation templates in source code that would be
further filled in by the used.  In this case the automatically generated
templates would be put into the generated source code and a patch formed
between the generated source and the original source.  The patch could
be easily inspected and applied to the original source code to place
the documentation templates into the original source.  The skeleton
of the documentation in the source code could been be filled in by
the use.  The template would have all relevant information obtained by analysis
(function parameters, system functions used, security information, side-effects,
anything that could come from an analysis of the source code using ROSE).

\subsection{Source Code Showing Automated Comment Generation}

    Figure~\ref{Tutorial:example_addComments}
shows an example translator which calls the mechanism to 
add a comment to the IR node representing a function declaration (SgFunctionDeclaration).

The input code is shown in figure~\ref{Tutorial:exampleInputCode_addComments},
the output of this code is shown in 
figure~\ref{Tutorial:exampleOutput_addComments}.

\begin{figure}[!h]
{\indent
{\mySmallFontSize

% Do this when processing latex to generate non-html (not using latex2html)
\begin{latexonly}
   \lstinputlisting{\TutorialExampleDirectory/addComments.C}
\end{latexonly}

% Do this when processing latex to build html (using latex2html)
\begin{htmlonly}
   \verbatiminput{\TutorialExampleDirectory/addComments.C}
\end{htmlonly}

% end of scope in font size
}
% End of scope in indentation
}
\caption{Example source code showing how automate comments. }
\label{Tutorial:example_addComments}
\end{figure}



\subsection{Input to Automated Addition of Comments}

   Figure~\ref{Tutorial:exampleInputCode_addComments}
shows the example input used for demonstration of an automated commenting.

\begin{figure}[!h]
{\indent
{\mySmallFontSize

% Do this when processing latex to generate non-html (not using latex2html)
\begin{latexonly}
   \lstinputlisting{\TutorialExampleDirectory/inputCode_addComments.C}
\end{latexonly}

% Do this when processing latex to build html (using latex2html)
\begin{htmlonly}
   \verbatiminput{\TutorialExampleDirectory/inputCode_addComments.C}
\end{htmlonly}

% end of scope in font size
}
% End of scope in indentation
}
\caption{Example source code used as input to automate generation of comments.}
\label{Tutorial:exampleInputCode_addComments}
\end{figure}



\subsection{Final Code After Automatically Adding Comments}

   Figure~\ref{Tutorial:exampleOutput_addComments} 
shows the results from the addition of comments to the generated source code.

\begin{figure}[!h]
{\indent
{\mySmallFontSize

% Do this when processing latex to generate non-html (not using latex2html)
\begin{latexonly}
   \lstinputlisting{\TutorialExampleBuildDirectory/rose_inputCode_addComments.C}
\end{latexonly}

% Do this when processing latex to build html (using latex2html)
\begin{htmlonly}
   \verbatiminput{\TutorialExampleBuildDirectory/rose_inputCode_addComments.C}
\end{htmlonly}

% end of scope in font size
}
% End of scope in indentation
}
\caption{Output of input code after automating generation of comments.}
\label{Tutorial:exampleOutput_addComments}
\end{figure}



\section{Addition of Arbitrary Text to Unparsed Code Generation}

   This section is different from the comment generation 
(section~\ref{sec:AutomatedGenerationOfComments}) because it 
is more flexible and does not introduce any formatting.  It also
does not use the same internal mechanism, this mechanism supports
the addition of new strings or the replacement of the IR node
(where the string is attached) with the new string.  It is 
fundamentally lower level and a more powerful mechanism to
support generation of tailored output where more than 
comments, CPP directives, or AST transformation are required.
It is also much more dangerous to use.

   This mechanism is expected to be used rarely and sparingly
since no analysis of the AST is likely to leverage this mechanism 
and search for code that introduced as a transformation here.
Code introduced using this mechanism is for the most part
unanalyzable since it would have to be reparsed in the
context of the location in the AST were it is attached.
(Technically this is possible and is the subject of the existing
ROSE AST Rewrite mechanism, but that is a different subject).

   Figure~\ref{Tutorial:example_addArbitraryText} shows an example of 
how to introduce arbitrary text into the AST for output by the unparser 
which will then show up in the generated source code. The purpose for 
this is generally to add backend compiler or tool specific code 
generation which don't map to any formal language constructs and so 
cannot be represented in the AST.  However, since most tools that
require specialized annotations read them as comments, the mechanism
in the previous section~\ref{sec:AutomatedGenerationOfComments} may be more appropriate. It is 
because this is not always that case that we have provide this more 
general mechanism (often useful for embedded system compilers).

\subsection{Source Code Showing Automated Arbitrary Text Generation}

    Figure~\ref{Tutorial:example_addArbitraryText}
shows an example translator which calls the mechanism to 
add a arbitrary text to the IR node representing a function 
declaration (SgFunctionDeclaration).

The input code is shown in figure~\ref{Tutorial:exampleInputCode_addArbitraryText},
the output of this code is shown in 
figure~\ref{Tutorial:exampleOutput_addArbitraryText}.

\begin{figure}[!h]
{\indent
{\mySmallFontSize

% Do this when processing latex to generate non-html (not using latex2html)
\begin{latexonly}
   \lstinputlisting{\TutorialExampleDirectory/addArbitraryTextForUnparser.C}
\end{latexonly}

% Do this when processing latex to build html (using latex2html)
\begin{htmlonly}
   \verbatiminput{\TutorialExampleDirectory/addArbitraryTextForUnparser.C}
\end{htmlonly}

% end of scope in font size
}
% End of scope in indentation
}
\caption{Example source code showing how automate the introduction of arbitrary text. }
\label{Tutorial:example_addArbitraryText}
\end{figure}


\subsection{Input to Automated Addition of Arbitrary Text}

   Figure~\ref{Tutorial:exampleInputCode_addArbitraryText}
shows the example input used for demonstration of the automated introduction of text via
the unparser.

\begin{figure}[!h]
{\indent
{\mySmallFontSize

% Do this when processing latex to generate non-html (not using latex2html)
\begin{latexonly}
   \lstinputlisting{\TutorialExampleDirectory/inputCode_addArbitraryTextForUnparser.C}
\end{latexonly}

% Do this when processing latex to build html (using latex2html)
\begin{htmlonly}
   \verbatiminput{\TutorialExampleDirectory/inputCode_addArbitraryTextForUnparser.C}
\end{htmlonly}

% end of scope in font size
}
% End of scope in indentation
}
\caption{Example source code used as input to automate generation of arbitrary text.}
\label{Tutorial:exampleInputCode_addArbitraryText}
\end{figure}



\subsection{Final Code After Automatically Adding Arbitrary Text}

   Figure~\ref{Tutorial:exampleOutput_addArbitraryText} 
shows the results from the addition of arbitrary text to the generated source code.

\begin{figure}[!h]
{\indent
{\mySmallFontSize

% Do this when processing latex to generate non-html (not using latex2html)
\begin{latexonly}
   \lstinputlisting{\TutorialExampleBuildDirectory/rose_inputCode_addArbitraryTextForUnparser.C}
\end{latexonly}

% Do this when processing latex to build html (using latex2html)
\begin{htmlonly}
   \verbatiminput{\TutorialExampleBuildDirectory/rose_inputCode_addArbitraryTextForUnparser.C}
\end{htmlonly}

% end of scope in font size
}
% End of scope in indentation
}
\caption{Output of input code after automating generation of arbitrary text.}
\label{Tutorial:exampleOutput_addArbitraryText}
\end{figure}



























