\chapter{Code Coverage}

   This translator is part of ongoing collaboration with IBM on the support of code
coverage analysis tools for C, C++ and F90 applications.  the subject of code coverage is
much more complex than this example code would cover.  The following web site:
{\em http://www.bullseye.com/coverage.html} contains more information and is the source 
for the descriptions below. Code coverage can include:
\begin{itemize}
   \item Statement Coverage \\
      This measure reports whether each executable statement is encountered.
   \item Decision Coverage \\
      This measure reports whether boolean expressions tested in control structures 
      (such as the if-statement and while-statement) evaluated to both true and false. 
      The entire boolean expression is considered one true-or-false predicate regardless 
      of whether it contains logical-and or logical-or operators. Additionally, this
      measure includes coverage of switch-statement cases, exception handlers, and
      interrupt handlers.
   \item Condition Coverage \\
      Condition coverage reports the true or false outcome of each boolean sub-expression, 
      separated by logical-and and logical-or if they occur. Condition coverage measures 
      the sub-expressions independently of each other.
   \item Multiple Condition Coverage \\
      Multiple condition coverage reports whether every possible combination of boolean 
      sub-expressions occurs. As with condition coverage, the sub-expressions are
      separated by logical-and and logical-or, when present. The test cases required for 
      full multiple condition coverage of a condition are given by the logical operator
      truth table for the condition.
   \item Condition/Decision Coverage \\
      Condition/Decision Coverage is a hybrid measure composed by the union of condition
      coverage and decision coverage. This measure was created at Boeing and is required 
      for aviation software by RCTA/DO-178B.
   \item Modified Condition/Decision Coverage \\
      This measure requires enough test cases to verify every condition can affect the
      result of its encompassing decision. 
   \item Path Coverage \\
      This measure reports whether each of the possible paths in each function have been 
      followed. A path is a unique sequence of branches from the function entry to the
      exit.
   \item Function Coverage \\
      This measure reports whether you invoked each function or procedure. It is useful
      during preliminary testing to assure at least some coverage in all areas of the
      software. Broad, shallow testing finds gross deficiencies in a test suite quickly.
   \item Call Coverage \\
      This measure reports whether you executed each function call. The hypothesis is that
      faults commonly occur in interfaces between modules.
   \item Linear Code Sequence and Jump (LCSAJ) Coverage \\
      This variation of path coverage considers only sub-paths that can easily be
      represented in the program source code, without requiring a flow graph.
      An LCSAJ is a sequence of source code lines executed in sequence. This
      "linear" sequence can contain decisions as long as the control flow actually continues
      from one line to the next at run-time. Sub-paths are constructed by concatenating
      LCSAJs. Researchers refer to the coverage ratio of paths of length n LCSAJs as the
      test effectiveness ratio (TER) n+2.
   \item Data Flow Coverage \\
      This variation of path coverage considers only the sub-paths from variable
      assignments to subsequent references of the variables.
   \item Object Code Branch Coverage \\
      This measure reports whether each machine language conditional branch instruction
      both took the branch and fell through.
   \item Loop Coverage \\
      This measure reports whether you executed each loop body zero times, exactly once,
      and more than once (consecutively). For do-while loops, loop coverage reports whether
      you executed the body exactly once, and more than once. The valuable aspect of this
      measure is determining whether while-loops and for-loops execute more than once,
      information not reported by others measure.
   \item Race Coverage \\
      This measure reports whether multiple threads execute the same code at the same
      time. It helps detect failure to synchronize access to resources. It is useful for
      testing multi-threaded programs such as in an operating system.
   \item Relational Operator Coverage \\
      This measure reports whether boundary situations occur with relational operators (<,
      <=, >, >=). The hypothesis is that boundary test cases find off-by-one errors and
      mistaken uses of wrong relational operators such as < instead of <=. 
   \item Weak Mutation Coverage \\
      This measure is similar to relational operator coverage but much more general
      [Howden1982]. It reports whether test cases occur which would expose the use of wrong
      operators and also wrong operands. It works by reporting coverage of conditions
      derived by substituting (mutating) the program's expressions with alternate operators,
      such as "-" substituted for "+", and with alternate variables substituted.
   \item Table Coverage \\
      This measure indicates whether each entry in a particular array has been
      referenced. This is useful for programs that are controlled by a finite state machine.
\end{itemize}


   The rest of this text must be changed to refer to the code coverage example within ROSE/tutorial.

   Figure~\ref{Tutorial:exampleCodeCoverage} shows the low level
construction of a more complex AST fragment (a function declaration) and its insertion 
into the AST at the top of each block.  Note that the code does not handle 
symbol table issues, yet.

   Building a function in global scope.

\begin{figure}[!h]
{\indent
{\mySmallestFontSize

\label{Tutorial:exampleCodeCoverage}

% Do this when processing latex to generate non-html (not using latex2html)
\begin{latexonly}
%  \lstinputlisting{\TutorialExampleDirectory/addFunctionDeclaration.C}
   \lstinputlisting{\TutorialExampleBuildDirectory/codeCoverage.aa}
\end{latexonly}

% Do this when processing latex to build html (using latex2html)
\begin{htmlonly}
   \verbatiminput{\TutorialExampleDirectory/codeCoverage.C}
\end{htmlonly}

% end of scope in font size
}
% End of scope in indentation
}
\caption{Example source code shows instrumentation to call a test function from the top of
         each function body in the application (part 1).}
\end{figure}

\begin{figure}[!h]
{\indent
{\mySmallestFontSize

\label{Tutorial:exampleCodeCoverage2}

% Do this when processing latex to generate non-html (not using latex2html)
\begin{latexonly}
   \lstinputlisting{\TutorialExampleBuildDirectory/codeCoverage.ab}
\end{latexonly}

% Do this when processing latex to build html (using latex2html)
\begin{htmlonly}
%   \verbatiminput{\TutorialExampleDirectory/codeCoverage.C}
\end{htmlonly}

% end of scope in font size
}
% End of scope in indentation
}
\caption{Example source code shows instrumentation to call a test function from the top of
         each function body in the application (part 2).}
\end{figure}

\begin{figure}[!h]
{\indent
{\mySmallestFontSize

\label{Tutorial:exampleCodeCoverage2}

% Do this when processing latex to generate non-html (not using latex2html)
\begin{latexonly}
   \lstinputlisting{\TutorialExampleBuildDirectory/codeCoverage.ac}
\end{latexonly}

% Do this when processing latex to build html (using latex2html)
\begin{htmlonly}
%   \verbatiminput{\TutorialExampleDirectory/codeCoverage.C}
\end{htmlonly}

% end of scope in font size
}
% End of scope in indentation
}
\caption{Example source code shows instrumentation to call a test function from the top of
         each function body in the application (part 3).}
\end{figure}

\begin{figure}[!h]
{\indent
{\mySmallFontSize

\label{Tutorial:exampleInputCode_ExampleCodeCoverage}

% Do this when processing latex to generate non-html (not using latex2html)
\begin{latexonly}
   \lstinputlisting{\TutorialExampleDirectory/inputCode_ExampleCodeCoverage.C}
\end{latexonly}

% Do this when processing latex to build html (using latex2html)
\begin{htmlonly}
   \verbatiminput{\TutorialExampleDirectory/inputCode_ExampleCodeCoverage.C}
\end{htmlonly}

% end of scope in font size
}
% End of scope in indentation
}
\caption{Example source code used as input to translator adding new function.}
\end{figure}

\begin{figure}[!h]
{\indent
{\mySmallFontSize

\label{Tutorial:exampleOutput_ExampleCodeCoverage}

% Do this when processing latex to generate non-html (not using latex2html)
\begin{latexonly}
   \lstinputlisting{\TutorialExampleBuildDirectory/rose_inputCode_ExampleCodeCoverage.C}
\end{latexonly}

% Do this when processing latex to build html (using latex2html)
\begin{htmlonly}
   \verbatiminput{\TutorialExampleBuildDirectory/rose_inputCode_ExampleCodeCoverage.C}
\end{htmlonly}

% end of scope in font size
}
% End of scope in indentation
}
\caption{Output of input to translator adding new function.}
\end{figure}


