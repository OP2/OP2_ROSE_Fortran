
\def \beginfig
{
%\begin {minipage}{\figtextwidth}
%\renewcommand{\baselinestretch}{1}
}

\def \endfig
{
%\renewcommand{\baselinestretch}{2}
%\end {minipage}
}

\chapter {Loop Transformations}
\section {Introduction}

The loop transformation package implements all the algorithms published by Yi and 
Kennedy~\cite{MyThesis,YK:Report,YK:LACSI02}, including the transitive dependence
analysis algorithm by Yi, Adve, and Kennedy~\cite{YAK:PLDI00}.\footnote{ The package does not 
include the recursion transformation algorithm in this publication.}
These algorithms automatically optimize the
loop structures of applications for better performance. For now, the
implementation aims only to improve the cache locality of 
applications running on a single-processor machine. 
In the future, it can be expanded to optimize parallel applications
by maximizing the parallelism and minimizing the communication cost 
of loop structures~\cite{WL:TOPDS91,CFH:CPPSC95,AJMY:SC98}.

To optimize applications for better cache locality, this package applies the following 
loop transformations: interchange, fusion, fission(or distribution), and blocking (or tiling). 
The implementation can successfully optimize arbitrary loop structures,
including complex, non-perfect loop nests such as the one from LU factorization
with no pivoting in Figure~\ref {fig-lu}. The following examples illustrate the effect of
applying the transformations.

Figure~\ref {fig-mm} uses a pseudo code of {\it matrix multiplication} to illustrate 
the effect of applying the package to optimize perfect loop nests. The original code is in (a). 
After performing dependence analysis on this loop nest, the package applies
loop interchange transformation to improve the data reuse in caches (note that in 
C/C++ language, the matrix is stored in row-major order).
The transformed code is shown in (b). The cache locality of
this code can be further improved by loop blocking, and the result is shown in (c).



\begin {figure*}
\beginfig
{\footnotesize %\fontfamily{pcr} \selectfont
\begin {minipage}{.2\textwidth}
\begin {tabbing}
=\==\==\==\==\kill
\> for ($i = 0; i <= n-1; i+= 1$)  \\
\>\>  for ($j = 0; j <= n-1; j+= 1$) \\
\>\>\>  for ($k = 0; k <= N-1; k+=1$)  \\
\>\>\>\>  $c[i][j] = c[i][j] + a[i][k] * b[k][j]$; 
\end {tabbing}
\end {minipage}
\begin {minipage}{.2\textwidth}
\begin {tabbing}
=\==\==\==\==\kill
\> for ($i = 0; i <= n-1; i+= 1$)  \\
\>\>  for ($k = 0; k <= n-1; k+= 1$) \\
\>\>\>  for ($j = 0; j <= N-1; j+=1$)  \\
\>\>\>\>  $c[i][j] = c[i][j] + a[i][k] * b[k][j]$; 
\end {tabbing}
\end {minipage}
\begin {minipage}{.3\textwidth}
\begin {tabbing}
=\==\==\==\==\==\==\==\==\kill
\> for ($x_k = 0; x_k <= n-1; x_k += b$)  \\
\>\>  for ($x_j = 0; x_j <= n-1; x_j += b$)  \\
\>\>\>  for ($i = 0; i <= n-1; i += 1$)  \\
\>\>\>\>  for ($k = x_k; k <= min(n-1,x_k + b-1); k += 1$)  \\
\>\>\>\>\>  for ($j = x_j; j <= min(n-1,x_j + b-1); j += 1$)  \\
\>\>\>\>\>\>  $c[i][j] = c[i][j] + a[i][k] * b[k][j]$; 
\end {tabbing}
\end {minipage}
\begin {minipage}{\textwidth}
\hspace{.08\textwidth} (a) original code  \hspace{.1\textwidth}
(b) after loop interchange 
\hspace{.1\textwidth} (c) after loop blocking
\end {minipage}
}
\endfig
\caption {Optimizing matrix multiplication, first applying loop interchange
to arrange the best nesting order in (b), then applying blocking to 
exploit data reuses carried by $k$ and $j$ loops in (c).}
\label {fig-mm}
\end {figure*}



Figure~\ref {fig-lu} uses the pseudo code of {\it LU factorization without pivoting}
to illustrates the effect of 
applying the package to optimize complex, non-perfectly nested loop structures. 
Although the original loops in (a) are not perfectly nested, the package recognizes that 
the $k(s_1)$ loop ($k$ loop surrounding statement $s_1$) can be re-combined with 
the loop $j(s_2)$ and that the recombined
loop can then be placed outside of the original $k(s_2)$ loop. The transformed code in 
(b) simultaneously achieves two effects: the fusion of $k(s_1)$ with $j(s_2)$ loop, and
the interchange of $k(s_2)$ with $j(s_2)$ loop. 
Section~\ref {sec-hoisting} explains this combined-interchange-and-fusion
transformation in more detail. 
The code in (b) can further be blocked, and the result is shown in (c). 


\begin {figure*}
\beginfig
{\footnotesize %\fontfamily{pcr} \selectfont
\begin {minipage}{.2\textwidth}
\begin {tabbing}
=\==\==\==\==\==\==\==\kill
\> for ($k = 0; k <= n - 2; k+=1$) \\
\> \{ \\
\>\>  for ($i = k + 1; i <= n-1; i+=1$) \\
$s_1$:\>\>\>   $a[k][i] = a[k][i] / a[k][k]$; \\
\>\>   for ($j = k+1; j <= n-1; j+=1$) \\
\>\>\>     for ($i = k + 1; i <= n-1; i+=1$) \\
$s_2$:\>\>\>\>    $a[j][i] = a[j][i] - a[j][k] * a[k][i]$; \\
\> \}
\end {tabbing}
\end {minipage}
\begin {minipage}{.2\textwidth}
\begin {tabbing}
=\==\==\==\==\==\==\==\kill
\>  for ($j = 0; j <= n-1; j += 1$)  \\
\> \{ \\
\>\>  for ($k = 0; k <= j-1; k += 1$) \\
\>\>\>  for ($i = k+1; i <= n-1; i += 1$) \\
$s_2$:\>\>\>\>   $a[j][i] = a[j][i] - a[j][k] * a[k][i]$; \\
\>\>  for ($i = j+1; i <= n-1; i += 1$) \\
$s_1$\>\>\>     $a[j][i] = a[j][i] / a[j][j]$; \\
\> \}
\end {tabbing}
\end {minipage}
\begin {minipage}{.2\textwidth}
\begin {tabbing}
=\==\==\==\==\==\==\==\kill
\> for ($x_k = 0; x_k <= n-2; x_k += b$) \\
\>\>  for ($j = x_k; j <= n-1; j += 1$) \\
\>\> \{ \\
\>\>\>  for ($k = x_k; k <= min(x_k+b-1,j-1); k += 1$) \\
\>\>\>\>  for ($i = k+1; i <= n-1; i += 1$) \\
$s_2$:\>\>\>\>\>   $a[j][i] = a[j][i] - a[j][k] * a[k][i]$; \\
\>\>\>  for ($i = j+1; i <= n-1; i += 1$) \\
$s_1$:\>\>\>\>     $a[j][i] = a[j][i] / a[j][j]$; \\
\>\> \}
\end {tabbing}
\end {minipage}
\begin {minipage}{\textwidth}
\vspace{.2in}
\hspace{.08\textwidth} (a) original code \hspace{.1\textwidth}
(b) after loop interchange
\hspace{.1\textwidth} (c) after blocking row dimension
\end {minipage}
}
\endfig
\caption {Optimizing non-pivoting LU. In (b),
the $k(s_1)$ loop is fused with the $j(s_2)$ loop
and the fused loop is then put at the outermost position, 
achieving a combined interchange and fusion transformation;
the code in (c) achieves blocking in the row dimension of the matrix
through combined interchange, fusion and tiling transformations.}
\label {fig-lu}
\end {figure*}



Figure~\ref {fig-tridvpk} illustrates the effect of applying loop fusion to 
a sequence of loop nests in the subroutine $tridvpk$ of the application benchmark 
{\it Erlebacher} from ICASE.
The original code in (a) contains four separate loop nests, all of which can be fused 
into a single one. The package performs multiple levels of
loop fusion simultaneously using a combined-interchange-and-fusion 
transformation(see Section~\ref {sec-hoisting}), 
and the optimized code is shown in (b).



\begin {figure*}
\beginfig
{\footnotesize %\fontfamily{pcr} \selectfont
\begin {minipage}{.4\textwidth}
\begin {tabbing}
=\==\==\==\==\==\==\==\kill
\>      for ($ j=0; j <=n-1; j+=1$)\\
\>\>     for ($ i=0; i<=n-1; i+=1$)\\
\>\>\>         $duz[i][j][1] = duz[i][j][1]*b[1]$;\\
\>      for ($k=1; k<=n-2; k+=1$)\\
\>\>      for ($j = 0; j <= n-1; j+=1$)\\
\>\>\>      for ($i=0; i<=n-1; i+=1$)\\
\>\>\>\>         $duz[i][j][k]=(duz[i][j][k]-a[k]*duz[i][j][k-1])*b[k]$;\\
\>      for ($ j=0; j <=n-1; j+=1$)\\
\>\>      for ($ i=0; i<=n-1; i+=1$)\\
\>\>         $tot[i][j] = 0$;\\
\>      for ($ k=0; k<=n-2; k+=1$)\\
\>\>      for ($ j=0; j <=n-1; j+=1$)\\
\>\>\>      for ($ i=0; i<=n-1; i+=1$)\\
\>\>\>\>            $tot[i][j] = tot[i][j] + d[k]*duz[i][j][k]$;\\
\>      for ($ j=0; j <=n-1; j+=1$)\\
\>\>      for ($ i=0; i<=n-1; i+=1$)\\
\>\>\>         $duz[i][j][n-1] = (duz[i][j][n-1] - tot[i][j])*b[n-1]$;\\
\>      for ($ j=0; j <=n-1; j+=1$)\\
\>\>      for ($ i=0; i<=n-1; i+=1$)\\
\>\>\>         $duz[i][j][n-2]=duz[i][j][n-2] - e[n-2]*duz[i][j][n-1]$;\\
\>      for ($k=n-3; k>=0; k+=-1$)\\
\>\>      for ($j = 0; j <= n-1; j+=1$)\\
\>\>\>      for ($i=0; i<=n-1; i+=1$)\\
\>\>\>\>       $duz[i][j][k] =$ \= $duz[i][j][k] - c[k]*duz[i][j][k+1]$\\
\>\>\>\>\> $- e[k]*duz[i][j][n-1]$;\\
\end {tabbing}
\end {minipage}
%\begin {minipage}{.4\textwidth}
%\begin {tabbing}\\
%=\==\==\==\==\==\==\==\kill
%\>  for ($i = 0; i <= n-1; i += 1$)\\
%\>\>  for ($j = 0; j <= n-1; j += 1$)\\
%\>\>\>     $tot[i][j] = 0.0$;\\
%\>  for ($i = 0; i <= n-1; i += 1$)\\
%\>\>  for ($j = 0; j <= n-1; j += 1$)\\
%\>\>\>       $duz[i][j][1] = duz[i][j][1] * b[1]$;\\
%\>  for ($i = 0; i <= n-1; i += 1$)\\
%\>\>  for ($j = 0; j <= n-1; j += 1$)\\
%\>\>\>  for ($k = 1; k <= n-2; k += 1$)\\
%\>\>\>\>   $duz[i][j][k] = (duz[i][j][k] - a[k] * duz[i][j][k - 1]) * b[k]$;\\
%\>  for ($i = 0; i <= n-1; i += 1$)\\
%\>\>  for ($j = 0; j <= n-1; j += 1$)\\
%\>\>\>  for ($k = 0; k <= n-2; k += 1$)\\
%\>\>\>\>    $tot[i][j] = tot[i][j] + d[k] * duz[i][j][k]$;\\
%\> for ($i = 0; i <= n-1; i += 1$)\\
%\>\>  for ($j = 0; j <= n-1; j += 1$)\\
%\>\>\>    $duz[i][j][n-1] = (duz[i][j][n-1] - tot[i][j]) * b[n-1]$;\\
%\> for ($i = 0; i <= n-1; i += 1$)\\
%\>\>  for ($j = 0; j <= n-1; j += 1$)\\
%\>\>\>     $duz[i][j][n-2] = duz[i][j][n-2] - e[n-2] * duz[i][j][n-1]$;\\
%\>  for ($i = 0; i <= n-1; i += 1$)\\
%\>\>  for ($j = 0; j <= n-1; j += 1$)\\
%\>\>\>  for ($k = n-3; k >= 0; k += -1$)\\
%\>\>\>\>     $duz[i][j][k] =$ \= $(duz[i][j][k] - c[k] * duz[i][j])[k + 1])$ \\
%\>\>\>\>\>   $- e[k] * ((duz[i])[j])[n-1]$;\\
%\end {tabbing}
%\end {minipage}
\begin {minipage}{.4\textwidth}
\begin {tabbing}
=\==\==\==\==\==\==\==\kill
\>  for ($i = 0; i <= n-1; i += 1$)\\
\>  \{ \\
\>\>  for ($j = 0; j <= n-1; j += 1$)\\
\>\>  \{ \\
\>\>\>     $tot[i][j] = 0.0$;\\
\>\>\>     $duz[i][j][1] = duz[i][j][1] * b[1]$;\\
\>\>\>  for ($k = 1; k <= n-2; k += 1$)\\
\>\>\> \{ \\
\>\>\>\>   $duz[i][j][k] = (duz[i][j][k] - a[k] * duz[i][j][k - 1]) * b[k]$;\\
\>\>\>\>   $tot[i][j] = tot[i][j] + d[k] * duz[i][j][k]$;\\
\>\>\>  \} \\
\>\>\>     $duz[i][j][n-1] = (duz[i][j][n-1] - tot[i][j]) * b[n-1]$;\\
\>\>\>     $duz[i][j][n-2] = duz[i][j][n-2] - e[n-2] * duz[i][j][n-1]$;\\
\>\>\>  for ($k = n-3; k >= 0; k += -1$)\\
\>\>\>\>    $duz[i][j][k] = $ \= $(duz[i][j][k] - c[k] * duz[i][j])[k + 1])$ \\
\>\>\>\>\>  $- e[k] * ((duz[i])[j])[n-1]$;\\
\>\> \} \\
\> \} 
\end {tabbing}
\end {minipage}
\begin {minipage}{\textwidth}
\vspace{.2in}
\hspace{.08\textwidth} (a) original code \hspace{.5\textwidth}
(b) after fusion
\end {minipage}
}
\endfig
\caption {Optimizing $tridvpk$ from Erlebacher: 
combining loop interchange and fusion, thus
fusing multiple levels of loops simultaneously} 
\label {fig-tridvpk}
\end {figure*}



\section {Interface for End-Users and Compiler Developers}

This package is written in C++ language in a object-oriented style. 
It utilizes traditional techniques developed to optimize
loop nests in Fortran programs. 
When optimizing C or C++ applications, 
this package only recognizes and optimizes a particular for-loop
that corresponds to the $DO$ loop construct
in Fortran programs. 
Within the ROSE source-to-source compiler infrastructure,
such a loop is defined to have the following formats:
\begin {equation}
for \ (i = lb; i <= ub; i += positiveStep) \ \ or 
\ \ \ for \ (i = ub; i >= lb; i+=negativeStep)
\end {equation}
Here $i$ is an arbitrary integer variable, $lb$ and $ub$ are arbitrary
integer expressions, and $positiveStep$ and $negativeStep$ are positive
and negative integer expressions respectively.
To expand this definition, the user can rewrite the $LoopTransformInterface$ class 
within the package distribution (see Section~\ref {sec-interface-developer}) or
use a preprocessor within ROSE to translate all the non-Fortran loops into
the aforementioned formats.
Such a loop-normalization  preprocessor will be provided within ROSE.

The package distribution within ROSE also includes a loop optimization tool called
$LoopProcessor$, which automatically transforms the {Fortran loops} 
in C/C++ applications for better performance. In addition,
the package also provides two levels of internal user interfaces:
one for end users that intend to apply this package
to optimize their applications, and  one for compiler
developers that intend to extend this package for various purposes.

\subsection{End-User Interface}
\label {sec-interface-user}

The following function comprises the package interface for end users of
the ROSE source-to-source infrastructure,  which applies various traversal
and rewrite mechanisms to transform C++ applications
using the SAGE intermediate representation.
\begin {equation}
  Boolean \ \ SageLoopTransformation( unsigned \ \ argc, char** \ \ argv, SgGlobal* \ \ r, SgNode* \ \ n);
\end {equation}
Here both $SgGlobal$ and $SgNode$ are classes defined by the SAGE intermediate representation:
the $SgGlobal$ pointer $r$ represents the global root of an input program, 
and the $SgNode$ pointer $n$
represents the root of the input code fragment to be transformed by the package.
The parameters $argc$ and $argv$ represent command-line arguments that
instruct the package to adopt specific optimization strategies:
$argc$ contains the number of arguments,
and $argv$ contains the vector of $string$ arguments.

The package currently recognizes the following arguments:
\begin {itemize}
\item {-bk1 $<$blocksize$>$ :} apply outer-loop blocking for better data reuse
\item {-bk2 $<$blocksize$>$ :} apply inner-loop blocking for better data reuse
\item {-ic1 :} apply loop interchange for better data reuse
\item {-fs0 :} perform maximum loop distribution with no fusion afterwards
\item {-fs1 :} apply hierarchical single-level loop fusion for better data reuse
\item {-fs2 :} apply simultaneous multi-level loop fusion for better data reuse
\item {-tm :}  report timing information for each phase of the transformation package
\item {-ta $<$int$>$ :} set the maximum number of split nodes when performing transitive dependence 
                analysis
\item {-clsize $<$int$>$ :} set cache-line size for spatial reuse analysis
\end {itemize}

The loop transformation tool $LoopProcessor$ within ROSE recognizes 
these command-line arguments and then automatically selects the corresponding
optimization strategies.
When invoked with no argument, $LoopProcessor$ prints out 
usage information of this package.

\subsection{Developer Interface}
\label {sec-interface-developer}

Utilizing the available internal interface,
compiler developers can easily extend this package in two aspects. First,
they can rewrite the outside-interface classes of the implementation
to port it to a different compiler infrastructure (other than ROSE).
Second, they can provide their own profitability analysis  
to expand the transformation policy classes of the implementation.

\paragraph {Porting to a different compiler infrastructure} 
The package provides the following infrastructure-independent
interface to compiler developers.
\begin {equation}
AstNodePtr \ \ LoopTransformation( LoopTransformInterface\  \&interface, const \ AstNodePtr\ \&head);
\end {equation}
Here the class $LoopTransformInterface$  provides the interface for
accessing the intermediate representation of an arbitrary compiler,
and the pointer reference $AstNodePtr$ represents an arbitrary code fragment
to be transformed. Both classes, $AstNodePtr$ and $LoopTransformInterface$, 
need to be defined at location $outsideInterface/LoopTransformInterface.h$, 
which currently contains the ROSE implementation of these two classes.
By rewriting this file, a compiler developer can port the package
to a completely different infrastructure (this package already works under 
two compiler infrastructures: the ROSE C++ infrastructure and the DSystem Fortran infrastructure 
at Rice University~\cite{AM:PLDI98}).

\paragraph {Plugging in different profitability analysis algorithms}
This package provides a static configuration class, $LoopTransformOptions$
(defined in the location 
$driver$/$LoopTransformOptions.h$ of the package distribution), 
for plugging in different loop optimization policies.
This configuration class uses a set of policy classes 
(automatically selected from the command-line arguments, as 
described in Section~\ref {sec-interface-user})
to control
the application of three loop transformations: interchange, fusion and blocking.
The currently available policy classes
are defined in the locations $driver$/$InterchangeAnal.h$,
$driver$/$FusionAnal.h$ and $driver$/$BlockingAnal.h$ respectively.
To plug in different optimization strategies,
the developer can write new profitability policy classes
and then configure $LoopTransformOptions$ to use 
the new algorithms.  
The command-line configurations are automatically
extended when the developer registers these new policy classes.

\section {Analysis and Transformation Techniques}
This package implements the following techniques to 
optimize applications for better cache locality.
This section provides only brief introductions to the
algorithms without going into any detail.
Most algorithms are described in detail in Qing Yi's Ph.D.
thesis~\cite{MyThesis}.

\subsection {Dependence and Transitive Dependence Analysis}
Similar to most of the existing  loop optimizing compilers, 
this package models the safety requirement of loop transformations
using a dependence graph. The dependence graph includes all the statements of 
the input code segment as vertices, and a dependence edge is put from
statement $s_1$ to $s_2$ in the graph if $s_1$ must be executed
before $s_2$. If a statement reordering transformation does not 
reverse the direction of any dependence edge, the transformation
is guaranteed to preserve the original semantics of the program.
If two statements, $s_1$ and $s_2$, are both surrounded by loops,
for each dependence edge between $s_1$ and $s_2$, 
the dependence graph also defines a condition that must hold
between the iterations of these loops. 
The compiler then uses the dependence
relations to determine the safety of transforming these loops.

In traditional unimodular and single loop transformation systems,
the dependence relation between each pair of statements $s_1$ to $s_2$ is 
defined using a vector of direction or distance entries, where
each direction or distance entry defines the relation between
the iterations of a common loop surrounding both $s_1$ and $s_2$.
The compiler then uses these dependence vectors to determine
the safety of transforming a set of common loops that are perfectly
nested.

In order to effectively transform arbitrary, non-perfectly nested
loop structures, this package extends the traditional dependence model 
with a new dependence representation,
{\em Extended Direction Matrix(EDM)}.
Given two statements, $s_1$ and $s_2$, a dependence EDM from $s_1$
to $s_2$ defines a direction or distance entry for each pair of 
loops ($\ell_1$, $\ell_2$) s.t. $\ell_1$ surrounds $s_1$ and
$\ell_s$ surrounds $s_2$. This new dependence representation
thus defines dependence conditions for not only common loops
surrounding both $s_1$ and $s_2$, but also non-common loops that
surround only one of the two statements.

To compute the EDM representation of dependences, this package uses
an adapted Gaussian elimination algorithm to solve a set of integer linear
equations of loop induction variables.  
For each array access in the original input program, the algorithm
first constructs a set of linear equations based on the index expressions
of the array access.  If no loop induction variable has 
a symbolic coefficient in the array access expressions, such as the 
ones in the {\it Matrix Multiplication} code in
Figure~\ref {fig-mm} and the {\it non-pivoting LU} in Figure~\ref {fig-lu}, 
the algorithm is at least as powerful as the combined ZIV, SIV, and Delta dependence 
tests described by Allen and Kennedy~\cite{AK:Book, Wolfe:Book}. 
However, when loop induction variables do have symbolic coefficients, 
the algorithm assumes a conservative solution and is less precise 
than the symbolic solution algorithms described in~\cite{AK:Book, Wolfe:Book}.

This package also extends the traditional dependence model 
by implementing the transitive dependence analysis algorithm
published by Yi, Adve, and Kennedy~\cite{YAK:PLDI00}.
Note that although the algorithm is quite efficient in summarizing the complete
transitive dependence information between statements,
this package applies transitive dependence analysis only
when  transforming complex loop structures that cannot
be translated into sequences of perfectly nested loops.
Because the safety of transforming perfect loop nests
can be determined based on individual dependence edges
alone, it is often more economic to do without the extra
cost of transitive dependence analysis.
This package examines the original loop structures
of programs and performs transitive dependence analysis only
when required.

\subsection {Dependence Hoisting Transformation}
\label {sec-hoisting}

As the base technique for loop interchange, fusion
and blocking, this package implements a novel loop transformation,
{\em dependence hoisting} (first introduced by Yi and Kennedy~\cite{YK:Report}), 
that facilitates a combined fusion and interchange transformation for a group of 
arbitrarily nested loops.
Applying the dependence and transitive dependence analysis
algorithms,
this transformation first selects a group
of arbitrarily nested loops, such as the $k(s_1)$ ($k$ loop surrounding $s_1$)
and the $j(s_2)$ loops in the non-pivoting LU code in
Figure~\ref {fig-lu}(a), that can be legally fused
and then placed at the outermost position of a code
segment.
It then performs the transformation through a compound
sequence of traditional transformations on single loops
and perfectly nested loops. A combined
interchange and fusion transformation is established 
on an arbitrary loop structure as a result. An example of
the transformation result is shown for the non-pivoting
LU code in Figure~\ref {fig-lu}(b) (here the transformation
is applied to the $k(s_1)$ and $j(s_2)$ loops in (a)).

Given a group of loops as input for a dependence hoisting
transformation, the safety of fusing and shifting these loops
is determined from the dependence constraints on iterations
of these loops. If the group is a single loop in the original code,
such as the $i$, $j$ or $k$ loop in the matrix multiplication code
in Figure~\ref{fig-mm}, 
traditional loop interchange analysis for perfect loop nests would suffice;
however, if the group includes non-common loops surrounding different
statements, such as the $k(s_1)$ and $j(s_2)$ loops in the non-pivoting
LU code in Figure~\ref {fig-lu}(a), transitive dependence analysis is performed
on the dependence graph
and the transitive dependences are used to determine
the safety of fusing and shifting these loops. 

Because dependence hoisting is realized by combining
a sequence of traditional loop distribution, interchange
and index set splitting transformations on single
or perfectly nested loops,
the complexity of applying dependence hoisting
is equivalent to that of the corresponding sequence
of sub-transformations.
In the worst case, applying dependence hoisting to
a loop nest takes time proportional to $N^2 + L^2D$,
where $N$ is the number of statements in the nest,
$L$ is the depth of the nest,
and $D$ is the size of the dependence graph for the nest.
In an average case, however, dependence hoisting requires
much less time to finish. For a perfect loop nest,
dependence hoisting is equivalent to a standard
loop interchange on perfect loop nests
followed by a single-loop distribution, in which case
the required complexity is $O(N + D)$.

\subsection { Transformation Framework}

To optimize applications for better locality,
this package  uses {\it dependence hoisting} to
achieve three loop transformations: loop fusion,
interchange and blocking. 
It uses a construct, {\it computation slice} (or simply {\it slice}), 
to encode the input information necessary to perform
each dependence hoisting transformation.
For example, for the dependence
hoisting transformation on the non-pivoting LU code
from Figure~\ref {fig-lu}(a) to (b), the
computation slice contains two loops: $k(s_1)$ and $j(s_2)$.
Each computation slice must be valid in that
the corresponding dependence hoisting transformation
does not reverse any dependence direction of the original
program.

To model the memory performance of applications, this package
associates each computation slice with a floating point
number, which defines the number of array references that can be reused 
at each iteration of the slice,  that is, the number of references
that can be reused when the loops in the slice are placed
at the innermost position of a loop structure~\cite{AK:Book}.
Here the floating point number is necessary 
to model the spatial reuses resulted from references residing in the same cache line,
where in average less than one reference could be reused at each iteration
of the {\it slicing loops} (loops in the computation slice).
These floating point numbers provide the data reuse information of
computation slices to the transformation framework, which
then uses the information to guide loop interchange, fusion and blocking
transformations. 

Using the data reuse information of computation slices,
the transformation framework optimizes a code segment in the
following steps.
First, it applies dependence analysis and constructs all the legal computation slices
for an input code segment. It then treats all the
valid computation slices as if they form a sequence of loop nests and
rearranges these slices to achieve better cache locality.
For each set of computation slices
that forms a single loop nest, the package first selects a nesting order so that
the loops that are associated with more reuses are nested inside. 
It then fuses each pair of disjunct computation slices (slices that contain disjunct
sets of statements) when their statements access a common set of data. 
After fusion, if some non-innermost slices
carry data reuses, the package marks the corresponding slice nest to be tiled later.
Finally, the framework uses the rearranged computation slices to
perform a sequence of dependence hoisting
transformations to achieve the desired transformation result.
Note  that all the transformations are applied only when legal, that is, no semantics
of the original program is violated by the transformations.

The following briefly describes the optimization strategies 
implemented in this package. For more details of the optimization
algorithms, see~\cite{MyThesis}.

\paragraph {Loop Interchange and Blocking}
To achieve loop interchange,
the package carefully arranges the order of applying
dependence hoisting transformations using different computation
slices. 
Because each slice represents a set of loops that can
be fused into a single loop, interchanging the nesting order
of two slices corresponds directly to the interchange
of the two sets of slicing loops.
The effects of applying loop interchange is shown for
{\it matrix multiplication} in Figure~\ref {fig-mm}(b)
and for {\it non-pivoting LU factorization} in Figure~\ref {fig-lu}(b).

Because this package implements loop interchange
using dependence hoisting, 
it achieves loop blocking by
combining a sequence of dependence hoisting with loop
strip-mining transformation.
Given an input loop nest $C$,
the algorithm takes the computation slices constructed for $C$
in the reverse of their desired nesting order and 
then uses each slice to perform a dependence hoisting
transformation.
After each dependence hoisting
transformation, 
if the new outermost loop $\ell_f$ should be blocked,
the algorithm
strip-mines $\ell_f$ into a strip-counting 
loop $\ell_c$ and
a strip-enumerating loop $\ell_t$. It then uses loop $\ell_t$
as the input loop nest for further dependence hoisting
transformations, which in turn will shift
a new set of loops outside loop $\ell_t$ but inside loop $\ell_c$,
thus blocking loop $\ell_f$. 
The effects of applying loop blocking is shown for
{\it matrix multiplication} in Figure~\ref {fig-mm}(c)
and for {\it non-pivoting LU factorization} in Figure~\ref {fig-lu}(c).


\paragraph {Loop Fusion and Distribution (Fission)}

To achieve an aggressive multi-level
loop fusion effect, the package merges multiple computation
slices and then uses the merged slices to transform the original code.
Given two disjunct computation slices (two slices that
contain disjunct sets of statements), because each computation
slice fuses a set of loops that can be shifted to
the same loop level, fusing these two slices
automatically achieves the fusion of the loops
in both slices. For example, in Figure~\ref {fig-tridvpk},
after transformation analysis, the package constructs
a computation slice for each of the loops in the original code in (a).
It then performs fusion analysis and realizes that all the
$j$ slices (and thus all the $j$ loops) can be legally fused 
into a single loop. After merging these slices, it uses a single $j$ slice
to perform a dependence hoisting transformation and thus automatically
achieves the fusion of all the $j$ loops in (a). Similarly,
all the $i$ loops are also fused into a single loop, and two of the
$k$ loops are fused.

Because the original loop structure may need to be distributed
to achieve better performance, before applying loop fusion
analysis, this package first performs maximum loop fission
to distribute all the loop nests in the original code. The distributed loop nests
are then recombined during the loop fusion phase. This strategy 
ensures that both loop fission and fusion optimizations are
applied and that the final result of the optimization does not
depend on the original loop structure of the application.

\paragraph { Combined  Loop Interchange and Fusion} 
This package optimizes applications to improve the memory performance
of applications through a combined loop interchange and multi-level fusion 
strategy~\cite{YK:LACSI02}.
Since loop fusion is implemented in terms of merging computation slices,
given a code segment $C$ to optimize, the package first 
constructs all the valid computation slices. It then
applies loop interchange analysis to these slices 
to arrange the best nesting order for each loop nest in $C$. 
When applying fusion analysis to merge the disjunct computation slices,
it performs data reuse analysis and performs the actual fusion
only when loop fusion does not interfere with
loop interchange or when fusion is more favorable even if
it interferes with loop interchange.
Because multiple computation slices are constructed for each loop nests,
and all of these slices participate in the fusion analysis simultaneously, 
multiple loops may be fused for each loop nest in a single pass of fusion analysis.
As the result, this package achieves a combined loop interchange and multi-level
fusion optimization for a collection of loop nests.
For example, in Figure~\ref{fig-tridvpk}, even though the $j$ and $i$
loops are nested at different levels in the original code in (a),
the package successfully achieves the fusion of these loops because 
all the loops are collected as computation slices
in a single pass and together they participate in the fusion analysis.

\subsection {Profitability Analysis}

This package separates the profitability analysis of loop transformations
from the actual transformations by encoding profitability analysis
algorithms in a set of policy classes and then using these policy classes to control
the application of loop transformations. A flexible internal interface
is provided for compiler writers to plug in their own performance model
for various optimization purposes (see Section~\ref {sec-interface-developer}).

The currently available performance model includes only the counting of
array references being reused, including both temporary and spatial cache reuses.
Because the package has not yet implemented the calculation of the working set size of 
each loop body, 
it cannot automatically decide the tile size for each blocked loop nest. Similarly,
because the current data reuse analysis is insufficient in calculating the trade-off between
outer-loop blocking and inner-loop blocking, the package asks the user to specify the desired
strategy. It then applies the specified strategy uniformly for all the loop nests. 

The profitability analysis algorithms within this package are not yet complete 
and will incorporate more sophisticated algorithms in the future.
These algorithms include not only various strategies to automate the decision of
blocking parameters, but also runtime tuning strategies that execute applications 
on a specific machine and then use the collected performance information to automatically 
select the best overall transformations.

