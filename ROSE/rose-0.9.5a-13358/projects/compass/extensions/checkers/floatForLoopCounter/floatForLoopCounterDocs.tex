% 9.5.07
% This is a sample documentation for Compass in the tex format.
% We restrict the use of tex to the following subset of commands:
%
% \section, \subsection, \subsubsection, \paragraph
% \begin{enumerate} (no-nesting), \begin{quote}, \item
% {\tt ... }, {\bf ...}, {\it ... }
% \htmladdnormallink{}{}
% \begin{verbatim}...\end{verbatim} is reserved for code segments
% ...''
%

\section{Float For Loop Counter}
\label{FloatForLoopCounter::overview}
``CERT Secure Coding'' states
\begin{quote}
Floating point arithmetic is inexact and is subject to rounding errors. Hence, floating point variables should not be used as loop counters.
\end{quote}

\subsection{Parameter Requirements}
This checker takes no parameters and inputs source file.

\subsection{Implementation}
This pattern is checked using a simple AST traversal that visits all for loop init statement nodes and checks the type of its counter variable declaration. If that type is {\tt float} or {\tt double} then flag a violation.

\subsection{Non-Compliant Code Example}

\begin{verbatim}
for (float count = 0.1f; count <= 1; count += 0.1f)
{
}
\end{verbatim}

\subsection{Compliant Solution}
The compliant solution uses an {\tt int} type loop counter.

\begin{verbatim}
for (int count = 1; count <= 10; count += 1) 
{
}
\end{verbatim}

\subsection{Mitigation Strategies}
\subsubsection{Static Analysis} 

Compliance with this rule can be checked using structural static analysis checkers using the following algorithm:

\begin{enumerate}
\item Perform simple AST traversal visiting all for loop initialization statement nodes.
\item For each node check the type of its variable declaration. If type is {\tt float} or {\tt double} then flag violation.
\item Report any violations. 
\end{enumerate}

\subsection{References}

\htmladdnormallink{FLP31-C. Do not use floating point variables as loop counters}{https://www.securecoding.cert.org/confluence/display/cplusplus/FLP31-C.+Do+not+use+floating+point+variables+as+loop+counters}
