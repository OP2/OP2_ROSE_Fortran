% 9.5.07
% This is a sample documentation for Compass in the tex format.
% We restrict the use of tex to the following subset of commands:
%
% \section, \subsection, \subsubsection, \paragraph
% \begin{enumerate} (no-nesting), \begin{quote}, \item
% {\tt ... }, {\bf ...}, {\it ... }
% \htmladdnormallink{}{}
% \begin{verbatim}...\end{verbatim} is reserved for code segments
% ...''
%

\section{Push Back}
\label{PushBack::overview}

Tests if the source uses front or back insertion in sequences using
insert or resize where a \texttt{push\_front} or \texttt{push\_back}
could be used. The patterns are very simple and matches simple calls
like \texttt{vector.insert(vector.end(), ...)}.

In these case, \texttt{push\_back} and \texttt{push\_front} only are
insured to be efficient. All other calls may be quadratic.

%%FIXME: use bibtex
This test is inspired by the rule 80 of C++ Coding Standards: ``Use the
accepted idioms to really shrink capacity and really erase elements''.

\subsection{Parameter Requirements}

There is no parameter required.

\subsection{Implementation}

%Details of the implementation go here.
   No implementation yet!

\subsection{Non-Compliant Code Example}

\begin{verbatim}
#include <vector>
#include <list>

void g()
{
  std::vector<int> v;
  v.insert(v.end(), 1);

  v.resize(v.size() + 1, 1);

  std::list<int>* vv = new std::list<int>();
  vv->insert(vv->begin(), 1);

}
\end{verbatim}

\subsection{Compliant Solution}

\begin{verbatim}
#include <vector>
#include <list>

void g()
{
  std::vector<int> v;
  v.push_back(1);
  v.push_back(1);

  std::list<int>* vv = new std::list<int>();
  vv->push_front(1);

}
\end{verbatim}

\subsection{Mitigation Strategies}
\subsubsection{Static Analysis}

Compliance with this rule can be checked using structural static
analysis checkers looking for these patterns:

For all types \texttt{T},\\
where \texttt{V} variable of type \texttt{vector<T>},\\
where \texttt{Vp} variable of type \texttt{vector<T>*},\\
where \texttt{L} variable of type \texttt{list<T>},\\
where \texttt{Lp} variable of type \texttt{list<T>*},\\
where \texttt{S} variable of type \texttt{slist<T>},\\
where \texttt{Sp} variable of type \texttt{slist<T>*},\\
where \texttt{D} variable of type \texttt{deque<T>},\\
where \texttt{Dp} variable of type \texttt{deque<T>*},\\
where \texttt{Value} expression of type \texttt{T},\\
the check will look for these patterns in the source code.

\begin{verbatim}
V.resize(V.size() + 1, Value)
Vp->resize(Vp->size() + 1, Value)
D.resize(D.size() + 1, Value)
Dp->resize(Dp->size() + 1, Value)
L.resize(L.size() + 1, Value)
Lp->resize(Lp->size() + 1, Value)
V.resize(1 + V.size(), Value)
Vp->resize(1 + Vp->size(), Value)
D.resize(1 + D.size(), Value)
Dp->resize(1 + Dp->size(), Value)
L.resize(1 + L.size(), Value)
Lp->resize(1 + Lp->size(), Value)
V.insert(V.end(), Value)
Vp->insert(Vp->end(), Value)
D.insert(D.end(), Value)
Dp->insert(Dp->end(), Value)
L.insert(L.end(), Value)
Lp->insert(Lp->end(), Value)
S.insert(S.begin(), Value)
Sp->insert(Sp->begin(), Value)
D.insert(D.begin(), Value)
Dp->insert(Dp->begin(), Value)
L.insert(L.begin(), Value)
Lp->insert(Lp->begin(), Value)
\end{verbatim}

For all \texttt{resize} and back \texttt{insert} patterns, \texttt{push\_back}
could be used. For front \texttt{insert} patterns, \texttt{push\_front} could
be used instead.

\subsection{References}

Alexandrescu A. and Sutter H. {\it C++ Coding Standards 101 Rules, Guidelines, and Best Practices}. Addison-Wesley 2005.